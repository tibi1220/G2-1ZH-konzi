\documentclass[xcolor={table}]{beamer}

\usepackage[magyar]{babel}

\title{Matematika G2 Első Zárthelyi Konzultáció}
\author{Sándor Tibor}
\date{\today}

\usepackage{amsmath,amssymb}
\usepackage{unicode-math}
\usepackage[table]{xcolor}
\usepackage{array}
\usepackage{multicol}
\usepackage{icomma}
\usepackage{tabto}

\newcolumntype{x}[1]{>{\centering\arraybackslash\hspace{0pt}}p{#1}}
\newcolumntype{X}[1]{>{$}x{#1}<{$}}

\newenvironment{bamatrix}[2]{\left[\begin{array}{*{#1}{X{#2}}}}{\end{array}\right]}
\newcommand{\gcc}[1]{\cellcolor{gray!25}{#1}}

\newcommand\iu{\mathbf{i}}
\newcommand{\nvec}{\text{\textbf{\textit{0}}}}
\newcommand\imat{{\mathbb{E}}}
\newcommand\nmat{{\mathbb{O}}}
\newcommand{\rvec}[1]{\mathbfit{#1}}
\newcommand{\uvec}[1]{\widehat{\mathbfit{#1}}}
\newcommand{\rmat}[1]{\mathbf{#1}}
\newcommand{\edet}[1]{\det \begin{pmatrix} \dots & #1 & \dots \end{pmatrix}}
\DeclareMathOperator{\rg}{rg}
\DeclareMathOperator{\adj}{adj}
\DeclareMathOperator{\defect}{def}
\let\dim\relax
\DeclareMathOperator{\dim}{dim}

\usepackage[stixtwo]{fontsetup}

\usepackage{tikz, pgfplots}
\usetikzlibrary{calc, matrix}
\pgfplotsset{compat=1.18}
\pgfkeys{/pgf/plot/gnuplot call={cd build && gnuplot}}

\usepackage{mystyles}

% Beamer general settings
\usetheme{Frankfurt}
\institute{Mechatronika szakosztály}
% \logo{\includegraphics[height=1cm]{mszo-logo.pdf}}

\mode<presentation>
\setbeamercovered{transparent}
\usefonttheme[onlymath]{serif}

\usepackage{ragged2e}
\addtobeamertemplate{block begin}{}{\justifying}

% Document begins here
\begin{document}

% Titlepage
\frame{\titlepage}

% About -- the structure of the consultation
\begin{frame}
  \frametitle{A konzultáció felépítése}
  \tableofcontents
\end{frame}

% Main elements
\section{Vektorterek}
\begin{frame}
  \frametitle{Vektorterek}
  \framesubtitle{Definíció}

  Legyen $V$ nem üreshalmaz, és $+; \lambda$ két művelet, valamint $T$ test.
  $(V; +; \lambda)$ a $T$ test feletti vektortér, ha az alábbiak teljesülnek:
  \begin{itemize}
    \def\arraystretch{1.2}
    \item $(V; +)$ Abel csoport:\\[1mm]
          \begin{tabular}{p{35mm} l}
            -- asszociatív:        &
            $\rvec a + (\rvec b + \rvec c) = (\rvec a + \rvec b) + \rvec c$,
            \\
            -- kommutatív:         &
            $\rvec a + \rvec b = \rvec b + \rvec a$,
            \\
            -- létezik zérus elem: &
            $\exists \nvec \in V \text{, melyre } \rvec a + \nvec = \rvec a$,
            \\
            -- létezik inverz:     &
            $\forall \rvec a \text{-re } \exists - \rvec a \text{, hogy } \rvec v + (- \rvec v) = \nvec$.
          \end{tabular}
    \item $(V; \lambda)$-ra pedig igaz:\\[1mm]
          \begin{tabular}{p{35mm} l}
            -- asszociatív:      & $(\alpha \beta) \rvec a = \alpha (\beta \rvec a)$.
            \\
            -- egységelem:       & $1 \in T \text{-re } 1\rvec a = \rvec a$,
            \\
            -- disztributivitás: & $\alpha(\rvec a + \rvec b) = \alpha \rvec a + \alpha \rvec b$,
            \\
                                 & $(\alpha + \beta) \rvec a = \alpha \rvec a + \beta \rvec a$.
          \end{tabular}
  \end{itemize}
\end{frame}

\begin{frame}
  \frametitle{Vektorterek}
  \framesubtitle{Vektorteres feladat}

  \begin{exercise}{%
    Döntsük el, hogy alteret alkotnak-e az alábbi számhármasok
    $\mathbb R^3$-ban?
  }
  \newcommand{\setmap}[2]{\Big\{\, #1 \,\Big|\, #2 \,\Big\}}
  \begin{enumerate}[a)]
    \item $Q_1 = \setmap{(x_1; x_2; x_3)}{x_1 + x_2 = 0}$
    \item $Q_2 = \setmap{(x_1; x_2; x_3)}{x_1 = \pi}$
    \item $Q_3 = \setmap{(x_1; x_2; x_3)}{x_1 = x_2 = x_3}$
    \item $Q_3 = \setmap{(x_1; x_2; x_3)}{x_1 = (x_2)^2}$
  \end{enumerate}

  \exsol{%
    \begin{enumerate}[a)]
      \item Igen, mert a műveletek sosem mutatnak ki a vektortérből.
            %    , hiszen ha\\
            % $\alpha(x_{11} + x_{12}) = 0$,
            % akkor $\alpha x_{11} + \alpha x_{12} = 0$.
      \item Nem, hiszen $\pi + \pi \neq \pi$.
      \item Igen, hiszen a műveletek sosem mutatnak ki a vektortérből.
      \item Nem, hiszen $(a + b)^2 \neq a^2 + b^2$.
    \end{enumerate}
  }
\end{exercise}

\end{frame}

\begin{frame}
  \frametitle{Vektorterek}
  \framesubtitle{További definíció}

  \begin{block}{Altér}
    Legyen $(V; +; \lambda)$ a $T$ test feletti vektortér, és
    $\emptyset \neq L \subset V$. $L$-t altérnek nevezzük $V$-ben, ha
    $(L; +; \lambda)$ ugyancsak vektortér.
  \end{block}

  \begin{block}{Generátorrendszer}
    Legyen $\emptyset \neq G \subset V$. Ekkor $G$ által generált altérnek
    nevezzük azt a legszűkebb alteret, amely tartalmazza $G$-t. Ha ez az
    altér maga $V$, akkor $G$ generátorrendszere $V$-nek. ($\mathcal L(G)=V$)
  \end{block}

  \begin{block}{Bázis}
    A $V$ vektortér egy lineárisan független generátorrendszerét a $V$ bázisának
    hívjuk.
  \end{block}
\end{frame}

\begin{frame}
  \frametitle{Vektorterek}
  \framesubtitle{Generátoros feladat}

  \begin{exercise}{%
    Alkotnak-e generátorrendszert vagy bázist az alábbi vektorok $\mathbb R^2$-ban
  }
  \begin{center}
    \begin{tikzpicture}[ultra thick]
      \foreach \i/\j/\l in {0/0/a,1/0/b,0/1/c,1/1/d}{
          \draw[draw=gray, dashed, rounded corners, thick]
          (\i*5cm,-\j*3cm)
          node[below right] {\l)}
          rectangle ++(4.5cm,-2.5cm)
          coordinate [midway] (\i\j);

          \draw[-latex, cyan!40!black] ($(\i\j)-(0,7.5mm)$)
          coordinate (\i\j)
          -- ++(1.5,0);
        }

      \begin{scope}[-latex, yellow!40!black]
        \draw (00) -- ++(0,1.5);
        \draw (10) -- ++(-1.5/1.41,1.5/1.41);
        \draw (01) -- ++(-1.5/1.41,1.5/1.41);
        \draw (11) -- ++(-1.5,0);

        \draw[red!40!black] (01) -- ++(0.574,1.386);
      \end{scope}
    \end{tikzpicture}
  \end{center}

  \exsol{%
    \begin{enumerate}[a)]
      \item Bázist és generátorrendszert is alkotnak.
      \item Bázist és generátorrendszert is alkotnak.
      \item Csak generátorrendszert alkotnak, hiszen nem lineárisan függetlenek.
      \item Sem bázist, sem generátorrendszert nem alkotnak.
    \end{enumerate}
  }
\end{exercise}

\end{frame}

\section{Mátrixok}
\begin{frame}
  \frametitle{Mátrixok}
  \framesubtitle{Alapfogalmak}

  \vfill

  \begin{block}{Mátrix}
    Az $m$ sorba és $n$ oszlopba rendezett rendezett számokat mátrixoknak
    nevezzük.
  \end{block}

  \vfill

  \begin{block}{Transzponált}
    Egy mátrix transzponáltja a főátlóra való tükörképe. Jele: $A^\mathsf T$.
  \end{block}

  \vfill

  \begin{block}{Szimmetrikus mátrix}
    Ha $A = A^\mathsf T$, akkor a mátrix szimmetrikus.
  \end{block}

  \vfill

  \begin{block}{Antiszimmetrikus mátrix}
    Ha $A = -A^\mathsf T$, akkor a mátrix antiszimmetrikus.
  \end{block}

  \vfill
\end{frame}

\begin{frame}
  \frametitle{Mátrixok}
  \framesubtitle{Elemi mátrixműveletek}

  \vfill

  \begin{block}{Összeadás}
    Ha $\rmat A; \rmat B \in \mathbb R^{n \times k}$, akkor az összegükön a
    megfelelő elempárok összeadásával keletkező mátrixot értjük.
  \end{block}

  \vfill

  \begin{block}{Skalárral való szorzás}
    Egy mátrix és egy skalár szorzata olyan mátrix, melynek minden eleme
    skalárszorosa az eredeti mátrix elemeinek.
  \end{block}

  \vfill
\end{frame}

\begin{frame}
  \frametitle{Mátrixok}
  \framesubtitle{Elemi mátrixműveletek}

  \begin{block}{Mátrix szorzás
      -- asszociatív, disztributív, de nem kommutatív!}
    \[
      \left.\begin{array}{ll}
        \rmat A \in \mathbb R^{m \times n} \\
        \rmat B \in \mathbb R^{n \times p}
      \end{array}\right\}
      \; \rightarrow \;
      \rmat A \cdot \rmat B \in \mathbb R^{m \times p}
    \]

    \def\arraystretch{1.1}
    \begin{align*}
       & \left[\begin{array}{X{2cm}cX{2cm}}
                   b_{11} & \dots  & b_{1p} \\
                   b_{21} & \dots  & b_{2p} \\
                   \vdots & \ddots & \vdots \\
                   b_{n1} & \dots  & b_{np}
                 \end{array}\right]
      \\
      \left[\begin{array}{cccc}
                a_{11} & a_{12} & \dots  & a_{1n} \\
                a_{21} & a_{22} & \dots  & a_{2n} \\
                \vdots & \vdots & \ddots & \vdots \\
                a_{m1} & a_{m2} & \dots  & a_{mn}
              \end{array}\right]
       & \left[\begin{array}{X{2cm}cX{2cm}}
                   \sum a_{1i} b_{i1} & \dots & \sum a_{1i} b_{ip} \\
                   \sum a_{2i} b_{i1} & \dots & \sum a_{2i} b_{ip} \\
                   \vdots             & \dots & \vdots             \\
                   \sum a_{mi} b_{i1} & \dots & \sum a_{mi} b_{ip}
                 \end{array}\right]
    \end{align*}
  \end{block}
\end{frame}

\begin{frame}
  \frametitle{Mátrixok}
  \framesubtitle{Determináns}

  \begin{block}{Kifejtési tétel -- előjelszabály!}
    \begin{align*}
      \begin{vmatrix}
        + & - & + \\
        - & + & - \\
        + & - & + \\
      \end{vmatrix}
      \; \rightarrow \;
      \begin{vmatrix}
        a & b & c \\
        d & e & f \\
        g & h & i \\
      \end{vmatrix}
       & = a \begin{vmatrix}
               e & f \\ h & i
             \end{vmatrix}
      - b \begin{vmatrix}
            d & f \\ g & i
          \end{vmatrix}
      + c \begin{vmatrix}
            d & e \\ g & h
          \end{vmatrix}
      \\
       & = a (ei - hf) - b(di - gf) + c(dh - eg)
    \end{align*}
  \end{block}

  \begin{block}{Sarrus-szabály -- csak ($3 \times 3$)-as mátrixoknál!}
    \centering
    \begin{tikzpicture}[ampersand replacement=\&]
      \matrix[
        matrix of math nodes,
        column sep=2mm,
      ] (sarrus) {
        a\vphantom{b} \& b \& c\vphantom{b} \& a \& b \\
        d \& e \& f \& d \& e                         \\
        g \& h\vphantom{g} \& i\vphantom{g} \& g \& h \\
      };

      \draw[red!40!gray, ultra thick, opacity=.5]
      (sarrus-1-1.center) -- (sarrus-3-3.center)
      (sarrus-1-2.center) -- (sarrus-3-4.center)
      (sarrus-1-3.center) -- (sarrus-3-5.center)
      ;

      \draw[blue!40!gray, ultra thick, opacity=.5]
      (sarrus-3-1.center) -- (sarrus-1-3.center)
      (sarrus-3-2.center) -- (sarrus-1-4.center)
      (sarrus-3-3.center) -- (sarrus-1-5.center)
      ;

      \draw[black, thick]
      (sarrus-1-1.north west) -- (sarrus-3-1.south west)
      (sarrus-1-3.north east) -- (sarrus-3-3.south east)
      ;

      \foreach \i in {1,2,3}{
          \node[above=-2.5mm, red!40!gray] at (sarrus-1-\i.north) {$+$};
          \node[below=-1.5mm, blue!40!gray] at (sarrus-3-\i.south) {$-$};
        }

      \node[] at (5,.25) {$\det \rmat A = + aei + bfg + cdh$};
      \node[] at (5,-.25) {$\phantom{\det \rmat A =} - gec - hfa - idb$};
    \end{tikzpicture}
  \end{block}
\end{frame}

\begin{frame}
  \frametitle{Mátrixok}
  \framesubtitle{Rang}

  \begin{block}{Mátrix rangja}
    A mátrix rangjának nevezzük az oszlopvektorai közül a lineárisan függetlenek
    maximális számát. A mátrix rangja elemi átalakítások során nem változik.
    \begin{itemize}
      \item tetszőleges sorát vagy oszlopát egy 0-tól különböző számmal
            megszorozzuk,
      \item tetszőleges sorát vagy oszlopát felcseréljük,
      \item tetszőleges sorához vagy oszlopához egy másik tetszőleges sorát
            vagy oszlopát adjuk.
    \end{itemize}
  \end{block}
\end{frame}

\begin{frame}
  \frametitle{Mátrixok}
  \framesubtitle{Mátrixos feladatok}

  \vfill
  \begin{exercise}{Határozzuk meg az alábbi mátrixok determinánsát!}
  \[
    \rmat A = \begin{bmatrix}
      1 & 4 & 8 \\
      0 & 1 & 0 \\
      2 & 4 & 6
    \end{bmatrix}
    \hspace{2cm}
    \rmat B = \begin{bmatrix}
      a  & 3  & x  \\
      -a & -2 & x  \\
      0  & 1  & 2x \\
    \end{bmatrix}
  \]

  \exsol{
    Határozzuk meg az $\rmat A$ mátrix determinánsát úgy, a kifejtési tétel
    segítségével. Mivel a második sor csak 1 nemzérus elemet tartalmaz, ezért
    innen fogunk kiindulni:
    \[
      \det \rmat A
      = \begin{vmatrix}
        1 & 4 & 8 \\
        0 & 1 & 0 \\
        2 & 4 & 6
      \end{vmatrix}
      = 1 \cdot \begin{vmatrix}
        1 & 8 \\
        2 & 6
      \end{vmatrix}
      = 1 \cdot (1 \cdot 6 - 2 \cdot 8)
      = -10
      \text.
    \]

    \tcbline

    A $\rmat B$ mátrix esetében vegyük észre, hogy a harmadik sor elemei
    pont az előző két sor megfelelő elemeinek összegei. Ebből következik,
    hogy a mátrix rangja nem maximális, vagyis $\det \rmat B = 0$.
  }
\end{exercise}

  \vfill
  \begin{exercise}{Határozzuk meg az alábbi mátrixok rangjait!}
  \[
    \rmat A = \begin{bmatrix}
      1 & 3 & 4 & 5 \\
      3 & 6 & 9 & 9 \\
      2 & 3 & 5 & 4
    \end{bmatrix} % (2)
    \hspace{2cm}
    \rmat B = \begin{bmatrix}
      2  & -2  & 1 & 6  \\
      4  & -4  & 2 & -2 \\
      10 & -10 & 5 & 2
    \end{bmatrix} % (1)
  \]

  \exsol{
    Először határozzuk meg az $\rmat A$ mátrix rangját!
    \begin{align*}
      \rg \rmat A
       & = \rg
      \begin{bamatrix}{4}{5.5mm}
        1 & 3 & 4 & 5 \\
        3 & 6 & 9 & 9 \\
        2 & 3 & 5 & 4
      \end{bamatrix}
      \;
      \begin{matrix}
        \\\\(+S_1 - S_2)
      \end{matrix}
      \\
       & = \rg
      \begin{bamatrix}{4}{5.5mm}
        1 & 3 & 4 & 5 \\
        3 & 6 & 9 & 9 \\
        0 & 0 & 0 & 0
      \end{bamatrix}
      \;
      \begin{matrix}
        \\(-3S_1)\\\,
      \end{matrix}
      \\
       & = \rg
      \begin{bamatrix}{4}{5.5mm}
        \gcc{1} & \gcc{3}  & \gcc{4}  & \gcc{5}  \\
        0       & \gcc{-3} & \gcc{-3} & \gcc{-6} \\
        0       & 0        & 0        & 0
      \end{bamatrix}
      \\
       & =2
    \end{align*}

    \tcbline

    Most pedig határozzuk meg $\rmat B$ mátrix rangját!
    \begin{align*}
      \rg \rmat B
       & = \rg
      \begin{bamatrix}{4}{7.5mm}
        2  & -2  & 1 & 6  \\
        4  & -4  & 2 & -2 \\
        10 & -10 & 5 & 2
      \end{bamatrix}
      \;
      \begin{matrix}
        \\\\(-S_1 - 2S_2)
      \end{matrix}
      \\
       & = \rg
      \begin{bamatrix}{4}{7.5mm}
        2  & -2  & 1 & 6  \\
        4  & -4  & 2 & -2 \\
        0 & 0 & 0 & 0
      \end{bamatrix}
      \;
      \begin{matrix}
        \\(-2S_1)\\\,
      \end{matrix}
      \\
       & = \rg
      \begin{bamatrix}{4}{7.5mm}
        \gcc{2} & \gcc{-2} & \gcc{1} & \gcc{6}   \\
        0       & 0        & 0       & \gcc{-14} \\
        0       & 0        & 0       & 0
      \end{bamatrix}
      \\
       & =2
    \end{align*}
  }
\end{exercise}

  \vfill
\end{frame}

\begin{frame}
  \frametitle{Mátrixok}
  \framesubtitle{Mátrix inverz}

  \vfill

  \begin{block}{Reguláris / Szinguláris mátrix}
    Egy kvadratikus ($\rmat A \in \mathbb R^{n \times n}$) mátrixot
    \textbf{reguláris}nak mondunk, ha determinánsa nem 0.
    \\[3mm]
    Ha a kvadratikus mátrix determinánsa 0, \textbf{szinguláris} mátrixról
    beszélünk.
  \end{block}

  \vfill

  \begin{block}{Inverz}
    Az $\rmat A \in \mathbb R^{n \times n}$ reguláris mátrix inverze alatt azt
    az $\rmat A^{-1} \in \mathbb R^{n \times n}$ mátrixot értjük, melyre
    $\rmat A \cdot \rmat A^{-1} = \imat$ egyenlőség teljesül.
    \\[3mm]
    Szinguláris mátrixnak nem létezik az inverze.
  \end{block}

  \vfill
\end{frame}

\begin{frame}
  \frametitle{Mátrixok}
  \framesubtitle{Inverz meghatározása}

  \begin{block}{Adjugált mátrix segítségével}
    \[
      \rmat A^{-1} = \frac{\adj \rmat A}{\det \rmat A}
      \hspace{1cm}
      \rmat A = \begin{bmatrix}
        a_{11} & a_{12} & a_{13} \\
        a_{21} & a_{22} & a_{23} \\
        a_{31} & a_{32} & a_{33}
      \end{bmatrix}
    \]
    \[
      \adj \rmat A = \begin{bmatrix}
        + \begin{vmatrix}
            a_{22} & a_{23} \\
            a_{32} & a_{33}
          \end{vmatrix}
         &
        - \begin{vmatrix}
            a_{12} & a_{13} \\
            a_{32} & a_{33}
          \end{vmatrix}
         &
        +\begin{vmatrix}
           a_{12} & a_{13} \\
           a_{22} & a_{23}
         \end{vmatrix}
        \\
        - \begin{vmatrix}
            a_{21} & a_{23} \\
            a_{31} & a_{33}
          \end{vmatrix}
         &
        + \begin{vmatrix}
            a_{11} & a_{13} \\
            a_{31} & a_{33}
          \end{vmatrix}
         &
        - \begin{vmatrix}
            a_{11} & a_{13} \\
            a_{21} & a_{23}
          \end{vmatrix}
        \\
        + \begin{vmatrix}
            a_{21} & a_{22} \\
            a_{31} & a_{32}
          \end{vmatrix}
         &
        - \begin{vmatrix}
            a_{11} & a_{12} \\
            a_{31} & a_{32}
          \end{vmatrix}
         &
        +\begin{vmatrix}
           a_{11} & a_{12} \\
           a_{21} & a_{22}
         \end{vmatrix}
      \end{bmatrix}
    \]
  \end{block}
\end{frame}

\begin{frame}
  \frametitle{Mátrixok}
  \framesubtitle{Inverz meghatározása}

  \vfill

  \begin{block}{Gauss-Jordan eliminációval}
    \[
      \left[\begin{array}{*{3}{>{\cdot}{c}}|*{3}c}
           &  &  & 1 & 0 & 0 \\
           &  &  & 0 & 1 & 0 \\
           &  &  & 0 & 0 & 1 \\
        \end{array}\right]
      \quad \sim \quad
      \left[\begin{array}{*{3}{c}|*{3}{>{\cdot}{c}}}
          1 & 0 & 0 &  &  & \\
          0 & 1 & 0 &  &  & \\
          0 & 0 & 1 &  &  & \\
        \end{array}\right]
    \]
  \end{block}

  \vfill

  \begin{exercise}{Határozzuk meg az alábbi mátrixok inverzét!}
  \[
    \rmat A = \begin{bmatrix}
      2 & 3 \\
      3 & 5
    \end{bmatrix}
    \hspace{2cm}
    \rmat B = \begin{bmatrix}
      1 & 4 & 8 \\
      0 & 1 & 0 \\
      2 & 4 & 6
    \end{bmatrix}
  \]

  \exsol{
    Határozzuk meg $\rmat A$ inverzét mindkét tanult módszer segítségével:
    \begin{itemize}
      \item Definíció szerint:
            \begin{itemize}
              \item A mátrix determinánsa:
                    \[
                      \det \rmat A = \begin{vmatrix}
                        2 & 3 \\
                        3 & 5 \\
                      \end{vmatrix} = 2 \cdot 5 - 3 \cdot 3 = 1
                      \text.
                    \]
              \item A mátrix transzponáltja:
                    \[
                      \rmat A^{\mathsf T} = \begin{bmatrix}
                        2 & 3 \\
                        3 & 5 \\
                      \end{bmatrix}
                      \text.
                    \]
              \item A mátrix adjugáltja:
                    \[
                      \adj \rmat A = \begin{bmatrix}
                        +5 & -3 \\
                        -3 & +2
                      \end{bmatrix}
                      \text.
                    \]
              \item Az inverz ezek alapján:
                    \[
                      \rmat A^{-1}
                      = \frac{\adj \rmat A}{\det \rmat A}
                      = \frac{1}{1} \begin{bmatrix}
                        5  & -3 \\
                        -3 & 2
                      \end{bmatrix}  = \begin{bmatrix}
                        5  & -3 \\
                        -3 & 2
                      \end{bmatrix}
                      \text.
                    \]
            \end{itemize}

      \item Gauss-Jordan eliminációval:
            \begin{align*}
               & \left[\begin{array}{X{6mm}X{8mm}|X{9mm}X{6mm}}
                           2 & 3 & 1 & 0 \\
                           3 & 5 & 0 & 1
                         \end{array}\right]\begin{matrix}(/2)\\\,\end{matrix}
              \\
               & \left[\begin{array}{X{6mm}X{8mm}|X{9mm}X{6mm}}
                           1 & 3/2 & 1/2 & 0 \\
                           3 & 5   & 0   & 1
                         \end{array}\right]\begin{matrix}\\(-3S_1)\end{matrix}
              \\
               & \left[\begin{array}{X{6mm}X{8mm}|X{9mm}X{6mm}}
                           1 & 3/2 & 1/2  & 0 \\
                           0 & 1/2 & -3/2 & 1
                         \end{array}\right]\begin{matrix}(-3S_2)\\\,\end{matrix}
              \\
               & \left[\begin{array}{X{6mm}X{8mm}|X{9mm}X{6mm}}
                           1 & 0   & 5    & -3 \\
                           0 & 1/2 & -3/2 & 1
                         \end{array}\right]\begin{matrix}(\cdot2)\\\,\end{matrix}
              \\
               & \left[\begin{array}{X{6mm}X{8mm}|X{9mm}X{6mm}}
                           1 & 0 & 5  & -3 \\
                           0 & 1 & -3 & 2
                         \end{array}\right]
            \end{align*}
    \end{itemize}

    \tcbline

    Határozzuk meg $\rmat B$ inverzét mindkét tanult módszer segítségével:
    \begin{itemize}
      \item Definíció alapján:
            \begin{itemize}
              \item A mátrix determinánsát már korábban meghatároztuk:
                    \[
                      \det \rmat B = -10
                      \text.
                    \]

              \item A mátrix transzponáltja:
                    \[
                      \rmat B^{\mathsf T} = \begin{bmatrix}
                        1 & 0 & 2 \\
                        4 & 1 & 4 \\
                        8 & 0 & 6
                      \end{bmatrix}
                      \text.
                    \]
              \item A mátrix adjugáltja:
                    \newcommand{\qvmat}[4]{\begin{vmatrix}#1 & #2 \\ #3 & #4\end{vmatrix}}
                    \[
                      \adj \rmat B = \begin{bmatrix}
                        +\qvmat{1}{4}{0}{6} & -\qvmat{4}{4}{8}{6} & +\qvmat{4}{1}{8}{0} \\
                        -\qvmat{0}{2}{0}{6} & +\qvmat{1}{2}{8}{6} & -\qvmat{1}{0}{8}{0} \\
                        +\qvmat{0}{2}{1}{4} & -\qvmat{1}{2}{4}{4} & +\qvmat{1}{0}{4}{1} \\
                      \end{bmatrix} = \begin{bmatrix}
                        6  & 8   & -8 \\
                        0  & -10 & 0  \\
                        -2 & 4   & 1  \\
                      \end{bmatrix}
                      \text.
                    \]
              \item Az inverz ezek alapján:
                    \begin{align*}
                      \rmat B^{-1}
                      = \frac{\adj \rmat B}{\det \rmat B}
                      = & \frac{1}{-10}
                      \left[\begin{array}{X{1.2cm}X{1.2cm}X{1.2cm}}
                                6  & 8   & -8 \\
                                0  & -10 & 0  \\
                                -2 & 4   & 1  \\
                              \end{array}\right]
                      \\
                      = & \phantom{\frac{1}{-10}} \left[\begin{array}{X{1.2cm}X{1.2cm}X{1.2cm}}
                                                            -6/10 & -8/10 & 8/10  \\
                                                            0     & 1     & 0     \\
                                                            2/10  & -4/10 & -1/10
                                                          \end{array}\right]
                      \\
                      = & \phantom{\frac{1}{-10}} \left[\begin{array}{X{1.2cm}X{1.2cm}X{1.2cm}}
                                                            -3/5 & -4/5 & 4/5   \\
                                                            0    & 1    & 0     \\
                                                            1/5  & -2/5 & -1/10
                                                          \end{array}\right]
                      \text.
                    \end{align*}
            \end{itemize}

      \item Gauss-Jordan eliminációval:
            \newcommand{\qgj}[6]{\left[\begin{array}{*{3}{X{12mm}}|*{3}{X{12mm}}}
                  #1 \\#3\\#5
                \end{array}\right]\begin{matrix}#2\\#4\\#6\end{matrix}}
            \begin{align*}
                 & \qgj
              {1 & 4    & 8   & 1     & 0     & 0}{}
              {0 & 1    & 0   & 0     & 1     & 0}{}
              {2 & 4    & 6   & 0     & 0     & 1}{(-2S_1)}
              \\
                 & \qgj
              {1 & 4    & 8   & 1     & 0     & 0}{(-4S_2)}
              {0 & 1    & 0   & 0     & 1     & 0}{}
              {0 & -4   & -10 & -2    & 0     & 1}{(+4S_2)}
              \\
                 & \qgj
              {1 & 0    & 8   & 1     & -4    & 0}{}
              {0 & 1    & 0   & 0     & 1     & 0}{}
              {0 & 0    & -10 & -2    & 4     & 1}{(/(-10))}
              \\
                 & \qgj
              {1 & 0    & 8   & 1     & -4    & 0}{(-8S_3)}
              {0 & 1    & 0   & 0     & 1     & 0}{}
              {0 & 0    & 1   & 2/10  & -4/10 & -1/10}{}
              \\
                 & \qgj
              {1 & 0    & 0   & -6/10 & -8/10 & 8/10}{}
              {0 & 1    & 0   & 0     & 1     & 0}{}
              {0 & 0    & 1   & 2/10  & -4/10 & -1/10}{}
              \\
                 & \qgj
              {1 & 0    & 0   & -3/5  & -4/5  & 4/5}{}
              {0 & 1    & 0   & 0     & 1     & 0}{}
              {0 & 0    & 1   & 1/5   & -2/5  & -1/10}{}
            \end{align*}
    \end{itemize}
  }
\end{exercise}


  \vfill
\end{frame}

\begin{frame}
  \frametitle{Mátrixok}
  \framesubtitle{Mátrix egyenletek}

  \begin{exercise}{Oldjuk meg az alábbi mátrix-egyenleteket!}
  \[
    \rmat A = \begin{bmatrix}
      2 & 3 \\
      3 & 5
    \end{bmatrix}
    \hspace{2cm}
    \rmat B = \begin{bmatrix}
      1 & 2 \\
      3 & 4 \\
      6 & 3
    \end{bmatrix}
  \]
  \begin{enumerate}[a)]
    \item $\rmat X \cdot \rmat A = \rmat B$
    \item $\rmat A \cdot \rmat X = \rmat B$
    \item $2(\rmat A + \rmat X) = 3(\rmat X - \rmat A^{-1})$
    \item $\rmat B \cdot \rmat B^{\mathsf T} = \rmat A \cdot \rmat X$
  \end{enumerate}

  \exsol[18.25cm]{
    \begin{enumerate}[a)]
      \item $\rmat X \cdot \rmat A = \rmat B$

            Rendezzük $\rmat X$-re az egyenletet, vagyis szorozzuk meg az
            egyenlet mindkét oldalát $\rmat A$ inverzével. Ekkor az alábbi
            egyenletet kapjuk:
            \[
              \rmat X = \rmat B \cdot \rmat A^{-1}
              \text{.}
            \]
            Az $\rmat A$ mátrix inverzét már korábban meghatároztuk.
            Az egyenletbe behelyettesítve:
            \[
              \rmat X
              = \rmat B \cdot\rmat A^{-1} = \begin{bmatrix}
                1 & 2 \\
                3 & 4 \\
                6 & 3
              \end{bmatrix} \begin{bmatrix}
                5  & -3 \\
                -3 & 2
              \end{bmatrix} = \begin{bmatrix}
                5 - 6   & -3 + 4  \\
                15 - 12 & -9 + 8  \\
                30-9    & -18 + 6
              \end{bmatrix} = \begin{bmatrix}
                -1 & 1   \\
                3  & -1  \\
                21 & -12
              \end{bmatrix}
              \text.
            \]

            \tcbline
      \item $\rmat A \cdot \rmat X = \rmat B$

            Ez az egyenlet nem megoldható, hiszen a $\rmat A^{-1} \rmat B$
            szorzás nem értelmezett, hiszen $\rmat A^{-1} \in
              \mathbb R^{2 \times 2}$, $\rmat B \in \mathbb R^{3 \times 2}$,
            vagyis az $\rmat A^{-1}$ mátrix oszlopainak száma nem egyezik meg
            a $\rmat B$ mátrix sorainak számával.

            \tcbline
      \item $2(\rmat A + \rmat X) = 3(\rmat X - \rmat A^{-1})$

            Rendezzük $\rmat X$-re az egyenletet:
            \begin{gather*}
              2 \rmat A + 2 \rmat X = 3 \rmat X - 3 \rmat A^{-1}
              \quad \rightarrow \quad
              \rmat X = 2 \rmat A + 3 \rmat A^{-1}
              \\
              \rmat X = 2 \begin{bmatrix}
                2 & 3 \\ 3 & 5
              \end{bmatrix} + 3 \begin{bmatrix}
                5 & -3 \\ -3 & 2
              \end{bmatrix} = \begin{bmatrix}
                19 & -3 \\ -3 & 16
              \end{bmatrix}
            \end{gather*}

            \tcbline
      \item $\rmat B \cdot \rmat B^{\mathsf T} = \rmat A \cdot \rmat X$

            Az egyenlet nem megoldható, hiszen $(\rmat B \cdot \rmat B^{\mathsf T})
              \in \mathbb R^{3 \times 3}$, $\rmat A \in \mathbb R^{2 \times 2}$.
    \end{enumerate}
  }
\end{exercise}

\end{frame}

% System of linear equations
\section{Lineáris egyenletrendszerek}
\begin{frame}
  \frametitle{Lineáris egyenletrendszerek}
  \framesubtitle{Definíció}

  \begin{block}{Lineáris egyenletrendszer}
    Véges sok elsőfokú egyenletet és véges sok ismeretlent tartalmazó
    egyenletrendszert lineáris egyenletrendszernek nevezünk.
    \\[2mm]
    Az $m$ egyenletből és $n$ ismeretlenből álló lineáris egyenletrendszer
    általános alakja:
    \[
      \begin{array}{*{9}{c}}
        a_{11} x_{1} & + & a_{12} x_{2} & + & \dots  & + & a_{1n} x_{n} & = & b_{1}\text, \\[1mm]
        a_{21} x_{1} & + & a_{22} x_{2} & + & \dots  & + & a_{2n} x_{n} & = & b_{2}\text, \\[1mm]
        \vdots       &   & \vdots       &   & \vdots &   & \vdots       &   & \vdots      \\[1mm]
        a_{m1} x_{1} & + & a_{m2} x_{2} & + & \dots  & + & a_{mn} x_{n} & = & b_{m}\text.
      \end{array}
    \]
  \end{block}
\end{frame}

\begin{frame}
  \frametitle{Lineáris egyenletrendszerek}
  \framesubtitle{Mátrixos alak}

  \begin{block}{LER mátrixos alakja}
    Egy lineáris egyenletrendszer felírható $\rmat A \rvec x = \rmat b$
    alakban, ahol $\rmat A$ az együttható mátrix, $\rvec x$ az ismeretlenek
    vektora, $\rvec b$ pedig a konstans vektor.

    \[
      \underbrace{\begin{bmatrix}
          a_{11} & a_{12} & \cdots & a_{1n} \\
          a_{21} & a_{22} & \cdots & a_{2n} \\
          \vdots & \vdots & \ddots & \vdots \\
          a_{m1} & a_{m2} & \cdots & a_{mn}
        \end{bmatrix}}_{\rmat A} \underbrace{\begin{bmatrix}
          x_{1} \\ x_{2} \\ \vdots \\ x_{n}
        \end{bmatrix}}_{\rvec x} = \underbrace{\begin{bmatrix}
          b_{1} \\ b_{2} \\ \vdots \\ b_{m}
        \end{bmatrix}}_{\rvec b}
    \]
  \end{block}
\end{frame}

\begin{frame}
  \frametitle{Lineáris egyenletrendszerek}
  \framesubtitle{Megoldhatóság}

  \begin{block}{LER megoldhatóságának szükséges és elégséges feltétele}
    Az $\rmat A \rvec x = \rvec b$ lineáris egyenletrendszer akkor és csak
    akkor oldható meg, ha $\rg(\rmat A) = \rg(\rmat A | \rvec b)$, ahol az
    $(\rmat A | \rvec b)$ mátrixot kibővített mátrixnak nevezzük.

    A feltétel mátrixosan:
    \[
      \rg \begin{bmatrix}
        a_{11} & a_{12} & \cdots & a_{1n} \\
        a_{21} & a_{22} & \cdots & a_{2n} \\
        \vdots & \vdots & \ddots & \vdots \\
        a_{m1} & a_{m2} & \cdots & a_{mn}
      \end{bmatrix} = \rg \left[\begin{array}{cccc|c}
          a_{11} & a_{12} & \cdots & a_{1n} & b_1    \\
          a_{21} & a_{22} & \cdots & a_{2n} & b_2    \\
          \vdots & \vdots & \ddots & \vdots & \vdots \\
          a_{m1} & a_{m2} & \cdots & a_{mn} & b_n
        \end{array}\right]\text.
    \]

    A feltételből következik, hogy homogén lineáris egyenletrendszer
    ($\rvec b = \nvec$) mindig megoldható, hiszen az együttható mátrixból és egy
    nullvektorból képzett kibővített mátrix rangja mindig meg fog egyezni az
    együttható mátrix rangjával.
  \end{block}
\end{frame}

\begin{frame}
  \frametitle{Lineáris egyenletrendszerek}
  \framesubtitle{Megoldási módszerek I}

  \begin{block}{Mátrix inverziós módszer}
    Ha az $\rmat A$ mátrix kvadratikus és reguláris, akkor invertálható:
    \[
      \rvec x = \rmat A^{-1} \rvec b
      \text.
    \]
  \end{block}

  \begin{block}{Cramer-szabály}
    Ha az $\rmat A$ mátrix kvadratikus és reguláris, akkor az együtthatók az
    alábbi módon számíthatóak:
    \[
      x_i = \frac{\det \rmat A_i}{\det \rmat A}
      \text,
    \]
    ahol az $\rmat A_i$ mátrixot úgy képezzük, hogy az $i$-edig sorába
    $\rvec b$ vektort írjuk be.
  \end{block}
\end{frame}

\begin{frame}
  \frametitle{Lineáris egyenletrendszerek}
  \framesubtitle{Megoldási módszerek II}

  \begin{block}{Gauss elimináció}
    Sorműveletekkel alakítjuk a kibővített mátrixot:
    \[
      \left[\begin{array}{cccc|c}
          a_{11} & a_{12} & \cdots & a_{1n} & b_1    \\
          a_{21} & a_{22} & \cdots & a_{2n} & b_2    \\
          \vdots & \vdots & \ddots & \vdots & \vdots \\
          a_{m1} & a_{m2} & \cdots & a_{mn} & b_n
        \end{array}\right]
      \quad\sim\quad
      \left[\begin{array}{cccc|c}
          \square & \square & \cdots & \square & \square \\
          0       & \square & \cdots & \square & \square \\
          \vdots  & \vdots  & \ddots & \vdots  & \vdots  \\
          0       & 0       & \cdots & \circ   & \circ
        \end{array}\right]
    \]
  \end{block}

  \begin{block}{Megoldások száma $\rmat A \in \mathbb R^{n \times n}$ esetben}
    \centering
    \begin{tikzpicture}[thick]
      \foreach \i in {0,1,2} {
          \node at (\i*4cm,0) {$\left[\phantom{\begin{matrix}10000000000\\\\\\\\\end{matrix}}\right]$};

          \draw (\i*4cm+4mm,.75cm) -- ++(0,-1.5cm);
        }

      \draw (5.35mm,7mm) rectangle ++(5mm, -14mm);
      \draw (45.35mm,7mm) rectangle ++(5mm, -14mm);
      \draw (85.35mm,7mm) rectangle ++(5mm, -9.25mm) node[below left] {$0$};

      \draw (2.65mm,7mm)
      -- ++(-13.75mm,0)
      -- ++(6.75mm,-9.25mm)
      -- ++(7mm,0)
      node[midway, below] {$0$}
      -- cycle;
      \draw (42.65mm,7mm)
      -- ++(-13.75mm,0)
      -- ++(13.75mm,-13.75mm)
      node[midway, below left=2.25mm] {$\quad0$}
      -- cycle;
      \draw (82.65mm,7mm)
      -- ++(-13.75mm,0)
      -- ++(6.75mm,-9.25mm)
      -- ++(7mm,0)
      node[midway, below] {$0$}
      -- cycle;

      \node[] at (0cm,1cm) {Nincs mo.};
      \node[] at (4cm,1cm) {1db mo.};
      \node[] at (8cm,1cm) {$\infty$ mo.};
    \end{tikzpicture}
  \end{block}
\end{frame}

\begin{frame}
  \frametitle{Lineáris egyenletrendszerek}
  \framesubtitle{Feladatok}

  \vfill
  \begin{exercise}{Oldjuk meg az alábbi lineáris egyenletrendszert!}
  \[
    \begin{array}{*{7}{c}}
      x_{1}   & + & 4 x_{2} & + & 8 x_{3} & = & 23 \\
              &   & x_{2}   &   &         & = & 1  \\
      2 x_{1} & + & 4 x_{2} & + & 6 x_{3} & = & 22
    \end{array}
  \]

  \exsol[14.25cm]{%
    Oldjuk meg a feladatot mátrix inverziós módszerrel! Írjuk fel az együttható
    mátrixot, az ismeretlenek vektorát és a konstans vektort:
    \[
      \rmat A = \begin{bmatrix}
        1 & 4 & 8 \\
        0 & 1 & 0 \\
        2 & 4 & 6
      \end{bmatrix}
      \text,
      \quad
      \rvec x = \begin{bmatrix}
        x_1 \\ x_2 \\ x_3
      \end{bmatrix}
      \text,
      \quad
      \rvec b = \begin{bmatrix}
        23 \\ 1 \\ 22
      \end{bmatrix}
      \text.
    \]

    Az $\rmat A$ mátrix inverzét már korábban meghatároztuk:
    \[
      \rmat A^{-1} = \begin{bmatrix}
        -3/5 & -4/5 & 4/5   \\
        0    & 1    & 0     \\
        1/5  & -2/5 & -1/10
      \end{bmatrix}
      \text.
    \]

    Az egyenletrendszer megoldása tehát:
    \[
      \rvec x
      = \rmat A^{-1} \rvec b
      = \begin{bmatrix}
        -3/5 & -4/5 & 4/5   \\
        0    & 1    & 0     \\
        1/5  & -2/5 & -1/10
      \end{bmatrix} \begin{bmatrix}
        23 \\ 1 \\ 22
      \end{bmatrix}
      = \begin{bmatrix}
        -3/5 \cdot 23 - 4/5 \cdot 1 + 4/5 \cdot 22 \\
        0 \cdot 23 + 1 \cdot 1 + 0 \cdot 22        \\
        1/5 \cdot 23 - 2/5 \cdot 1 - 1/10 \cdot 22
      \end{bmatrix}
      = \begin{bmatrix}
        3 \\ 1 \\ 2
      \end{bmatrix}
      \text.
    \]

    Tehát az egyes változók értékei: $x_1 = 3$, $x_2 = 1$, és $x_3 = 2$.

    \tcbline

    Oldjuk meg Gauss-eliminációval is az egyenletrendszert:
    \newcommand{\qgj}[6]{\left[\begin{array}{*{3}{X{8mm}}|X{8mm}}
          #1 \\#3\\#5
        \end{array}\right]\begin{matrix}#2\\#4\\#6\end{matrix}}
    \begin{align*}
         & \qgj
      {1 & 4    & 8   & 23}{}
      {0 & 1    & 0   & 1 }{}
      {2 & 4    & 6   & 22}{(-2S_1)}
      \\
         & \qgj
      {1 & 4    & 8   & 23 }{(-4S_2)}
      {0 & 1    & 0   & 1  }{}
      {0 & -4   & -10 & -24}{(+4S_2)}
      \\
         & \qgj
      {1 & 0    & 8   & 19 }{}
      {0 & 1    & 0   & 1  }{}
      {0 & 0    & -10 & -20}{(/(-10))}
      \\
         & \qgj
      {1 & 0    & 8   & 19}{(-8S_3)}
      {0 & 1    & 0   & 1 }{}
      {0 & 0    & 1   & 2 }{\;}
      \\
         & \qgj
      {1 & 0    & 0   & 3}{}
      {0 & 1    & 0   & 1}{}
      {0 & 0    & 1   & 2}{}
    \end{align*}
    % Látható, hogy ezzel a módszerrel is ugyan azokat a megoldásokat kaptuk.
  }
\end{exercise}

  \vfill
  \begin{exercise}{%
		Milyen együtthatók esetén nem lesz megoldható az egyenletrendszer
		Cramer-szabállyal?
	}
	\[
		\begin{array}{*{7}{c}}
			a x_{1}  & + & -2 x_{2} & + & 1 x_{3} & = & 1 \\
			-4 x_{1} & + & b x_{2}  & + & 2 x_{3} & = & 2 \\
			-8 x_{1} & + & 3b x_{2} & + & 4 x_{3} & = & 3
		\end{array}
	\]

	\exsol{%
		A Cramer-szabály akkor alkalmazható, ha az együttható mátrix reguláris.
		Írjuk fel a mátrixot, majd vizsgáljuk meg, hogy milyen együtthatók esetén
		lesz szinguláris, hiszen ilyen esetekben nem használható a Cramer-szabály:
		\[
			\rmat A = \begin{bmatrix}
				a  & -2 & 1 \\
				-4 & b  & 2 \\
				-8 & 3b & 4
			\end{bmatrix}
			\text.
		\]
		Az mátrix az alábbi esetekben lesz szinguláris:
		\begin{itemize}
			\item ha az első és a harmadik oszlop egymás skalárszorosa,
			      vagyis ha $a = -2$,
			\item ha a második és a harmadik sor egymás skalárszorosa,
			      vagyis ha $b = 0$.
		\end{itemize}
		Tehát a Cramer-szabályt akkor nem tudjuk alkalmazni,
		ha $a = 2$, ha vagy $b = 0$.
	}
\end{exercise}

  \vfill
\end{frame}

% Linear mappings
\section{Lineáris leképezések}
\begin{frame}
  \frametitle{Lineáris leképezések}
  \framesubtitle{Alapfogalmak I}

  \begin{block}{Lineáris leképezés}
    Legyenek $V_1$ és $V_2$ ugyanazon $T$ test feletti vektorterek. Legyen
    $\varphi: V_1 \rightarrow V_2$ leképezés, melyet lineáris leképezésnek
    nevezünk, ha tetszőleges két $V_1$-beli vektor ($\forall \rvec a; \rvec b \in
      V_1$) és $T$-beli skalár ($\alpha \in T$) esetén teljesülnek az alábbiak:
    \begin{itemize}
      \item $\varphi(\rvec a + \rvec b) = \varphi(\rvec a) + \varphi(\rvec b)$
            $\quad \sim \quad$ összegre tagonként hat,
      \item $\varphi(\alpha \rvec a) = \alpha \varphi(\rvec a)$
            $\hspace{18mm} \sim \quad$ skalár kiemelhető.
    \end{itemize}
  \end{block}
\end{frame}

\begin{frame}
  \frametitle{Lineáris leképezések}
  \framesubtitle{Alapfogalmak II}

  \vfill
  \begin{block}{Magtér}
    Legyen $\varphi: V_1 \rightarrow V_2$ lineáris leképezés. A leképezés
    magtere:
    \[
      \ker \varphi = \Big\{\;
      \rvec v \;\Big|\; \rvec v \in V_1 \;\land\;
      \varphi(\rvec v) = \nvec
      \;\Big\}
      \text.
    \]
  \end{block}
  \vfill
  \begin{block}{Defektus}
    A magtér dimenzióját a leképezés defektusának nevezzük:
    \[
      \dim \ker \varphi = \defect \varphi
      \text.
    \]
  \end{block}
  \vfill
\end{frame}

\begin{frame}
  \frametitle{Lineáris leképezések}
  \framesubtitle{Alapfogalmak III}

  \vfill
  \begin{block}{Képtér}
    Egy $\varphi: V_1 \rightarrow V_2$ lineáris leképezés rangjának nevezzük
    a képtér dimenzióját:
    \[
      \rg \varphi = \dim V_2
      \text.
    \]
  \end{block}
  \vfill
  \begin{block}{Rang nullitás tétele}
    Legyen $V_1$ véges dimenziós vektortér, $\varphi: V_1 \rightarrow V_2$
    lineáris leképezés, ekkor:
    \[
      \rg \varphi + \defect \varphi = \dim V_1
    \]
  \end{block}
  \vfill
\end{frame}

\begin{frame}
  \frametitle{Lineáris leképezések}
  \framesubtitle{Feladatok I}

  \begin{exercise}{Lineárisak-e az alábbi leképezések?}
  \begin{enumerate}[a)]
    \item \(
          \varphi: \mathbb R^2 \rightarrow \mathbb R^2;
          (x; y) \mapsto (e^x; y)
          \),
    \item \(
          \varphi: \mathbb R^2 \rightarrow \mathbb R^2;
          (x; y) \mapsto (42; 0)
          \),
    \item \(
          \varphi: \mathbb R^2 \rightarrow \mathbb R^2;
          (x; y) \mapsto (\arctan \tan x; -12)
          \),
    \item \(
          \varphi: \mathbb R^2 \rightarrow \mathbb R^2;
          (x; y) \mapsto (y + x; 2x)
          \),
  \end{enumerate}

  \exsol[4cm]{%
    \begin{enumerate}[a)]
      \item Nem lineáris, hiszen az $e^x$ függvény sem az.
      \item Nem lineáris, hiszen $\alpha \cdot 42 = 42$, nem igaz tetszőleges
            $\alpha$ számra,
      \item Nem lineáris, hiszen $\alpha \arctan \tan x \neq \arctan \tan \alpha x$.
      \item Lineáris, hiszen teljesülnek az alábbi feltételek:
    \end{enumerate}
    \[
      \varphi \begin{bmatrix}
        x_1 + x_2 \\ y_1 + y_2
      \end{bmatrix} = \begin{bmatrix}
        (y_1 + y_2) + (x_1 + x_2) \\ 2 (x_1 + x_2)
      \end{bmatrix} = \begin{bmatrix}
        y_1 + x_1 \\ 2x_1
      \end{bmatrix} + \begin{bmatrix}
        y_2 + x_2 \\ 2x_2
      \end{bmatrix} = \varphi \begin{bmatrix}
        x_1 \\ y_1
      \end{bmatrix} + \varphi \begin{bmatrix}
        x_1 \\ y_1
      \end{bmatrix}
      \text,
    \]\[
      \varphi \begin{bmatrix}
        \alpha x \\ \alpha y
      \end{bmatrix} = \begin{bmatrix}
        \alpha y + \alpha x \\ 2\alpha x
      \end{bmatrix} = \begin{bmatrix}
        \alpha(y + x) \\ \alpha(2x)
      \end{bmatrix} = \alpha \begin{bmatrix}
        y + x \\ 2x
      \end{bmatrix} = \alpha \varphi \begin{bmatrix}
        x \\ y
      \end{bmatrix}
      \text.
    \]
  }
\end{exercise}

\end{frame}

\begin{frame}
  \frametitle{Lineáris leképezések}
  \framesubtitle{Feladatok II}

  \begin{exercise}{%
    Adjuk meg az alábbi lineáris leképezés mátrixát a standard bázisban!
  }
  \[
    \varphi: \mathbb R^3 \rightarrow \mathbb R^3;
    \begin{bmatrix}
      x \\ y \\ z
    \end{bmatrix} \mapsto \begin{bmatrix}
      y + z \\
      x + z \\
      x + y
    \end{bmatrix}
  \]
  Adjuk meg továbbá:
  \begin{itemize}
    \item a képtér és magtér dimenzióját,
    \item a leképezés rangját és defektusát,
    \item a magtér egy tetszőleges elemét,
    \item az $(1; 2; 3)$ pont képét,
    \item a $(2; 2; 2)$ kép ősképét.
  \end{itemize}

  \exsol{%
    A leképezés mátrixa:
    \[
      \rmat A = \begin{bmatrix}
        0 & 1 & 1 \\
        1 & 0 & 1 \\
        1 & 1 & 0 \\
      \end{bmatrix}
      \text.
    \]

    Határozzuk meg a mátrix rangját a determinánsának számítása segítségével:
    \[
      \det \rmat A = 1 + 1 = 2
      \text.
    \]

    Mivel a mátrix reguláris ($\det \rmat A \neq 0$), ezért rangja maximális
    ($\rg \rmat A = 3$). A leképezés mátrix-reprezentációjának rangja megegyezik
    a leképezés rangjával és a képtér dimenziójával:
    \[
      \rg \rmat A = \rg \varphi = \dim V_2 = 3
      \text.
    \]
    A rang nullitás tételéből pedig következik, hogy:
    \[
      \defect \varphi = \dim \ker \varphi = \dim V_1 - \rg \varphi = 3 - 3 = 0
      \text.
    \]
    Mivel a leképezés defektusa 0, ezért a magtérnek egyetlen egy eleme van,
    mégpedig a nullvektor:
    \[
      \ker \varphi = \Big\{\; \nvec \;\Big\}
      \text.
    \]
    Az $(1;2;3)$ pont képe:
    \[
      \varphi \begin{bmatrix}
        1 \\ 2 \\ 3
      \end{bmatrix} = \begin{bmatrix}
        2 + 3 \\
        1 + 3 \\
        1 + 2
      \end{bmatrix} = \begin{bmatrix}
        5 \\ 4 \\3
      \end{bmatrix}
      \text.
    \]
    A $(2;2;2)$ kép ősképét meghatározhatjuk az $\rmat A$ mátrix invertálásával,
    vagy egy 3 ismeretből és 3 egyenletből álló lineáris egyenletrendszer
    megoldásával.
    \begin{enumerate}
      \newcommand\qadj[4]{\begin{vmatrix}#1&#2\\#3&#4\end{vmatrix}}
      \item Mátrix invertálásos módszer: \\[2mm]
            Határozzuk meg először az $\rmat A$ mátrix adjugáltját:
            \[
              \adj \rmat A = \begin{bmatrix}
                +\qadj{0}{1}{1}{0} & -\qadj{1}{1}{1}{0} & +\qadj{1}{0}{1}{1} \\
                -\qadj{1}{1}{1}{0} & +\qadj{0}{1}{1}{0} & -\qadj{0}{1}{1}{1} \\
                +\qadj{1}{1}{0}{1} & -\qadj{0}{1}{1}{1} & +\qadj{0}{1}{1}{0} \\
              \end{bmatrix} = \begin{bmatrix}
                -1 & 1  & 1  \\
                1  & -1 & 1  \\
                1  & 1  & -1 \\
              \end{bmatrix}
              \text.
            \]
            A mátrix inverze tehát:
            \[
              \rmat A^{-1}
              = \frac{\adj \rmat A}{\det \rmat A}
              = \frac{1}{2} \begin{bmatrix}
                -1 & 1  & 1  \\
                1  & -1 & 1  \\
                1  & 1  & -1 \\
              \end{bmatrix}
              \text.
            \]
            A keresett kép ősképe tehát:
            \[
              \varphi^{-1} \begin{bmatrix}
                2 \\ 2 \\ 2
              \end{bmatrix} = \frac{1}{2} \begin{bmatrix}
                -1 & 1  & 1  \\
                1  & -1 & 1  \\
                1  & 1  & -1 \\
              \end{bmatrix} \begin{bmatrix}
                2 \\ 2 \\ 2
              \end{bmatrix} = \frac{1}{2} \begin{bmatrix}
                -2 + 2 + 2 \\ 2 - 2 + 2 \\ 2 + 2 - 2
              \end{bmatrix} = \frac{1}{2} \begin{bmatrix}
                2 \\ 2 \\ 2
              \end{bmatrix} = \begin{bmatrix}
                1 \\ 1 \\ 1
              \end{bmatrix}
              \text.
            \]

      \item Lineáris egyenletrendszeres megoldás:
            \newcommand{\qgj}[6]{\left[\begin{array}{*{3}{X{6mm}}|X{6mm}}
                  #1 \\#3\\#5
                \end{array}\right]\begin{matrix}#2\\#4\\#6\end{matrix}}
            \begin{align*}
                 & \qgj
              {0 & 1    & 1  & 2}{}
              {1 & 0    & 1  & 2}{}
              {1 & 1    & 0  & 2}{}
              \\
                 & \qgj
              {1 & 0    & 1  & 2}{}
              {0 & 1    & 1  & 2}{}
              {1 & 1    & 0  & 2}{(-S_1)}
              \\
                 & \qgj
              {1 & 0    & 1  & 2}{}
              {0 & 1    & 1  & 2}{}
              {0 & 1    & -1 & 0}{(-S_2)}
              \\
                 & \qgj
              {1 & 0    & 1  & 2 }{}
              {0 & 1    & 1  & 2 }{}
              {0 & 0    & -2 & -2}{(/(-2))}
              \\
                 & \qgj
              {1 & 0    & 1  & 2}{(-S_3)}
              {0 & 1    & 1  & 2}{(-S_3)}
              {0 & 0    & 1  & 1}{}
              \\
                 & \qgj
              {1 & 0    & 0  & 1}{}
              {0 & 1    & 0  & 1}{}
              {0 & 0    & 1  & 1}{}
            \end{align*}
    \end{enumerate}
    Láthatjuk, hogy mindkét módszerrel ugyanazt az eredményt kaptuk.
  }
\end{exercise}

\end{frame}

% Eigenvalues and eigenvectors
\section{Sajátértékek és sajátvektorok}
\begin{frame}
  \frametitle{Sajátértékek és sajátvektorok}
\end{frame}



\end{document}
