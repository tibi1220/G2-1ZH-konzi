\documentclass[xcolor={table}]{beamer}

\usepackage[magyar]{babel}

\title{Matematika G2 Első Zárthelyi Konzultáció}
\author{Sándor Tibor}
\date{\today}

\usepackage{amsmath,amssymb}
\usepackage{unicode-math}
\usepackage[table]{xcolor}
\usepackage{array}
\usepackage{multicol}
\usepackage{icomma}
\usepackage{tabto}
\usepackage{hyperref}

\newcolumntype{x}[1]{>{\centering\arraybackslash\hspace{0pt}}p{#1}}
\newcolumntype{X}[1]{>{$}x{#1}<{$}}

\newenvironment{bamatrix}[2]{\left[\begin{array}{*{#1}{X{#2}}}}{\end{array}\right]}
\newcommand{\gcc}[1]{\cellcolor{gray!25}{#1}}

\newcommand\iu{\mathbf{i}}
\newcommand{\nvec}{\text{\textbf{\textit{0}}}}
\newcommand\imat{{\mathbb{E}}}
\newcommand\nmat{{\mathbb{O}}}
\newcommand{\rvec}[1]{\mathbfit{#1}}
\newcommand{\uvec}[1]{\widehat{\mathbfit{#1}}}
\newcommand{\rmat}[1]{\mathbf{#1}}
\newcommand{\edet}[1]{\det \begin{pmatrix} \phantom{i}\dots & #1 & \dots\phantom{i} \end{pmatrix}}
\DeclareMathOperator{\rg}{rg}
\DeclareMathOperator{\adj}{adj}
\DeclareMathOperator{\defect}{def}
\let\dim\relax
\DeclareMathOperator{\dim}{dim}

\usepackage[stixtwo]{fontsetup}

\usepackage{tikz, pgfplots}
\usetikzlibrary{calc, matrix, patterns, patterns.meta}
\pgfplotsset{compat=1.18}
\pgfkeys{/pgf/plot/gnuplot call={cd build && gnuplot}}

\usepackage{mystyles}

% Beamer general settings
\usetheme{Frankfurt}
\institute{Mechatronika szakosztály}
% \logo{\includegraphics[height=1cm]{mszo-logo.pdf}}

\mode<presentation>
\setbeamercovered{transparent}
\usefonttheme[onlymath]{serif}

\titlegraphic{\includegraphics[scale=.25]{static/mszo-768x267.png}}
% \logo{\includegraphics[]{static/mszo-768x267.png}}
% \setbeamertemplate{headline}{\hfill\includegraphics[width=1.5cm]{static/mszo-768x267.png}\hspace{0.2cm}\vspace{-1cm}}


\usepackage{ragged2e}
\addtobeamertemplate{block begin}{}{\justifying}
\addtobeamertemplate{frametitle}{}{%
\begin{tikzpicture}[remember picture,overlay]
\node[anchor=north east,yshift=-6mm] at (current page.north east) {
  \includegraphics[height=10mm]{static/mszo-768x267.png}
};
\end{tikzpicture}\vspace{-5mm}}

% Document begins here
\begin{document}

% Titlepage
\frame{\titlepage}

% About -- the structure of the consultation
\begin{frame}
  \frametitle{A konzultáció felépítése}
  \framesubtitle{\;}
  \tableofcontents
\end{frame}

% Main elements
\section{Vektorterek}
\begin{frame}
  \frametitle{Vektorterek}
  \framesubtitle{Definíció}

  Legyen $V$ nem üreshalmaz, és $+; \lambda$ két művelet, valamint $T$ test.
  $(V; +; \lambda)$ a $T$ test feletti vektortér, ha az alábbiak teljesülnek:
  \begin{itemize}
    \def\arraystretch{1.2}
    \item $(V; +)$ Abel csoport:\\[1mm]
          \begin{tabular}{p{35mm} l}
            -- asszociatív:        &
            $\rvec a + (\rvec b + \rvec c) = (\rvec a + \rvec b) + \rvec c$,
            \\
            -- kommutatív:         &
            $\rvec a + \rvec b = \rvec b + \rvec a$,
            \\
            -- létezik zérus elem: &
            $\exists \nvec \in V \text{, melyre } \rvec a + \nvec = \rvec a$,
            \\
            -- létezik inverz:     &
            $\forall \rvec a \text{-re } \exists - \rvec a \text{, hogy } \rvec v + (- \rvec v) = \nvec$.
          \end{tabular}
    \item $(V; \lambda)$-ra pedig igaz:\\[1mm]
          \begin{tabular}{p{35mm} l}
            -- asszociatív:      & $(\alpha \beta) \rvec a = \alpha (\beta \rvec a)$.
            \\
            -- egységelem:       & $1 \in T \text{-re } 1\rvec a = \rvec a$,
            \\
            -- disztributivitás: & $\alpha(\rvec a + \rvec b) = \alpha \rvec a + \alpha \rvec b$,
            \\
                                 & $(\alpha + \beta) \rvec a = \alpha \rvec a + \beta \rvec a$.
          \end{tabular}
  \end{itemize}
\end{frame}

\begin{frame}
  \frametitle{Vektorterek}
  \framesubtitle{Vektorteres feladat}

  \begin{exercise}{%
		Döntsük el, hogy vektorteret alkotnak-e az alábbi számhármasok
		$\mathbb R^3$-ban?
	}
	\newcommand{\setmap}[2]{\Big\{\, #1 \,\Big|\, #2 \,\Big\}}
	\begin{enumerate}[a)]
		\item $Q_1 = \setmap{(x_1; x_2; x_3)}{x_1 + x_2 = 0}$
		\item $Q_2 = \setmap{(x_1; x_2; x_3)}{x_1 = \pi}$
		\item $Q_3 = \setmap{(x_1; x_2; x_3)}{x_1 = x_2 = x_3}$
		\item $Q_3 = \setmap{(x_1; x_2; x_3)}{x_1 = (x_2)^2}$
	\end{enumerate}

	\exsol{%
		\begin{enumerate}[a)]
			\item Igen, mert a műveletek sosem mutatnak ki a vektortérből, hiszen ha\\
			      $\alpha(x_{11} + x_{12}) = 0$,
			      akkor $\alpha x_{11} + \alpha x_{12} = 0$.
			\item Nem, hiszen $\pi + \pi \neq \pi$.
			\item Igen, hiszen a műveletek sosem mutatnak ki a vektortérből.
			\item Nem, hiszen $(a + b)^2 \neq a^2 + b^2$.
		\end{enumerate}
	}
\end{exercise}

\end{frame}

\begin{frame}
  \frametitle{Vektorterek}
  \framesubtitle{További definíció}

  \begin{block}{Altér}
    Legyen $(V; +; \lambda)$ a $T$ test feletti vektortér, és
    $\emptyset \neq L \subset V$. $L$-t altérnek nevezzük $V$-ben, ha
    $(L; +; \lambda)$ ugyancsak vektortér.
  \end{block}

  \begin{block}{Generátorrendszer}
    Legyen $\emptyset \neq G \subset V$. Ekkor $G$ által generált altérnek
    nevezzük azt a legszűkebb alteret, amely tartalmazza $G$-t. Ha ez az
    altér maga $V$, akkor $G$ generátorrendszere $V$-nek. ($\mathcal L(G)=V$)
  \end{block}

  \begin{block}{Bázis}
    A $V$ vektortér egy lineárisan független generátorrendszerét a $V$ bázisának
    hívjuk.
  \end{block}
\end{frame}

\begin{frame}
  \frametitle{Vektorterek}
  \framesubtitle{Generátoros feladat}

  \vspace{-7.5mm}
  \input{exercise/span.tex}
\end{frame}

\section{Mátrixok}
\begin{frame}
  \frametitle{Mátrixok}
  \framesubtitle{Alapfogalmak}

  \vfill

  \begin{block}{Mátrix}
    Az $m$ sorba és $n$ oszlopba rendezett rendezett számokat mátrixoknak
    nevezzük.
  \end{block}

  \vfill

  \begin{block}{Transzponált}
    Egy mátrix transzponáltja a főátlóra való tükörképe. Jele: $A^\mathsf T$.
  \end{block}

  \vfill

  \begin{block}{Szimmetrikus mátrix}
    Ha $A = A^\mathsf T$, akkor a mátrix szimmetrikus.
  \end{block}

  \vfill

  \begin{block}{Antiszimmetrikus mátrix}
    Ha $A = -A^\mathsf T$, akkor a mátrix antiszimmetrikus.
  \end{block}

  \vfill
\end{frame}

\begin{frame}
  \frametitle{Mátrixok}
  \framesubtitle{Elemi mátrixműveletek}

  \vfill

  \begin{block}{Összeadás}
    Ha $\rmat A; \rmat B \in \mathbb R^{n \times k}$, akkor az összegükön a
    megfelelő elempárok összeadásával keletkező mátrixot értjük.
  \end{block}

  \vfill

  \begin{block}{Skalárral való szorzás}
    Egy mátrix és egy skalár szorzata olyan mátrix, melynek minden eleme
    skalárszorosa az eredeti mátrix elemeinek.
  \end{block}

  \vfill
\end{frame}

\begin{frame}
  \frametitle{Mátrixok}
  \framesubtitle{Elemi mátrixműveletek}

  \vspace{-7.5mm}
  \begin{block}{Mátrix szorzás
      -- asszociatív, disztributív, de nem kommutatív!}
    \[
      \left.\begin{array}{ll}
        \rmat A \in \mathbb R^{m \times n} \\
        \rmat B \in \mathbb R^{n \times p}
      \end{array}\right\}
      \; \rightarrow \;
      \rmat A \cdot \rmat B \in \mathbb R^{m \times p}
    \]

    \def\arraystretch{1.1}
    \begin{align*}
       & \left[\begin{array}{X{2cm}cX{2cm}}
                   b_{11} & \dots  & b_{1p} \\
                   b_{21} & \dots  & b_{2p} \\
                   \vdots & \ddots & \vdots \\
                   b_{n1} & \dots  & b_{np}
                 \end{array}\right]
      \\
      \left[\begin{array}{cccc}
                a_{11} & a_{12} & \dots  & a_{1n} \\
                a_{21} & a_{22} & \dots  & a_{2n} \\
                \vdots & \vdots & \ddots & \vdots \\
                a_{m1} & a_{m2} & \dots  & a_{mn}
              \end{array}\right]
       & \left[\begin{array}{X{2cm}cX{2cm}}
                   \sum a_{1i} b_{i1} & \dots & \sum a_{1i} b_{ip} \\
                   \sum a_{2i} b_{i1} & \dots & \sum a_{2i} b_{ip} \\
                   \vdots             & \dots & \vdots             \\
                   \sum a_{mi} b_{i1} & \dots & \sum a_{mi} b_{ip}
                 \end{array}\right]
    \end{align*}
  \end{block}
\end{frame}

\begin{frame}
  \frametitle{Mátrixok}
  \framesubtitle{Determináns}

  \vspace{-5mm}
  \begin{block}{Kifejtési tétel -- előjelszabály!}
    \begin{align*}
      \begin{vmatrix}
        + & - & + \\
        - & + & - \\
        + & - & + \\
      \end{vmatrix}
      \; \rightarrow \;
      \begin{vmatrix}
        a & b & c \\
        d & e & f \\
        g & h & i \\
      \end{vmatrix}
       & = a \begin{vmatrix}
               e & f \\ h & i
             \end{vmatrix}
      - b \begin{vmatrix}
            d & f \\ g & i
          \end{vmatrix}
      + c \begin{vmatrix}
            d & e \\ g & h
          \end{vmatrix}
      \\
       & = a (ei - hf) - b(di - gf) + c(dh - eg)
    \end{align*}
  \end{block}

  \begin{block}{Sarrus-szabály -- csak ($3 \times 3$)-as mátrixoknál!}
    \centering
    \begin{tikzpicture}[ampersand replacement=\&]
      \matrix[
        matrix of math nodes,
        column sep=2mm,
      ] (sarrus) {
        a\vphantom{b} \& b \& c\vphantom{b} \& a \& b \\
        d \& e \& f \& d \& e                         \\
        g \& h\vphantom{g} \& i\vphantom{g} \& g \& h \\
      };

      \draw[red!40!gray, ultra thick, opacity=.5]
      (sarrus-1-1.center) -- (sarrus-3-3.center)
      (sarrus-1-2.center) -- (sarrus-3-4.center)
      (sarrus-1-3.center) -- (sarrus-3-5.center)
      ;

      \draw[blue!40!gray, ultra thick, opacity=.5]
      (sarrus-3-1.center) -- (sarrus-1-3.center)
      (sarrus-3-2.center) -- (sarrus-1-4.center)
      (sarrus-3-3.center) -- (sarrus-1-5.center)
      ;

      \draw[black, thick]
      (sarrus-1-1.north west) -- (sarrus-3-1.south west)
      (sarrus-1-3.north east) -- (sarrus-3-3.south east)
      ;

      \foreach \i in {1,2,3}{
          \node[above=-2.5mm, red!40!gray] at (sarrus-1-\i.north) {$+$};
          \node[below=-1.5mm, blue!40!gray] at (sarrus-3-\i.south) {$-$};
        }

      \node[] at (5,.25) {$\det \rmat A = + aei + bfg + cdh$};
      \node[] at (5,-.25) {$\phantom{\det \rmat A =} - gec - hfa - idb$};
    \end{tikzpicture}
  \end{block}
\end{frame}

\begin{frame}
  \frametitle{Mátrixok}
  \framesubtitle{Rang}

  \begin{block}{Mátrix rangja}
    A mátrix rangjának nevezzük az oszlopvektorai közül a lineárisan függetlenek
    maximális számát. A mátrix rangja elemi átalakítások során nem változik.
    \begin{itemize}
      \item tetszőleges sorát vagy oszlopát egy 0-tól különböző számmal
            megszorozzuk,
      \item tetszőleges sorát vagy oszlopát felcseréljük,
      \item tetszőleges sorához vagy oszlopához egy másik tetszőleges sorát
            vagy oszlopát adjuk.
    \end{itemize}
  \end{block}
\end{frame}

\begin{frame}
  \frametitle{Mátrixok}
  \framesubtitle{Mátrixos feladatok}

  \vfill
  \input{exercise/determinant}
  \vfill
  \input{exercise/rank}
  \vfill
\end{frame}

\begin{frame}
  \frametitle{Mátrixok}
  \framesubtitle{Mátrix inverz}

  \vfill

  \begin{block}{Reguláris / Szinguláris mátrix}
    Egy kvadratikus ($\rmat A \in \mathbb R^{n \times n}$) mátrixot
    \textbf{reguláris}nak mondunk, ha determinánsa nem 0.
    \\[3mm]
    Ha a kvadratikus mátrix determinánsa 0, \textbf{szinguláris} mátrixról
    beszélünk.
  \end{block}

  \vfill

  \begin{block}{Inverz}
    Az $\rmat A \in \mathbb R^{n \times n}$ reguláris mátrix inverze alatt azt
    az $\rmat A^{-1} \in \mathbb R^{n \times n}$ mátrixot értjük, melyre
    $\rmat A \cdot \rmat A^{-1} = \imat$ egyenlőség teljesül.
    \\[3mm]
    Szinguláris mátrixnak nem létezik az inverze.
  \end{block}

  \vfill
\end{frame}

\begin{frame}
  \frametitle{Mátrixok}
  \framesubtitle{Inverz meghatározása}

  \begin{block}{Adjugált mátrix segítségével}
    \[
      \rmat A^{-1} = \frac{\adj \rmat A}{\det \rmat A}
      \hspace{1cm}
      \rmat A = \begin{bmatrix}
        a_{11} & a_{12} & a_{13} \\
        a_{21} & a_{22} & a_{23} \\
        a_{31} & a_{32} & a_{33}
      \end{bmatrix}
    \]
    \[
      \adj \rmat A = \begin{bmatrix}
        + \begin{vmatrix}
            a_{22} & a_{23} \\
            a_{32} & a_{33}
          \end{vmatrix}
         &
        - \begin{vmatrix}
            a_{12} & a_{13} \\
            a_{32} & a_{33}
          \end{vmatrix}
         &
        +\begin{vmatrix}
           a_{12} & a_{13} \\
           a_{22} & a_{23}
         \end{vmatrix}
        \\
        - \begin{vmatrix}
            a_{21} & a_{23} \\
            a_{31} & a_{33}
          \end{vmatrix}
         &
        + \begin{vmatrix}
            a_{11} & a_{13} \\
            a_{31} & a_{33}
          \end{vmatrix}
         &
        - \begin{vmatrix}
            a_{11} & a_{13} \\
            a_{21} & a_{23}
          \end{vmatrix}
        \\
        + \begin{vmatrix}
            a_{21} & a_{22} \\
            a_{31} & a_{32}
          \end{vmatrix}
         &
        - \begin{vmatrix}
            a_{11} & a_{12} \\
            a_{31} & a_{32}
          \end{vmatrix}
         &
        +\begin{vmatrix}
           a_{11} & a_{12} \\
           a_{21} & a_{22}
         \end{vmatrix}
      \end{bmatrix}
    \]
  \end{block}
\end{frame}

\begin{frame}
  \frametitle{Mátrixok}
  \framesubtitle{Inverz meghatározása}

  \vfill

  \begin{block}{Gauss-Jordan eliminációval}
    \[
      \left[\begin{array}{*{3}{>{\cdot}{c}}|*{3}c}
           &  &  & 1 & 0 & 0 \\
           &  &  & 0 & 1 & 0 \\
           &  &  & 0 & 0 & 1 \\
        \end{array}\right]
      \quad \sim \quad
      \left[\begin{array}{*{3}{c}|*{3}{>{\cdot}{c}}}
          1 & 0 & 0 &  &  & \\
          0 & 1 & 0 &  &  & \\
          0 & 0 & 1 &  &  & \\
        \end{array}\right]
    \]
  \end{block}

  \vfill

  \input{exercise/inverse.tex}

  \vfill
\end{frame}

\begin{frame}
  \frametitle{Mátrixok}
  \framesubtitle{Mátrix egyenletek}

  \begin{exercise}{Oldjuk meg az alábbi mátrix-egyenleteket!}
	\[
		\rmat A = \begin{bmatrix}
			2 & 3 \\
			3 & 5
		\end{bmatrix}
		\hspace{1cm}
		\rmat B = \begin{bmatrix}
			1 & 2 \\
			3 & 4 \\
			6 & 3
		\end{bmatrix}
		\hspace{1cm}
		\rmat Q = \begin{bmatrix}
			\sqrt{3}/2 & -1/2       \\
			1/2        & \sqrt{3}/2
		\end{bmatrix}
	\]
	\begin{multicols}{2}
		\begin{enumerate}
			\item $\rmat X \cdot \rmat A = \rmat B$
			\item $\rmat A \cdot \rmat X = \rmat B$
			\item $2(\rmat A + \rmat X) = 3(\rmat A^{-1} + \rmat X)$
			\item $\rmat X = \rmat A \cdot \rmat Q^{6}$
		\end{enumerate}
	\end{multicols}

	\exsol{
		\begin{enumerate}
			\item $\rmat X \cdot \rmat A = \rmat B$

			      Rendezzük $\rmat X$-re az egyenletet, vagyis szorozzuk meg az
			      egyenlet mindkét oldalát $\rmat A$ inverzével. Ekkor az alábbi
			      egyenletet kapjuk:
			      \[
				      \rmat X = \rmat B \cdot \rmat A^{-1}
				      \text{.}
			      \]
			      Az $\rmat A$ mátrix inverzét már korábban meghatároztuk.
			      Az egyenletbe behelyettesítve:
			      \[
				      \rmat X
				      = \rmat B \cdot\rmat A^{-1} = \begin{bmatrix}
					      1 & 2 \\
					      3 & 4 \\
					      6 & 3
				      \end{bmatrix} \begin{bmatrix}
					      5  & -3 \\
					      -3 & 2
				      \end{bmatrix} = \begin{bmatrix}
					      5 - 6   & -3 + 4  \\
					      15 - 12 & -9 + 8  \\
					      30-9    & -18 + 6
				      \end{bmatrix} = \begin{bmatrix}
					      -1 & 1   \\
					      3  & -1  \\
					      21 & -12
				      \end{bmatrix}
				      \text.
			      \]

			      \tcbline
			\item $\rmat A \cdot \rmat X = \rmat B$
			      Ez az egyenlet nem megoldható.

			      \tcbline
			\item $2(\rmat A + \rmat X) = 3(\rmat A^{-1} + \rmat X)$

			      \tcbline
			\item $\rmat X = \rmat A \cdot \rmat Q^{6}$
		\end{enumerate}
	}
\end{exercise}

\end{frame}

% System of linear equations
\section{Lineáris egyenletrendszerek}
\begin{frame}
  \frametitle{Lineáris egyenletrendszerek}
  \framesubtitle{Definíció}

  \begin{block}{Lineáris egyenletrendszer}
    Véges sok elsőfokú egyenletet és véges sok ismeretlent tartalmazó
    egyenletrendszert lineáris egyenletrendszernek nevezünk.
    \\[2mm]
    Az $m$ egyenletből és $n$ ismeretlenből álló lineáris egyenletrendszer
    általános alakja:
    \[
      \begin{array}{*{9}{c}}
        a_{11} x_{1} & + & a_{12} x_{2} & + & \dots  & + & a_{1n} x_{n} & = & b_{1}\text, \\[1mm]
        a_{21} x_{1} & + & a_{22} x_{2} & + & \dots  & + & a_{2n} x_{n} & = & b_{2}\text, \\[1mm]
        \vdots       &   & \vdots       &   & \vdots &   & \vdots       &   & \vdots      \\[1mm]
        a_{m1} x_{1} & + & a_{m2} x_{2} & + & \dots  & + & a_{mn} x_{n} & = & b_{m}\text.
      \end{array}
    \]
  \end{block}
\end{frame}

\begin{frame}
  \frametitle{Lineáris egyenletrendszerek}
  \framesubtitle{Mátrixos alak}

  \begin{block}{LER mátrixos alakja}
    Egy lineáris egyenletrendszer felírható $\rmat A \rvec x = \rmat b$
    alakban, ahol $\rmat A$ az együttható mátrix, $\rvec x$ az ismeretlenek
    vektora, $\rvec b$ pedig a konstans vektor.

    \[
      \underbrace{\begin{bmatrix}
          a_{11} & a_{12} & \cdots & a_{1n} \\
          a_{21} & a_{22} & \cdots & a_{2n} \\
          \vdots & \vdots & \ddots & \vdots \\
          a_{m1} & a_{m2} & \cdots & a_{mn}
        \end{bmatrix}}_{\rmat A} \underbrace{\begin{bmatrix}
          x_{1} \\ x_{2} \\ \vdots \\ x_{n}
        \end{bmatrix}}_{\rvec x} = \underbrace{\begin{bmatrix}
          b_{1} \\ b_{2} \\ \vdots \\ b_{m}
        \end{bmatrix}}_{\rvec b}
    \]
  \end{block}
\end{frame}

\begin{frame}
  \frametitle{Lineáris egyenletrendszerek}
  \framesubtitle{Megoldhatóság}

  \vspace{-7.5mm}
  \begin{block}{LER megoldhatóságának szükséges és elégséges feltétele}
    Az $\rmat A \rvec x = \rvec b$ lineáris egyenletrendszer akkor és csak
    akkor oldható meg, ha $\rg(\rmat A) = \rg(\rmat A | \rvec b)$, ahol az
    $(\rmat A | \rvec b)$ mátrixot kibővített mátrixnak nevezzük.

    A feltétel mátrixosan:
    \[
      \rg \begin{bmatrix}
        a_{11} & a_{12} & \cdots & a_{1n} \\
        a_{21} & a_{22} & \cdots & a_{2n} \\
        \vdots & \vdots & \ddots & \vdots \\
        a_{m1} & a_{m2} & \cdots & a_{mn}
      \end{bmatrix} = \rg \left[\begin{array}{cccc|c}
          a_{11} & a_{12} & \cdots & a_{1n} & b_1    \\
          a_{21} & a_{22} & \cdots & a_{2n} & b_2    \\
          \vdots & \vdots & \ddots & \vdots & \vdots \\
          a_{m1} & a_{m2} & \cdots & a_{mn} & b_n
        \end{array}\right]\text.
    \]

    A feltételből következik, hogy homogén lineáris egyenletrendszer
    ($\rvec b = \nvec$) mindig megoldható, hiszen az együttható mátrixból és egy
    nullvektorból képzett kibővített mátrix rangja mindig meg fog egyezni az
    együttható mátrix rangjával.
  \end{block}
\end{frame}

\begin{frame}
  \frametitle{Lineáris egyenletrendszerek}
  \framesubtitle{Megoldási módszerek I}

  \begin{block}{Mátrix inverziós módszer}
    Ha az $\rmat A$ mátrix kvadratikus és reguláris, akkor invertálható:
    \[
      \rvec x = \rmat A^{-1} \rvec b
      \text.
    \]
  \end{block}

  \begin{block}{Cramer-szabály}
    Ha az $\rmat A$ mátrix kvadratikus és reguláris, akkor az együtthatók az
    alábbi módon számíthatóak:
    \[
      x_i = \frac{\det \rmat A_i}{\det \rmat A}
      \text,
    \]
    ahol az $\rmat A_i$ mátrixot úgy képezzük, hogy az $i$-edig sorába
    $\rvec b$ vektort írjuk be.
  \end{block}
\end{frame}

\begin{frame}
  \frametitle{Lineáris egyenletrendszerek}
  \framesubtitle{Megoldási módszerek II}

  \vspace{-7.5mm}
  \begin{block}{Gauss elimináció}
    Sorműveletekkel alakítjuk a kibővített mátrixot:
    \[
      \left[\begin{array}{cccc|c}
          a_{11} & a_{12} & \cdots & a_{1n} & b_1    \\
          a_{21} & a_{22} & \cdots & a_{2n} & b_2    \\
          \vdots & \vdots & \ddots & \vdots & \vdots \\
          a_{m1} & a_{m2} & \cdots & a_{mn} & b_n
        \end{array}\right]
      \quad\sim\quad
      \left[\begin{array}{cccc|c}
          \square & \square & \cdots & \square & \square \\
          0       & \square & \cdots & \square & \square \\
          \vdots  & \vdots  & \ddots & \vdots  & \vdots  \\
          0       & 0       & \cdots & \circ   & \circ
        \end{array}\right]
    \]
  \end{block}

  \begin{block}{Megoldások száma $\rmat A \in \mathbb R^{n \times n}$ esetben}
    \centering
    \begin{tikzpicture}[thick]
      \foreach \i in {0,1,2} {
          \node at (\i*4cm,0) {$\left[\phantom{\begin{matrix}10000000000\\\\\\\\\end{matrix}}\right]$};

          \draw (\i*4cm+4mm,.75cm) -- ++(0,-1.5cm);
        }

      \draw (5.35mm,7mm) rectangle ++(5mm, -14mm);
      \draw (45.35mm,7mm) rectangle ++(5mm, -14mm);
      \draw (85.35mm,7mm) rectangle ++(5mm, -9.25mm) node[below left] {$0$};

      \draw (2.65mm,7mm)
      -- ++(-13.75mm,0)
      -- ++(6.75mm,-9.25mm)
      -- ++(7mm,0)
      node[midway, below] {$0$}
      -- cycle;
      \draw (42.65mm,7mm)
      -- ++(-13.75mm,0)
      -- ++(13.75mm,-13.75mm)
      node[midway, below left=2.25mm] {$\quad0$}
      -- cycle;
      \draw (82.65mm,7mm)
      -- ++(-13.75mm,0)
      -- ++(6.75mm,-9.25mm)
      -- ++(7mm,0)
      node[midway, below] {$0$}
      -- cycle;

      \node[] at (0cm,1cm) {Nincs mo.};
      \node[] at (4cm,1cm) {1db mo.};
      \node[] at (8cm,1cm) {$\infty$ mo.};
    \end{tikzpicture}
  \end{block}
\end{frame}

\begin{frame}
  \frametitle{Lineáris egyenletrendszerek}
  \framesubtitle{Feladatok}

  \vfill
  \begin{exercise}{Oldjuk meg az alábbi lineáris egyenletrendszert!}
	\[
		\begin{array}{*{7}{c}}
			1 x_{1} & + & 4 x_{2} & + & 8 x_{3} & = & 23 \\
			        &   & 1 x_{2} &   &         & = & 1  \\
			2 x_{1} & + & 4 x_{2} & + & 6 x_{3} & = & 22
		\end{array}
	\]
\end{exercise}

  \vfill
  \input{exercise/cramer}
  \vfill
\end{frame}

% Linear mappings
\section{Lineáris leképezések}
\begin{frame}
  \frametitle{Lineáris leképezések}
  \framesubtitle{Alapfogalmak I}

  \begin{block}{Lineáris leképezés}
    Legyenek $V_1$ és $V_2$ ugyanazon $T$ test feletti vektorterek. Legyen
    $\varphi: V_1 \rightarrow V_2$ leképezés, melyet lineáris leképezésnek
    nevezünk, ha tetszőleges két $V_1$-beli vektor ($\forall \rvec a; \rvec b \in
      V_1$) és $T$-beli skalár ($\alpha \in T$) esetén teljesülnek az alábbiak:
    \begin{itemize}
      \item $\varphi(\rvec a + \rvec b) = \varphi(\rvec a) + \varphi(\rvec b)$
            $\quad \sim \quad$ összegre tagonként hat,
      \item $\varphi(\alpha \rvec a) = \alpha \varphi(\rvec a)$
            $\hspace{18mm} \sim \quad$ skalár kiemelhető.
    \end{itemize}
  \end{block}
\end{frame}

\begin{frame}
  \frametitle{Lineáris leképezések}
  \framesubtitle{Alapfogalmak II}

  \vfill
  \begin{block}{Magtér}
    Legyen $\varphi: V_1 \rightarrow V_2$ lineáris leképezés. A leképezés
    magtere:
    \[
      \ker \varphi = \Big\{\;
      \rvec v \;\Big|\; \rvec v \in V_1 \;\land\;
      \varphi(\rvec v) = \nvec
      \;\Big\}
      \text.
    \]
  \end{block}
  \vfill
  \begin{block}{Defektus}
    A magtér dimenzióját a leképezés defektusának nevezzük:
    \[
      \dim \ker \varphi = \defect \varphi
      \text.
    \]
  \end{block}
  \vfill
\end{frame}

\begin{frame}
  \frametitle{Lineáris leképezések}
  \framesubtitle{Alapfogalmak III}

  \vfill
  \begin{block}{Képtér}
    Egy $\varphi: V_1 \rightarrow V_2$ lineáris leképezés rangjának nevezzük
    a képtér dimenzióját:
    \[
      \rg \varphi = \dim V_2
      \text.
    \]
  \end{block}
  \vfill
  \begin{block}{Rang nullitás tétele}
    Legyen $V_1$ véges dimenziós vektortér, $\varphi: V_1 \rightarrow V_2$
    lineáris leképezés, ekkor:
    \[
      \rg \varphi + \defect \varphi = \dim V_1
    \]
  \end{block}
  \vfill
\end{frame}

\begin{frame}
  \frametitle{Lineáris leképezések}
  \framesubtitle{Feladatok I}

  \input{exercise/linear-map-tf}
\end{frame}

\begin{frame}
  \frametitle{Lineáris leképezések}
  \framesubtitle{Feladatok II}

  \input{exercise/linear-map}
\end{frame}

% Rotations etc
\begin{frame}
  \frametitle{Lineáris leképezések}
  \framesubtitle{Alap geometriai leképezések I}

  \begin{block}{Tükrözés valamelyik tengelyre}
    \[
      \rmat A_x = \overset{x\text{-tengelyre}}{\begin{bmatrix}
          1 & 0  & 0  \\
          0 & -1 & 0  \\
          0 & 0  & -1
        \end{bmatrix}}
      \text,
      \quad
      \rmat A_y = \overset{y\text{-tengelyre}}{\begin{bmatrix}
          -1 & 0 & 0  \\
          0  & 1 & 0  \\
          0  & 0 & -1
        \end{bmatrix}}
      \text,
      \quad
      \rmat A_z = \overset{z\text{-tengelyre}}{\begin{bmatrix}
          -1 & 0  & 0 \\
          0  & -1 & 0 \\
          0  & 0  & 1
        \end{bmatrix}}
      \text.
    \]
  \end{block}

  \begin{block}{Tükrözés valamelyik síkra}
    \[
      \rmat A_{xy} = \overset{xy\text{-síkra}}{\begin{bmatrix}
          1 & 0 & 0  \\
          0 & 1 & 0  \\
          0 & 0 & -1
        \end{bmatrix}}
      \text,
      \quad
      \rmat A_{xz} = \overset{xz\text{-síkra}}{\begin{bmatrix}
          1 & 0  & 0 \\
          0 & -1 & 0 \\
          0 & 0  & 1
        \end{bmatrix}}
      \text,
      \quad
      \rmat A_{yz} = \overset{yz\text{-síkra}}{\begin{bmatrix}
          1 & 0 & 0  \\
          0 & 1 & 0  \\
          0 & 0 & -1
        \end{bmatrix}}
      \text.
    \]
  \end{block}
\end{frame}

\begin{frame}
  \frametitle{Lineáris leképezések}
  \framesubtitle{Alap geometriai leképezések II}

  \begin{block}{Vetítés valamelyik tengelyre}
    \[
      \rmat A_x = \overset{x\text{-tengelyre}}{\begin{bmatrix}
          1 & 0 & 0 \\
          0 & 0 & 0 \\
          0 & 0 & 0
        \end{bmatrix}}
      \text,
      \quad
      \rmat A_y = \overset{y\text{-tengelyre}}{\begin{bmatrix}
          0 & 0 & 0 \\
          0 & 1 & 0 \\
          0 & 0 & 0
        \end{bmatrix}}
      \text,
      \quad
      \rmat A_z = \overset{z\text{-tengelyre}}{\begin{bmatrix}
          0 & 0 & 0 \\
          0 & 0 & 0 \\
          0 & 0 & 1
        \end{bmatrix}}
      \text.
    \]
  \end{block}

  \begin{block}{Vetítés valamelyik síkra}
    \[
      \rmat A_{xy} = \overset{xy\text{-síkra}}{\begin{bmatrix}
          1 & 0 & 0 \\
          0 & 1 & 0 \\
          0 & 0 & 0
        \end{bmatrix}}
      \text,
      \quad
      \rmat A_{xz} = \overset{xz\text{-síkra}}{\begin{bmatrix}
          1 & 0 & 0 \\
          0 & 0 & 0 \\
          0 & 0 & 1
        \end{bmatrix}}
      \text,
      \quad
      \rmat A_{yz} = \overset{yz\text{-síkra}}{\begin{bmatrix}
          1 & 0 & 0 \\
          0 & 1 & 0 \\
          0 & 0 & 0
        \end{bmatrix}}
      \text.
    \]
  \end{block}
\end{frame}

\begin{frame}
  \frametitle{Lineáris leképezések}
  \framesubtitle{Alap geometriai leképezések III}

  \begin{block}{Forgatások -- ortogonális transzformációk}
    \[
      \begin{array}{X{67mm} c p{26mm}}
        \rmat A_x(\varphi) =
        \left[\begin{array}{*{3}{X{12mm}}}
                  1 & 0            & 0             \\
                  0 & \cos \varphi & -\sin \varphi \\
                  0 & \sin \varphi & \cos \varphi
                \end{array}\right]
         & -
         & $x$\text{-tengely körül,}
        \\
        \rmat A_y(\varphi) =
        \left[\begin{array}{*{3}{X{12mm}}}
                  \cos \varphi  & 0 & \sin \varphi \\
                  0             & 1 & 0            \\
                  -\sin \varphi & 0 & \cos \varphi
                \end{array}\right]
         & -
         & $y$\text{-tengely körül,}
        \\
        \rmat A_z(\varphi) =
        \left[\begin{array}{*{3}{X{12mm}}}
                  \cos \varphi & -\sin \varphi & 0 \\
                  \sin \varphi & \cos \varphi  & 0 \\
                  0            & 0             & 1
                \end{array}\right]
         & -
         & $z$\text{-tengely körül.}
      \end{array}
    \]
  \end{block}
\end{frame}

\begin{frame}
  \frametitle{Lineáris leképezések}
  \framesubtitle{Geometriai leképezések feladat}

  \input{exercise/rotation}
\end{frame}

% Eigenvalues and eigenvectors
\begin{frame}
  \frametitle{Lineáris leképezések}
  \framesubtitle{Sajátértékek és sajátvektorok I}

  \begin{block}{Definíció}
    Legyen $V$ a $T$ test feletti vektortér, $\rvec v \in V$, $\rvec v \neq
      \nvec$. $\rvec v$-t a $\varphi: V \rightarrow V$ lineáris leképezés
    sajátvektorának mondjuk, ha önmaga skalárszorosába megy át a leképezés
    során, azaz $\varphi(\rvec v) = \lambda \rvec v$,  $\lambda \in T$.
    $\lambda$-t a $\rvec v$ sajátvektorhoz tartozó sajátértéknek mondjuk.
  \end{block}
\end{frame}

\begin{frame}
  \frametitle{Lineáris leképezések}
  \framesubtitle{Sajátértékek és sajátvektorok II}

  \begin{block}{Sajátértékek számítása}
    Az $\rmat A$ mátrix sajátértékei a karakterisztikus egyenlet megoldásával
    határozhatóak meg:
    \[
      \det (\rmat A - \lambda \imat) = 0
      \text.
    \]
  \end{block}

  \begin{block}{Sajátvektorok meghatározása}
    Az egyes sajátértékekhez tartozó sajátvektorok az alábbi egyenlet alapján
    határozhatóak meg:
    \[
      (\rmat A - \lambda_i \imat) \rvec v_i = \nvec
      \text.
    \]
  \end{block}
\end{frame}

\begin{frame}
  \frametitle{Lineáris leképezések}
  \framesubtitle{Sajátértékek és sajátvektorok II}

  \input{exercise/eigen}
  \input{exercise/exponentiation}
\end{frame}


% Conic sections
\begin{frame}
  \frametitle{Lineáris leképezések}
  \framesubtitle{Másodfokú görbék}

  \begin{block}{Kvadratikus forma}
    Csupa másodfokú tagot tartalmazó polinomok átírhatóak mátrixos alakra:
    \[
      a x^2 + 2 b x y + c y^2
      =
      \begin{bmatrix} x & y \end{bmatrix}
      \begin{bmatrix} a & b \\ b & c \end{bmatrix}
      \begin{bmatrix} x \\ y \end{bmatrix}
      =
      \rvec x^{\mathsf T} \rmat A \rvec x
      \text.
    \]
  \end{block}

  \begin{block}{Mátrix definitsége}
    \centering
    \begin{tabular}{l>{$\sim}{c}<{$}>{$\forall \lambda_i}{c}<{$}}
      pozitív definit      &  & > 0    \\
      pozitív szemidefinit &  & \geq 0 \\
      negatív definit      &  & < 0    \\
      negatív szemidefinit &  & \leq 0 \\
    \end{tabular}
    \\[2mm]
    indefinit, ha az előzőek közül egyik sem
  \end{block}
\end{frame}

\end{document}
