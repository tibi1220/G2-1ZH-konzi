\section{Lineáris leképezések}
\begin{frame}
  \frametitle{Lineáris leképezések}
  \framesubtitle{Alapfogalmak I}

  \begin{block}{Lineáris leképezés}
    Legyenek $V_1$ és $V_2$ ugyanazon $T$ test feletti vektorterek. Legyen
    $\varphi: V_1 \rightarrow V_2$ leképezés, melyet lineáris leképezésnek
    nevezünk, ha tetszőleges két $V_1$-beli vektor ($\forall \rvec a; \rvec b \in
      V_1$) és $T$-beli skalár ($\alpha \in T$) esetén teljesülnek az alábbiak:
    \begin{itemize}
      \item $\varphi(\rvec a + \rvec b) = \varphi(\rvec a) + \varphi(\rvec b)$
            $\quad \sim \quad$ összegre tagonként hat,
      \item $\varphi(\alpha \rvec a) = \alpha \varphi(\rvec a)$
            $\hspace{18mm} \sim \quad$ skalár kiemelhető.
    \end{itemize}
  \end{block}
\end{frame}

\begin{frame}
  \frametitle{Lineáris leképezések}
  \framesubtitle{Alapfogalmak II}

  \vfill
  \begin{block}{Magtér}
    Legyen $\varphi: V_1 \rightarrow V_2$ lineáris leképezés. A leképezés
    magtere:
    \[
      \ker \varphi = \Big\{\;
      \rvec v \;\Big|\; \rvec v \in V_1 \;\land\;
      \varphi(\rvec v) = \nvec
      \;\Big\}
      \text.
    \]
  \end{block}
  \vfill
  \begin{block}{Defektus}
    A magtér dimenzióját a leképezés defektusának nevezzük:
    \[
      \dim \ker \varphi = \defect \varphi
      \text.
    \]
  \end{block}
  \vfill
\end{frame}

\begin{frame}
  \frametitle{Lineáris leképezések}
  \framesubtitle{Alapfogalmak III}

  \vfill
  \begin{block}{Képtér}
    Egy $\varphi: V_1 \rightarrow V_2$ lineáris leképezés rangjának nevezzük
    a képtér dimenzióját:
    \[
      \rg \varphi = \dim V_2
      \text.
    \]
  \end{block}
  \vfill
  \begin{block}{Rang nullitás tétele}
    Legyen $V_1$ véges dimenziós vektortér, $\varphi: V_1 \rightarrow V_2$
    lineáris leképezés, ekkor:
    \[
      \rg \varphi + \defect \varphi = \dim V_1
    \]
  \end{block}
  \vfill
\end{frame}

\begin{frame}
  \frametitle{Lineáris leképezések}
  \framesubtitle{Feladatok I}

  \begin{exercise}{Lineárisak-e az alábbi leképezések?}
  \begin{enumerate}[a)]
    \item \(
          \varphi: \mathbb R^2 \rightarrow \mathbb R^2;
          (x; y) \mapsto (e^x; y)
          \),
    \item \(
          \varphi: \mathbb R^2 \rightarrow \mathbb R^2;
          (x; y) \mapsto (42; 0)
          \),
    \item \(
          \varphi: \mathbb R^2 \rightarrow \mathbb R^2;
          (x; y) \mapsto (\arctan \tan x; -12)
          \),
    \item \(
          \varphi: \mathbb R^2 \rightarrow \mathbb R^2;
          (x; y) \mapsto (y + x; 2x)
          \),
  \end{enumerate}

  \exsol[4cm]{%
    \begin{enumerate}[a)]
      \item Nem lineáris, hiszen az $e^x$ függvény sem az.
      \item Nem lineáris, hiszen $\alpha \cdot 42 = 42$, nem igaz tetszőleges
            $\alpha$ számra,
      \item Nem lineáris, hiszen $\alpha \arctan \tan x \neq \arctan \tan \alpha x$.
      \item Lineáris, hiszen teljesülnek az alábbi feltételek:
    \end{enumerate}
    \[
      \varphi \begin{bmatrix}
        x_1 + x_2 \\ y_1 + y_2
      \end{bmatrix} = \begin{bmatrix}
        (y_1 + y_2) + (x_1 + x_2) \\ 2 (x_1 + x_2)
      \end{bmatrix} = \begin{bmatrix}
        y_1 + x_1 \\ 2x_1
      \end{bmatrix} + \begin{bmatrix}
        y_2 + x_2 \\ 2x_2
      \end{bmatrix} = \varphi \begin{bmatrix}
        x_1 \\ y_1
      \end{bmatrix} + \varphi \begin{bmatrix}
        x_1 \\ y_1
      \end{bmatrix}
      \text,
    \]\[
      \varphi \begin{bmatrix}
        \alpha x \\ \alpha y
      \end{bmatrix} = \begin{bmatrix}
        \alpha y + \alpha x \\ 2\alpha x
      \end{bmatrix} = \begin{bmatrix}
        \alpha(y + x) \\ \alpha(2x)
      \end{bmatrix} = \alpha \begin{bmatrix}
        y + x \\ 2x
      \end{bmatrix} = \alpha \varphi \begin{bmatrix}
        x \\ y
      \end{bmatrix}
      \text.
    \]
  }
\end{exercise}

\end{frame}

\begin{frame}
  \frametitle{Lineáris leképezések}
  \framesubtitle{Feladatok II}

  \begin{exercise}{%
    Adjuk meg az alábbi lineáris leképezés mátrixát a standard bázisban!
  }
  \[
    \varphi: \mathbb R^3 \rightarrow \mathbb R^3;
    \begin{bmatrix}
      x \\ y \\ z
    \end{bmatrix} \mapsto \begin{bmatrix}
      y + z \\
      x + z \\
      x + y
    \end{bmatrix}
  \]
  Adjuk meg továbbá:
  \begin{itemize}
    \item a képtér és magtér dimenzióját,
    \item a leképezés rangját és defektusát,
    \item a magtér egy tetszőleges elemét,
    \item az $(1; 2; 3)$ pont képét,
    \item a $(2; 2; 2)$ kép ősképét.
  \end{itemize}

  \exsol{%
    A leképezés mátrixa:
    \[
      \rmat A = \begin{bmatrix}
        0 & 1 & 1 \\
        1 & 0 & 1 \\
        1 & 1 & 0 \\
      \end{bmatrix}
      \text.
    \]

    Határozzuk meg a mátrix rangját a determinánsának számítása segítségével:
    \[
      \det \rmat A = 1 + 1 = 2
      \text.
    \]

    Mivel a mátrix reguláris ($\det \rmat A \neq 0$), ezért rangja maximális
    ($\rg \rmat A = 3$). A leképezés mátrix-reprezentációjának rangja megegyezik
    a leképezés rangjával és a képtér dimenziójával:
    \[
      \rg \rmat A = \rg \varphi = \dim V_2 = 3
      \text.
    \]
    A rang nullitás tételéből pedig következik, hogy:
    \[
      \defect \varphi = \dim \ker \varphi = \dim V_1 - \rg \varphi = 3 - 3 = 0
      \text.
    \]
    Mivel a leképezés defektusa 0, ezért a magtérnek egyetlen egy eleme van,
    mégpedig a nullvektor:
    \[
      \ker \varphi = \Big\{\; \nvec \;\Big\}
      \text.
    \]
    Az $(1;2;3)$ pont képe:
    \[
      \varphi \begin{bmatrix}
        1 \\ 2 \\ 3
      \end{bmatrix} = \begin{bmatrix}
        2 + 3 \\
        1 + 3 \\
        1 + 2
      \end{bmatrix} = \begin{bmatrix}
        5 \\ 4 \\3
      \end{bmatrix}
      \text.
    \]
    A $(2;2;2)$ kép ősképét meghatározhatjuk az $\rmat A$ mátrix invertálásával,
    vagy egy 3 ismeretből és 3 egyenletből álló lineáris egyenletrendszer
    megoldásával.
    \begin{enumerate}
      \newcommand\qadj[4]{\begin{vmatrix}#1&#2\\#3&#4\end{vmatrix}}
      \item Mátrix invertálásos módszer: \\[2mm]
            Határozzuk meg először az $\rmat A$ mátrix adjugáltját:
            \[
              \adj \rmat A = \begin{bmatrix}
                +\qadj{0}{1}{1}{0} & -\qadj{1}{1}{1}{0} & +\qadj{1}{0}{1}{1} \\
                -\qadj{1}{1}{1}{0} & +\qadj{0}{1}{1}{0} & -\qadj{0}{1}{1}{1} \\
                +\qadj{1}{1}{0}{1} & -\qadj{0}{1}{1}{1} & +\qadj{0}{1}{1}{0} \\
              \end{bmatrix} = \begin{bmatrix}
                -1 & 1  & 1  \\
                1  & -1 & 1  \\
                1  & 1  & -1 \\
              \end{bmatrix}
              \text.
            \]
            A mátrix inverze tehát:
            \[
              \rmat A^{-1}
              = \frac{\adj \rmat A}{\det \rmat A}
              = \frac{1}{2} \begin{bmatrix}
                -1 & 1  & 1  \\
                1  & -1 & 1  \\
                1  & 1  & -1 \\
              \end{bmatrix}
              \text.
            \]
            A keresett kép ősképe tehát:
            \[
              \varphi^{-1} \begin{bmatrix}
                2 \\ 2 \\ 2
              \end{bmatrix} = \frac{1}{2} \begin{bmatrix}
                -1 & 1  & 1  \\
                1  & -1 & 1  \\
                1  & 1  & -1 \\
              \end{bmatrix} \begin{bmatrix}
                2 \\ 2 \\ 2
              \end{bmatrix} = \frac{1}{2} \begin{bmatrix}
                -2 + 2 + 2 \\ 2 - 2 + 2 \\ 2 + 2 - 2
              \end{bmatrix} = \frac{1}{2} \begin{bmatrix}
                2 \\ 2 \\ 2
              \end{bmatrix} = \begin{bmatrix}
                1 \\ 1 \\ 1
              \end{bmatrix}
              \text.
            \]

      \item Lineáris egyenletrendszeres megoldás:
            \newcommand{\qgj}[6]{\left[\begin{array}{*{3}{X{6mm}}|X{6mm}}
                  #1 \\#3\\#5
                \end{array}\right]\begin{matrix}#2\\#4\\#6\end{matrix}}
            \begin{align*}
                 & \qgj
              {0 & 1    & 1  & 2}{}
              {1 & 0    & 1  & 2}{}
              {1 & 1    & 0  & 2}{}
              \\
                 & \qgj
              {1 & 0    & 1  & 2}{}
              {0 & 1    & 1  & 2}{}
              {1 & 1    & 0  & 2}{(-S_1)}
              \\
                 & \qgj
              {1 & 0    & 1  & 2}{}
              {0 & 1    & 1  & 2}{}
              {0 & 1    & -1 & 0}{(-S_2)}
              \\
                 & \qgj
              {1 & 0    & 1  & 2 }{}
              {0 & 1    & 1  & 2 }{}
              {0 & 0    & -2 & -2}{(/(-2))}
              \\
                 & \qgj
              {1 & 0    & 1  & 2}{(-S_3)}
              {0 & 1    & 1  & 2}{(-S_3)}
              {0 & 0    & 1  & 1}{}
              \\
                 & \qgj
              {1 & 0    & 0  & 1}{}
              {0 & 1    & 0  & 1}{}
              {0 & 0    & 1  & 1}{}
            \end{align*}
    \end{enumerate}
    Láthatjuk, hogy mindkét módszerrel ugyanazt az eredményt kaptuk.
  }
\end{exercise}

\end{frame}

% Rotations etc
\begin{frame}
  \frametitle{Lineáris leképezések}
  \framesubtitle{Alap geometriai leképezések I}

  \begin{block}{Tükrözés valamelyik tengelyre}
    \[
      \rmat A_x = \overset{x\text{-tengelyre}}{\begin{bmatrix}
          1 & 0  & 0  \\
          0 & -1 & 0  \\
          0 & 0  & -1
        \end{bmatrix}}
      \text,
      \quad
      \rmat A_y = \overset{y\text{-tengelyre}}{\begin{bmatrix}
          -1 & 0 & 0  \\
          0  & 1 & 0  \\
          0  & 0 & -1
        \end{bmatrix}}
      \text,
      \quad
      \rmat A_z = \overset{z\text{-tengelyre}}{\begin{bmatrix}
          -1 & 0  & 0 \\
          0  & -1 & 0 \\
          0  & 0  & 1
        \end{bmatrix}}
      \text.
    \]
  \end{block}

  \begin{block}{Tükrözés valamelyik síkra}
    \[
      \rmat A_{xy} = \overset{xy\text{-síkra}}{\begin{bmatrix}
          1 & 0 & 0  \\
          0 & 1 & 0  \\
          0 & 0 & -1
        \end{bmatrix}}
      \text,
      \quad
      \rmat A_{xz} = \overset{xz\text{-síkra}}{\begin{bmatrix}
          1 & 0  & 0 \\
          0 & -1 & 0 \\
          0 & 0  & 1
        \end{bmatrix}}
      \text,
      \quad
      \rmat A_{yz} = \overset{yz\text{-síkra}}{\begin{bmatrix}
          -1 & 0 & 0 \\
          0  & 1 & 0 \\
          0  & 0 & 1
        \end{bmatrix}}
      \text.
    \]
  \end{block}
\end{frame}

\begin{frame}
  \frametitle{Lineáris leképezések}
  \framesubtitle{Alap geometriai leképezések II}

  \begin{block}{Vetítés valamelyik tengelyre}
    \[
      \rmat A_x = \overset{x\text{-tengelyre}}{\begin{bmatrix}
          1 & 0 & 0 \\
          0 & 0 & 0 \\
          0 & 0 & 0
        \end{bmatrix}}
      \text,
      \quad
      \rmat A_y = \overset{y\text{-tengelyre}}{\begin{bmatrix}
          0 & 0 & 0 \\
          0 & 1 & 0 \\
          0 & 0 & 0
        \end{bmatrix}}
      \text,
      \quad
      \rmat A_z = \overset{z\text{-tengelyre}}{\begin{bmatrix}
          0 & 0 & 0 \\
          0 & 0 & 0 \\
          0 & 0 & 1
        \end{bmatrix}}
      \text.
    \]
  \end{block}

  \begin{block}{Vetítés valamelyik síkra}
    \[
      \rmat A_{xy} = \overset{xy\text{-síkra}}{\begin{bmatrix}
          1 & 0 & 0 \\
          0 & 1 & 0 \\
          0 & 0 & 0
        \end{bmatrix}}
      \text,
      \quad
      \rmat A_{xz} = \overset{xz\text{-síkra}}{\begin{bmatrix}
          1 & 0 & 0 \\
          0 & 0 & 0 \\
          0 & 0 & 1
        \end{bmatrix}}
      \text,
      \quad
      \rmat A_{yz} = \overset{yz\text{-síkra}}{\begin{bmatrix}
          0 & 0 & 0 \\
          0 & 1 & 0 \\
          0 & 0 & 1
        \end{bmatrix}}
      \text.
    \]
  \end{block}
\end{frame}

\begin{frame}
  \frametitle{Lineáris leképezések}
  \framesubtitle{Alap geometriai leképezések III}

  \begin{block}{Forgatások -- ortogonális transzformációk}
    \[
      \begin{array}{X{67mm} c p{26mm}}
        \rmat A_x(\varphi) =
        \left[\begin{array}{*{3}{X{12mm}}}
                  1 & 0            & 0             \\
                  0 & \cos \varphi & -\sin \varphi \\
                  0 & \sin \varphi & \cos \varphi
                \end{array}\right]
         & -
         & $x$\text{-tengely körül,}
        \\
        \rmat A_y(\varphi) =
        \left[\begin{array}{*{3}{X{12mm}}}
                  \cos \varphi  & 0 & \sin \varphi \\
                  0             & 1 & 0            \\
                  -\sin \varphi & 0 & \cos \varphi
                \end{array}\right]
         & -
         & $y$\text{-tengely körül,}
        \\
        \rmat A_z(\varphi) =
        \left[\begin{array}{*{3}{X{12mm}}}
                  \cos \varphi & -\sin \varphi & 0 \\
                  \sin \varphi & \cos \varphi  & 0 \\
                  0            & 0             & 1
                \end{array}\right]
         & -
         & $z$\text{-tengely körül.}
      \end{array}
    \]
  \end{block}
\end{frame}

\begin{frame}
  \frametitle{Lineáris leképezések}
  \framesubtitle{Geometriai leképezések feladat}

  \begin{exercise}{%
    Mi lesz a $P (2;6;8)$ pont képe, ha hattatjuk rá az alábbi leképezéseket?
  }
  \begin{itemize}
    \item először tükrözzük az $x$ tengelyre,
    \item majd az $y$ tengely körül $-60^\circ$-kal forgatjuk,
    \item végül $y$ irányban a $-5$-szörösére nyújtjuk.
  \end{itemize}

  \exsol{%
    \[
      \begin{array}{rc}
         & \left[\begin{array}{X{2cm}} 2 \\ 6 \\ 8 \end{array}\right]
        \\
        \left[\begin{array}{*{3}{X{22mm}}}
                  1 & 0  & 0  \\
                  0 & -1 & 0  \\
                  0 & 0  & -1
                \end{array}\right]
         & \left[\begin{array}{X{2cm}} 1 + 4 \sqrt{3} \\ -6 \\ \sqrt{3} - 4 \end{array}\right]
        \\
        \left[\begin{array}{*{3}{X{22mm}}}
                  \cos (-60^\circ)  & 0 & \sin (-60^\circ) \\
                  0                 & 1 & 0                \\
                  -\sin (-60^\circ) & 0 & \cos (-60^\circ)
                \end{array}\right]
         & \left[\begin{array}{X{2cm}} 2 \\ -6 \\ 8 \end{array}\right]
        \\
        \left[\begin{array}{*{3}{X{22mm}}}
                  1 & 0 & 0 \\
                  0 & 1 & 0 \\
                  0 & 0 & 5
                \end{array}\right]
         & \left[\begin{array}{X{2cm}} 1 + 4 \sqrt{3} \\ 30 \\ \sqrt{3} - 4 \end{array}\right]
      \end{array}
    \]
  }
\end{exercise}

\end{frame}

% Eigenvalues and eigenvectors
\begin{frame}
  \frametitle{Lineáris leképezések}
  \framesubtitle{Sajátértékek és sajátvektorok I}

  \begin{block}{Definíció}
    Legyen $V$ a $T$ test feletti vektortér, $\rvec v \in V$, $\rvec v \neq
      \nvec$. $\rvec v$-t a $\varphi: V \rightarrow V$ lineáris leképezés
    sajátvektorának mondjuk, ha önmaga skalárszorosába megy át a leképezés
    során, azaz $\varphi(\rvec v) = \lambda \rvec v$,  $\lambda \in T$.
    $\lambda$-t a $\rvec v$ sajátvektorhoz tartozó sajátértéknek mondjuk.
  \end{block}
\end{frame}

\begin{frame}
  \frametitle{Lineáris leképezések}
  \framesubtitle{Sajátértékek és sajátvektorok II}

  \begin{block}{Sajátértékek számítása}
    Az $\rmat A$ mátrix sajátértékei a karakterisztikus egyenlet megoldásával
    határozhatóak meg:
    \[
      \det (\rmat A - \lambda \imat) = 0
      \text.
    \]
  \end{block}

  \begin{block}{Sajátvektorok meghatározása}
    Az egyes sajátértékekhez tartozó sajátvektorok az alábbi egyenlet alapján
    határozhatóak meg:
    \[
      (\rmat A - \lambda_i \imat) \rvec v_i = \nvec
      \text.
    \]
  \end{block}
\end{frame}

\begin{frame}
  \frametitle{Lineáris leképezések}
  \framesubtitle{Sajátértékek és sajátvektorok II}

  \begin{exercise}{%
    Határozzuk meg az alábbi mátrixok sajátértékeit és sajátvektorait!
    A sajátvektorok hosszai legyenek egységnyiek, valamint az első koordinátájuk
    legyen pozitív!
  }
  \[
    \rmat A = \begin{bmatrix}
      2 & 3 \\
      6 & 5
    \end{bmatrix}
    \hspace{1cm}
    \rmat B = \begin{bmatrix}
      1 & 0 \\
      0 & 1
    \end{bmatrix}
    \hspace{1cm}
    \rmat C = \begin{bmatrix}
      -8  & 6  \\
      -15 & 11 \\
    \end{bmatrix}
  \]

  \exsol[21.6cm]{%
  Az $\rmat A$ mátrix sajátértékeinek meghatározásához írjuk fel a
  karakterisztikus egyenletet:
  \[
    \det \left( \rmat A - \lambda \imat \right) = 0
    \text{.}
  \]
  Számítsuk ki ezen determináns értékét:
  \[
    \begin{vmatrix}
      2 - \lambda & 3          \\
      6           & 5 -\lambda
    \end{vmatrix}
    = (2 - \lambda)(5 - \lambda) - 3 \cdot 6
    = 10 - 7\lambda + \lambda^2 - 18
    = \lambda^2 - 7\lambda - 8
    = (\lambda - 8)(\lambda + 1)
    = 0
    \text{.}
  \]
  Vagyis a sajátértékek: $\lambda_1 = 8$ és $\lambda_2 = -1$.

  \vspace{.66em}
  A sajátvektorokat az alábbi egyenlet segítségével kereshetjük:
  \[
    (\rmat A - \lambda_i \imat) \rvec v_i = 0
    \text{.}
  \]

  A $\lambda_1 = 8$-hoz tartozó sajátvektor:
  \[
    \rmat A - \lambda_1 \imat = \begin{bmatrix}
      2 - 8 & 3     \\
      6     & 5 - 8
    \end{bmatrix} = \begin{bmatrix}
      -6 & 3  \\
      6  & -3
    \end{bmatrix}
    \quad \rightarrow \quad
    \left[\begin{matrix}
        -6 & 3  \\
        6  & -3
      \end{matrix}\right.\left|\begin{matrix}
        \,0 \\ \,0
      \end{matrix}\right]
    \quad \sim \quad
    \left[\begin{matrix}
        -6 & 3 \\
        0  & 0
      \end{matrix}\right.\left|\begin{matrix}
        \,0 \\ \,0
      \end{matrix}\right]
    \text{.}
  \]
  Ezek alapján a koordináták közötti viszony:
  \[
    -6v_{11} + 3v_{12} = 0
    \quad \rightarrow \quad
    v_{12} = 2 v_{11}
    \text{.}
  \]
  A sajátvektor paraméteresen, majd egységhosszúra normálva:
  \[
    \rvec v_1 = t_1 \begin{bmatrix}
      1 \\ 2
    \end{bmatrix}
    \quad
    \rightarrow
    \quad
    \uvec v_1 = \frac{1}{\sqrt{1^2 + 2^2}}\begin{bmatrix}
      1 \\ 2
    \end{bmatrix} = \begin{bmatrix}
      1 / \sqrt{5} \\
      2 / \sqrt{5}
    \end{bmatrix}
    \text{.}
  \]

  A $\lambda_2 = -1$-hoz tartozó sajátvektor:
  \[
    \rmat A - \lambda_2 \imat = \begin{bmatrix}
      2 + 1 & 3     \\
      6     & 5 + 1
    \end{bmatrix} = \begin{bmatrix}
      3 & 3 \\
      6 & 6
    \end{bmatrix}
    \quad \rightarrow \quad
    \left[\begin{matrix}
        3 & 3 \\
        6 & 6
      \end{matrix}\right.\left|\begin{matrix}
        \,0 \\ \,0
      \end{matrix}\right]
    \quad \sim \quad
    \left[\begin{matrix}
        1 & 1 \\
        0 & 0
      \end{matrix}\right.\left|\begin{matrix}
        \,0 \\ \,0
      \end{matrix}\right]
    \text{.}
  \]
  Ezek alapján a koordináták közötti viszony:
  \[
    v_{21} + v_{22} = 0
    \quad \rightarrow \quad
    v_{22} = -v_{21}
    \text{.}
  \]
  A sajátvektor paraméteresen, majd egységhosszúra normálva:
  \[
    \rvec v_2 = t_2 \begin{bmatrix}
      1 \\ -1
    \end{bmatrix}
    \quad
    \rightarrow
    \quad
    \uvec v_2 = \frac{1}{\sqrt{1^2 + 1^2}}\begin{bmatrix}
      1 \\ -1
    \end{bmatrix} = \begin{bmatrix}
      1 / \sqrt{2} \\
      -1 / \sqrt{2}
    \end{bmatrix}
    \text{.}
  \]

  \tcbline

  A $\rmat{B}$ mátrixról ránézésre megállapítható, hogy sajátértékei $\lambda_1
    = \lambda_2 = 1$. Keressük meg a sajátvektorait:
  \[
    \rmat B - \lambda_{12} \imat = \begin{bmatrix}
      1 - 1 & 0     \\
      0     & 1 - 1
    \end{bmatrix} = \begin{bmatrix}
      0 & 0 \\
      0 & 0
    \end{bmatrix}
    \quad \rightarrow \quad
    \left[\begin{matrix}
        0 & 0 \\
        0 & 0
      \end{matrix}\right.\left|\begin{matrix}
        \,0 \\ \,0
      \end{matrix}\right]
    \text.
  \]

  Látható, hogy ennek a lineáris egyenletrendszernek végtelen sok megoldása van.
  A sajátvektorok ennek tudatában paraméteresen:
  \[
    \rvec v = \begin{bmatrix}
      t_1 \\ t_2
    \end{bmatrix} = \begin{bmatrix}
      t_1 \\ 0
    \end{bmatrix} + \begin{bmatrix}
      0 \\ t_2
    \end{bmatrix} = t_1 \begin{bmatrix}
      1 \\ 0
    \end{bmatrix} + t_2 \begin{bmatrix}
      0 \\ 1
    \end{bmatrix}
    \text{.}
  \]
  Az sajátvektorok egységnyi hosszúra normálva:
  \[
    \uvec v_1 = \begin{bmatrix}
      1 \\ 0
    \end{bmatrix}
    \text{,}
    \quad \text{és} \quad
    \uvec v_2 = \begin{bmatrix}
      0 \\ 1
    \end{bmatrix}
    \text{.}
  \]

  \tcbline

  A $\rmat C$ mátrix sajátértékeinek meghatározásához írjuk fel a
  karakterisztikus egyenletet:
  \[
    \det \left( \rmat C - \lambda \imat \right) = 0
    \text{.}
  \]
  Számítsuk ki ezen determináns értékét:
  \[
    \begin{vmatrix}
      -8 - \lambda & 6           \\
      -15          & 11 -\lambda
    \end{vmatrix}
    = (-8 - \lambda)(11 - \lambda) - (-15) \cdot 6
    = \lambda^2 - 3\lambda + 2
    = (\lambda - 2)(\lambda - 1)
    = 0
    \text{.}
  \]
  Vagyis a sajátértékek: $\lambda_1 = 1$ és $\lambda_2 = 2$.

  \vspace{.66em}
  A sajátvektorokat az alábbi egyenlet segítségével kereshetjük:
  \[
    (\rmat C - \lambda_i \imat) \rvec v_i = 0
    \text{.}
  \]

  A $\lambda_1 = 1$-hez tartozó sajátvektor:
  \[
    \rmat C - \lambda_1 \imat = \begin{bmatrix}
      -8 - 1 & 6      \\
      -15    & 11 - 1
    \end{bmatrix} = \begin{bmatrix}
      -9  & 6  \\
      -15 & 10
    \end{bmatrix}
    \quad \rightarrow \quad
    \left[\begin{matrix}
        -9  & 6  \\
        -15 & 10
      \end{matrix}\right.\left|\begin{matrix}
        \,0 \\ \,0
      \end{matrix}\right]
    \quad \sim \quad
    \left[\begin{matrix}
        -9 & 6 \\
        0  & 0
      \end{matrix}\right.\left|\begin{matrix}
        \,0 \\ \,0
      \end{matrix}\right]
    \text{.}
  \]
  Ezek alapján a koordináták közötti viszony:
  \[
    -9v_{11} + 6v_{12} = 0
    \quad \rightarrow \quad
    2 v_{12} = 3 v_{11}
    \text{.}
  \]
  A sajátvektor paraméteresen, majd egységhosszúra normálva:
  \[
    \rvec v_1 = t_1 \begin{bmatrix}
      2 \\ 3
    \end{bmatrix}
    \quad
    \rightarrow
    \quad
    \uvec v_1 = \frac{1}{\sqrt{2^2 + 3^2}}\begin{bmatrix}
      2 \\ 3
    \end{bmatrix} = \begin{bmatrix}
      2 / \sqrt{13} \\
      3 / \sqrt{13}
    \end{bmatrix}
    \text{.}
  \]

  A $\lambda_2 = 2$-höz tartozó sajátvektor:
  \[
    \rmat C - \lambda_2 \imat = \begin{bmatrix}
      -8 - 2 & 6      \\
      -15    & 11 - 2
    \end{bmatrix} = \begin{bmatrix}
      -10 & 6 \\
      -15 & 9
    \end{bmatrix}
    \quad \rightarrow \quad
    \left[\begin{matrix}
        -10 & 6 \\
        -15 & 9
      \end{matrix}\right.\left|\begin{matrix}
        \,0 \\ \,0
      \end{matrix}\right]
    \quad \sim \quad
    \left[\begin{matrix}
        -10 & 6 \\
        0   & 0
      \end{matrix}\right.\left|\begin{matrix}
        \,0 \\ \,0
      \end{matrix}\right]
    \text{.}
  \]
  Ezek alapján a koordináták közötti viszony:
  \[
    -10v_{21} + 6v_{22} = 0
    \quad \rightarrow \quad
    3v_{22} = 5v_{21}
    \text{.}
  \]
  A sajátvektor paraméteresen, majd egységhosszúra normálva:
  \[
    \rvec v_2 = t_2 \begin{bmatrix}
      3 \\ 5
    \end{bmatrix}
    \quad
    \rightarrow
    \quad
    \uvec v_2 = \frac{1}{\sqrt{3^2 + 5^2}}\begin{bmatrix}
      3 \\ 5
    \end{bmatrix} = \begin{bmatrix}
      3 / \sqrt{34} \\
      5 / \sqrt{34}
    \end{bmatrix}
    \text{.}
  \]
  }
\end{exercise}

  \begin{exercise}{Határozzuk meg $\rmat A^{12}$ és $\rmat B^{25}$ mátrixokat!}
  \[
    \rmat A = \begin{bmatrix}
      -8  & 6  \\
      -15 & 11
    \end{bmatrix}
    \hspace{2cm}
    \rmat B = \begin{bmatrix}
      \sqrt{3} / 2 & -1/2         \\
      1/2          & \sqrt{3} / 2
    \end{bmatrix}
  \]

  \exsol[21.85cm]{%
  A hatványozást elvégezni nagyon hosszadalmas lenne nem diagonális
  mátrixon. Transzformáljuk át a mátrixot a saját koordináta-rendszerébe,
  végezzük el ott a hatványozást, majd utána transzformáljuk vissza a
  jelenlegi ko\-or\-di\-ná\-ta-rendszerünkbe. A bázis-transzformáció
  elvégzéséhez szükségünk van a hatványozandó mátrix sajátértékeire és
  sajátvektoraira. Ezeket már korábban meghatároztuk:
  \[
    \lambda_1 = 1
    \text,
    \quad
    \rvec v_1 = \begin{bmatrix}
      2 \\ 3
    \end{bmatrix}
    \text,
    \qquad
    \lambda_2 = 2
    \text,
    \quad
    \rvec v_2 = \begin{bmatrix}
      3 \\ 5
    \end{bmatrix}
    \text.
  \]
  Az eredeti és a diagonális mátrix közötti kapcsolat:
  \[
    \rmat A = \rmat T \mbfLambda \rmat T^{-1}
    \text.
  \]
  A $\rmat T$ transzformációs mátrixot úgy kapjuk meg, hogy a sajátvektorokat,
  mint oszlopvektorokat a mátrix egyes soraiba helyezzük el:
  \[
    \rmat T = \begin{bmatrix}
      2 & 3 \\
      3 & 5
    \end{bmatrix}
    \text.
  \]
  Szükségünk van továbbá ennek a mátrixnak az inverzére is. Ezt korábban már
  meghatároztuk:
  \[
    \rmat T^{-1} = \begin{bmatrix}
      5  & -3 \\
      -3 & 2
    \end{bmatrix}
    \text.
  \]
  A diagonális mátrix elemei maguk a sajátértékek lesznek, mégpedig olyan
  sorrendben, ahogyan a hozzájuk tartozó sajátvektorokat beírtuk a
  transzformációs mátrixba. Ezt számítással ellenőrizhetjük:
  \[
    \mbfLambda
    = \rmat T^{-1} \rmat A \rmat T
    = \begin{bmatrix}
      5  & -3 \\
      -3 & 2
    \end{bmatrix} \begin{bmatrix}
      -8  & 6  \\
      -15 & 11
    \end{bmatrix} \begin{bmatrix}
      2 & 3 \\
      3 & 5
    \end{bmatrix}
    =
    \begin{bmatrix}
      5  & -3 \\
      -6 & 4
    \end{bmatrix} \begin{bmatrix}
      2 & 3 \\
      3 & 5
    \end{bmatrix}
    =
    \begin{bmatrix}
      1 & 0 \\
      0 & 2
    \end{bmatrix}
    \text.
  \]
  A hatványozást itt már könnyen el tudjuk végezni:
  \[
    \mbfLambda^{12}
    = \begin{bmatrix} 1 & 0 \\ 0 & 2 \end{bmatrix}^{12}
    = \begin{bmatrix} 1^{12} & 0 \\ 0 & 2^{12} \end{bmatrix}
    = \begin{bmatrix} 1 & 0 \\ 0 & 4096  \end{bmatrix}
    \text.
  \]
  Transzformáljuk vissza az eredményül kapott mátrixunkat:
  \begin{align*}
    \rmat A^{12} = \rmat T \mbfLambda^{12} \rmat T^{-1}
     & =
    \begin{bmatrix} 2 & 3 \\ 3 & 5 \end{bmatrix}
    \begin{bmatrix} 1 & 0 \\ 0 & 4096 \end{bmatrix}
    \begin{bmatrix} 5  & -3 \\ -3 & 2 \end{bmatrix}
    \\
     & =
    \begin{bmatrix} 2 & 12288 \\ 3 & 20480 \end{bmatrix}
    \begin{bmatrix} 5  & -3 \\ -3 & 2 \end{bmatrix}
    \\
     & =
    \begin{bmatrix}
      -36854 & 24570 \\
      -61425 & 40951
    \end{bmatrix}
    \text.
  \end{align*}

  \tcbline

  Ahelyett, hogy elkezdenénk vadul számolni, álljunk meg egy percre, és
  vizsgáljuk meg az $\rmat B$ mátrixot. Megállapíthatjuk róla, hogy ez egy
  forgatási mátrix, mégpedig a $z$ tengely körül $+30^\circ$-ot forgat:
  \[
    \rmat B = \begin{bmatrix}
      \cos 30^\circ & - \sin 30^\circ \\
      \sin 30^\circ & \cos 30^\circ
    \end{bmatrix}
    \text.
  \]
  A huszonötödik hatványra való emelés azt jelenti, hogy huszonötször forgatunk
  $30^\circ$-kal egymás után, amely 2 teljes körbefordulást ($2 \times 12 \times
    30^\circ = 2 \times 360^\circ$), majd utána még egy $30^\circ$-os forgatást
  jelent, vagyis ezen mátrix huszonötödik hatványa önmaga:
  \[
    \rmat B^{25} = \rmat B
    \text.
  \]
  }
\end{exercise}

\end{frame}

\begin{frame}
  \frametitle{Lineáris leképezések}
  \framesubtitle{Másodfokú görbék I}

  \begin{block}{Kvadratikus forma}
    Csupa másodfokú tagot tartalmazó polinomok átírhatóak mátrixos alakra:
    \[
      a x^2 + 2 b x y + c y^2
      =
      \begin{bmatrix} x & y \end{bmatrix}
      \begin{bmatrix} a & b \\ b & c \end{bmatrix}
      \begin{bmatrix} x \\ y \end{bmatrix}
      =
      \rvec x^{\mathsf T} \rmat A \rvec x
      \text.
    \]
  \end{block}

  \begin{block}{Mátrix definitsége}
    \centering
    \begin{tabular}{l>{$\sim}{c}<{$}>{$\forall \lambda_i}{c}<{$}}
      pozitív definit      &  & > 0    \\
      pozitív szemidefinit &  & \geq 0 \\
      negatív definit      &  & < 0    \\
      negatív szemidefinit &  & \leq 0 \\
    \end{tabular}
    \\[2mm]
    indefinit, ha az előzőek közül egyik sem
  \end{block}
\end{frame}

\begin{frame}
  \frametitle{Lineáris leképezések}
  \framesubtitle{Másodfokú görbék II}

  \begin{block}{Másodfokú görbe a mátrix definitsége alapján}
    \centering
    \begin{tabular}{l>{$\sim}{c}<{$}c}
      definit      &  & ellipszis \\
      szemidefinit &  & parabola  \\
      indefinit    &  & hiperbola \\
    \end{tabular}
  \end{block}

  \begin{exercise}{%
    Milyen alakzatot írnak le az alábbi másodfokú görbék?
  }
  \begin{enumerate}[a)]
    \item $5x^2 + 3y^2 + 12xy = -169$
    \item $5x^2 + 3y^2 + 4xy = 49$
  \end{enumerate}

  \exsol{%
    \begin{enumerate}[a)]
      \item $5x^2 + 3y^2 + 12xy = -169$

            Hozzuk az egyenlet bal oldalát mátrixos alakra:
            \[
              5x^2 + 3y^2 + 12xy =
              \begin{bmatrix} x & y \end{bmatrix}
              \begin{bmatrix} 5 & 6 \\ 6 & 3 \end{bmatrix}
              \begin{bmatrix} x \\ y \end{bmatrix}
              \text.
            \]
            Határozzuk meg a mátrix sajátértékeit:
            \[
              \begin{vmatrix} 5 - \lambda & 6 \\ 6 & 3 - \lambda \end{vmatrix}
              = (5 - \lambda)(3 - \lambda) - 36
              = \lambda^2 - 8 \lambda + 15 - 36
              = \lambda^2 - 8 \lambda -21
              = 0
              \text,
            \]
            \[
              \lambda_{12}
              = \frac{8 \pm \sqrt{8^2 + 4 \cdot 1 \cdot 21}}{2}
              = 4 \pm \sqrt{37}
              = \begin{cases}
                - 2,0828 \text, \\
                10.083 \text.
              \end{cases}
            \]
            Látható, hogy az egyik sajátérték pozitív, a másik pedig negatív,
            vagyis indefinit. Az alakzat ebből következőleg hiperbola.


            \tcbline
      \item $5x^2 + 3y^2 + 4xy = 49$

            Hozzuk az egyenlet bal oldalát mátrixos alakra:
            \[
              5x^2 + 3y^2 + 4xy =
              \begin{bmatrix} x & y \end{bmatrix}
              \begin{bmatrix} 5 & 2 \\ 2 & 3 \end{bmatrix}
              \begin{bmatrix} x \\ y \end{bmatrix}
              \text.
            \]
            Határozzuk meg a mátrix sajátértékeit:
            \[
              \begin{vmatrix} 5 - \lambda & 2 \\ 2 & 3 - \lambda \end{vmatrix}
              = (5 - \lambda)(3 - \lambda) - 4
              = \lambda^2 - 8 \lambda + 15 - 4
              = \lambda^2 - 8 \lambda + 11
              = 0
              \text,
            \]
            \[
              \lambda_{12}
              = \frac{8 \pm \sqrt{8^2 - 4 \cdot 1 \cdot 11}}{2}
              = 4 \pm \sqrt{5}
              = \begin{cases}
                1,7639 \text, \\
                6,2361 \text.
              \end{cases}
            \]
            Látható, hogy minkét sajátérték pozitív, vagyis a mátrix pozitív
            definit. Az alakzat ebből következőleg ellipszis.
    \end{enumerate}
  }
\end{exercise}

\end{frame}
