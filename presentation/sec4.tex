\section{Lineáris leképezések}
\begin{frame}
  \frametitle{Lineáris leképezések}
  \framesubtitle{Alapfogalmak I}

  \begin{block}{Lineáris leképezés}
    Legyenek $V_1$ és $V_2$ ugyanazon $T$ test feletti vektorterek. Legyen
    $\varphi: V_1 \rightarrow V_2$ leképezés, melyet lineáris leképezésnek
    nevezünk, ha tetszőleges két $V_1$-beli vektor ($\forall \rvec a; \rvec b \in
      V_1$) és $T$-beli skalár ($\alpha \in T$) esetén teljesülnek az alábbiak:
    \begin{itemize}
      \item $\varphi(\rvec a + \rvec b) = \varphi(\rvec a) + \varphi(\rvec b)$
            $\quad \sim \quad$ összegre tagonként hat,
      \item $\varphi(\alpha \rvec a) = \alpha \varphi(\rvec a)$
            $\hspace{18mm} \sim \quad$ skalár kiemelhető.
    \end{itemize}
  \end{block}
\end{frame}

\begin{frame}
  \frametitle{Lineáris leképezések}
  \framesubtitle{Alapfogalmak II}

  \vfill
  \begin{block}{Magtér}
    Legyen $\varphi: V_1 \rightarrow V_2$ lineáris leképezés. A leképezés
    magtere:
    \[
      \ker \varphi = \Big\{\;
      \rvec v \;\Big|\; \rvec v \in V_1 \;\land\;
      \varphi(\rvec v) = \nvec
      \;\Big\}
      \text.
    \]
  \end{block}
  \vfill
  \begin{block}{Defektus}
    A magtér dimenzióját a leképezés defektusának nevezzük:
    \[
      \dim \ker \varphi = \defect \varphi
      \text.
    \]
  \end{block}
  \vfill
\end{frame}

\begin{frame}
  \frametitle{Lineáris leképezések}
  \framesubtitle{Alapfogalmak III}

  \vfill
  \begin{block}{Képtér}
    Egy $\varphi: V_1 \rightarrow V_2$ lineáris leképezés rangjának nevezzük
    a képtér dimenzióját:
    \[
      \rg \varphi = \dim V_2
      \text.
    \]
  \end{block}
  \vfill
  \begin{block}{Rang nullitás tétele}
    Legyen $V_1$ véges dimenziós vektortér, $\varphi: V_1 \rightarrow V_2$
    lineáris leképezés, ekkor:
    \[
      \rg \varphi + \defect \varphi = \dim V_1
    \]
  \end{block}
  \vfill
\end{frame}

\begin{frame}
  \frametitle{Lineáris leképezések}
  \framesubtitle{Feladatok I}

  \input{exercise/linear-map-tf}
\end{frame}

\begin{frame}
  \frametitle{Lineáris leképezések}
  \framesubtitle{Feladatok II}

  \input{exercise/linear-map}
\end{frame}

% Rotations etc
\begin{frame}
  \frametitle{Lineáris leképezések}
  \framesubtitle{Alap geometriai leképezések I}

  \begin{block}{Tükrözés valamelyik tengelyre}
    \[
      \rmat A_x = \overset{x\text{-tengelyre}}{\begin{bmatrix}
          1 & 0  & 0  \\
          0 & -1 & 0  \\
          0 & 0  & -1
        \end{bmatrix}}
      \text,
      \quad
      \rmat A_y = \overset{y\text{-tengelyre}}{\begin{bmatrix}
          -1 & 0 & 0  \\
          0  & 1 & 0  \\
          0  & 0 & -1
        \end{bmatrix}}
      \text,
      \quad
      \rmat A_z = \overset{z\text{-tengelyre}}{\begin{bmatrix}
          -1 & 0  & 0 \\
          0  & -1 & 0 \\
          0  & 0  & 1
        \end{bmatrix}}
      \text.
    \]
  \end{block}

  \begin{block}{Tükrözés valamelyik síkra}
    \[
      \rmat A_{xy} = \overset{xy\text{-síkra}}{\begin{bmatrix}
          1 & 0 & 0  \\
          0 & 1 & 0  \\
          0 & 0 & -1
        \end{bmatrix}}
      \text,
      \quad
      \rmat A_{xz} = \overset{xz\text{-síkra}}{\begin{bmatrix}
          1 & 0  & 0 \\
          0 & -1 & 0 \\
          0 & 0  & 1
        \end{bmatrix}}
      \text,
      \quad
      \rmat A_{yz} = \overset{yz\text{-síkra}}{\begin{bmatrix}
          1 & 0 & 0  \\
          0 & 1 & 0  \\
          0 & 0 & -1
        \end{bmatrix}}
      \text.
    \]
  \end{block}
\end{frame}

\begin{frame}
  \frametitle{Lineáris leképezések}
  \framesubtitle{Alap geometriai leképezések II}

  \begin{block}{Vetítés valamelyik tengelyre}
    \[
      \rmat A_x = \overset{x\text{-tengelyre}}{\begin{bmatrix}
          1 & 0 & 0 \\
          0 & 0 & 0 \\
          0 & 0 & 0
        \end{bmatrix}}
      \text,
      \quad
      \rmat A_y = \overset{y\text{-tengelyre}}{\begin{bmatrix}
          0 & 0 & 0 \\
          0 & 1 & 0 \\
          0 & 0 & 0
        \end{bmatrix}}
      \text,
      \quad
      \rmat A_z = \overset{z\text{-tengelyre}}{\begin{bmatrix}
          0 & 0 & 0 \\
          0 & 0 & 0 \\
          0 & 0 & 1
        \end{bmatrix}}
      \text.
    \]
  \end{block}

  \begin{block}{Vetítés valamelyik síkra}
    \[
      \rmat A_{xy} = \overset{xy\text{-síkra}}{\begin{bmatrix}
          1 & 0 & 0 \\
          0 & 1 & 0 \\
          0 & 0 & 0
        \end{bmatrix}}
      \text,
      \quad
      \rmat A_{xz} = \overset{xz\text{-síkra}}{\begin{bmatrix}
          1 & 0 & 0 \\
          0 & 0 & 0 \\
          0 & 0 & 1
        \end{bmatrix}}
      \text,
      \quad
      \rmat A_{yz} = \overset{yz\text{-síkra}}{\begin{bmatrix}
          1 & 0 & 0 \\
          0 & 1 & 0 \\
          0 & 0 & 0
        \end{bmatrix}}
      \text.
    \]
  \end{block}
\end{frame}

\begin{frame}
  \frametitle{Lineáris leképezések}
  \framesubtitle{Alap geometriai leképezések III}

  \begin{block}{Forgatások -- ortogonális transzformációk}
    \[
      \begin{array}{X{67mm} c p{26mm}}
        \rmat A_x(\varphi) =
        \left[\begin{array}{*{3}{X{12mm}}}
                  1 & 0            & 0             \\
                  0 & \cos \varphi & -\sin \varphi \\
                  0 & \sin \varphi & \cos \varphi
                \end{array}\right]
         & -
         & $x$\text{-tengely körül,}
        \\
        \rmat A_y(\varphi) =
        \left[\begin{array}{*{3}{X{12mm}}}
                  \cos \varphi  & 0 & \sin \varphi \\
                  0             & 1 & 0            \\
                  -\sin \varphi & 0 & \cos \varphi
                \end{array}\right]
         & -
         & $y$\text{-tengely körül,}
        \\
        \rmat A_z(\varphi) =
        \left[\begin{array}{*{3}{X{12mm}}}
                  \cos \varphi & -\sin \varphi & 0 \\
                  \sin \varphi & \cos \varphi  & 0 \\
                  0            & 0             & 1
                \end{array}\right]
         & -
         & $z$\text{-tengely körül.}
      \end{array}
    \]
  \end{block}
\end{frame}

\begin{frame}
  \frametitle{Lineáris leképezések}
  \framesubtitle{Geometriai leképezések feladat}

  \input{exercise/rotation}
\end{frame}

% Eigenvalues and eigenvectors
\begin{frame}
  \frametitle{Lineáris leképezések}
  \framesubtitle{Sajátértékek és sajátvektorok I}

  \begin{block}{Definíció}
    Legyen $V$ a $T$ test feletti vektortér, $\rvec v \in V$, $\rvec v \neq
      \nvec$. $\rvec v$-t a $\varphi: V \rightarrow V$ lineáris leképezés
    sajátvektorának mondjuk, ha önmaga skalárszorosába megy át a leképezés
    során, azaz $\varphi(\rvec v) = \lambda \rvec v$,  $\lambda \in T$.
    $\lambda$-t a $\rvec v$ sajátvektorhoz tartozó sajátértéknek mondjuk.
  \end{block}
\end{frame}

\begin{frame}
  \frametitle{Lineáris leképezések}
  \framesubtitle{Sajátértékek és sajátvektorok II}

  \begin{block}{Sajátértékek számítása}
    Az $\rmat A$ mátrix sajátértékei a karakterisztikus egyenlet megoldásával
    határozhatóak meg:
    \[
      \det (\rmat A - \lambda \imat) = 0
      \text.
    \]
  \end{block}

  \begin{block}{Sajátvektorok meghatározása}
    Az egyes sajátértékekhez tartozó sajátvektorok az alábbi egyenlet alapján
    határozhatóak meg:
    \[
      (\rmat A - \lambda_i \imat) \rvec v_i = \nvec
      \text.
    \]
  \end{block}
\end{frame}

\begin{frame}
  \frametitle{Lineáris leképezések}
  \framesubtitle{Sajátértékek és sajátvektorok II}

  \input{exercise/eigen}
  \input{exercise/exponentiation}
\end{frame}

\begin{frame}
  \frametitle{Lineáris leképezések}
  \framesubtitle{Másodfokú görbék I}

  \begin{block}{Kvadratikus forma}
    Csupa másodfokú tagot tartalmazó polinomok átírhatóak mátrixos alakra:
    \[
      a x^2 + 2 b x y + c y^2
      =
      \begin{bmatrix} x & y \end{bmatrix}
      \begin{bmatrix} a & b \\ b & c \end{bmatrix}
      \begin{bmatrix} x \\ y \end{bmatrix}
      =
      \rvec x^{\mathsf T} \rmat A \rvec x
      \text.
    \]
  \end{block}

  \begin{block}{Mátrix definitsége}
    \centering
    \begin{tabular}{l>{$\sim}{c}<{$}>{$\forall \lambda_i}{c}<{$}}
      pozitív definit      &  & > 0    \\
      pozitív szemidefinit &  & \geq 0 \\
      negatív definit      &  & < 0    \\
      negatív szemidefinit &  & \leq 0 \\
    \end{tabular}
    \\[2mm]
    indefinit, ha az előzőek közül egyik sem
  \end{block}
\end{frame}

\begin{frame}
  \frametitle{Lineáris leképezések}
  \framesubtitle{Másodfokú görbék II}

  \begin{block}{Másodfokú görbe a mátrix definitsége alapján}
    \begin{tabular}{l>{$\sim}{c}<{$}c}
      definit      &  & ellipszis \\
      szemidefinit &  & parabola  \\
      indefinit    &  & hiperbola \\
    \end{tabular}
  \end{block}

  \begin{exercise}{%
    Milyen alakzatot írnak le az alábbi másodfokú görbék?
  }
  \begin{enumerate}[a)]
    \item $5x^2 + 3y^2 + 12xy = -169$
    \item $5x^2 + 3y^2 + 4xy = 49$
  \end{enumerate}

  \exsol{%
    \begin{enumerate}[a)]
      \item $5x^2 + 3y^2 + 12xy = -169$

            Hozzuk az egyenlet bal oldalát mátrixos alakra:
            \[
              5x^2 + 3y^2 + 12xy =
              \begin{bmatrix} x & y \end{bmatrix}
              \begin{bmatrix} 5 & 6 \\ 6 & 3 \end{bmatrix}
              \begin{bmatrix} x \\ y \end{bmatrix}
              \text.
            \]
            Határozzuk meg a mátrix sajátértékeit:
            \[
              \begin{vmatrix} 5 - \lambda & 6 \\ 6 & 3 - \lambda \end{vmatrix}
              = (5 - \lambda)(3 - \lambda) - 36
              = \lambda^2 - 8 \lambda + 15 - 36
              = \lambda^2 - 8 \lambda -21
              = 0
              \text,
            \]
            \[
              \lambda_{12}
              = \frac{8 \pm \sqrt{8^2 + 4 \cdot 1 \cdot 21}}{2}
              = 4 \pm \sqrt{37}
              = \begin{cases}
                - 2,0828 \text, \\
                10.083 \text.
              \end{cases}
            \]
            Látható, hogy az egyik sajátérték pozitív, a másik pedig negatív,
            vagyis indefinit. Az alakzat ebből következőleg hiperbola.


            \tcbline
      \item $5x^2 + 3y^2 + 4xy = 49$

            Hozzuk az egyenlet bal oldalát mátrixos alakra:
            \[
              5x^2 + 3y^2 + 4xy =
              \begin{bmatrix} x & y \end{bmatrix}
              \begin{bmatrix} 5 & 2 \\ 2 & 3 \end{bmatrix}
              \begin{bmatrix} x \\ y \end{bmatrix}
              \text.
            \]
            Határozzuk meg a mátrix sajátértékeit:
            \[
              \begin{vmatrix} 5 - \lambda & 2 \\ 2 & 3 - \lambda \end{vmatrix}
              = (5 - \lambda)(3 - \lambda) - 4
              = \lambda^2 - 8 \lambda + 15 - 4
              = \lambda^2 - 8 \lambda + 11
              = 0
              \text,
            \]
            \[
              \lambda_{12}
              = \frac{8 \pm \sqrt{8^2 - 4 \cdot 1 \cdot 11}}{2}
              = 4 \pm \sqrt{5}
              = \begin{cases}
                1,7639 \text, \\
                6,2361 \text.
              \end{cases}
            \]
            Látható, hogy minkét sajátérték pozitív, vagyis a mátrix pozitív
            definit. Az alakzat ebből következőleg ellipszis.
    \end{enumerate}
  }
\end{exercise}

\end{frame}
