\section{Mátrixok}
\begin{frame}
  \frametitle{Mátrixok}
  \framesubtitle{Alapfogalmak I}

  \vfill

  \begin{block}{Mátrix}
    Az $m$ sorba és $n$ oszlopba rendezett rendezett számokat mátrixoknak
    nevezzük.
  \end{block}

  \vfill

  \begin{block}{Speciális elnevezések}
    \begin{itemize}
      \item $\rmat M \in \mathbb R^{n \times 1}$ -- oszlopmátrix
      \item $\rmat M \in \mathbb R^{1 \times k}$ -- sormátrix
      \item $\rmat M \in \mathbb R^{n \times n}$ -- kvadratikus mátrix
            \[
              \imat = \overset{\text{egységmátrix}}{\begin{bmatrix}
                  1      & \dots  & 0      \\
                  \vdots & \ddots & \vdots \\
                  0      & \dots  & 1
                \end{bmatrix}}
              \hspace{2cm}
              \nmat = \overset{\text{nullmátrix}}{\begin{bmatrix}
                  0      & \dots  & 0      \\
                  \vdots & \ddots & \vdots \\
                  0      & \dots  & 0
                \end{bmatrix}}
            \]
    \end{itemize}
  \end{block}

  \vfill
\end{frame}

\begin{frame}
  \frametitle{Mátrixok}
  \framesubtitle{Alapfogalmak I}

  \vfill

  \begin{block}{Transzponált}
    Egy mátrix transzponáltja a főátlóra való tükörképe. Jele: $A^\mathsf T$.
  \end{block}

  \vfill

  \begin{block}{Szimmetrikus mátrix}
    Ha $A = A^\mathsf T$, akkor a mátrix szimmetrikus.
  \end{block}

  \vfill

  \begin{block}{Antiszimmetrikus mátrix}
    Ha $A = -A^\mathsf T$, akkor a mátrix antiszimmetrikus.
  \end{block}

  \vfill
\end{frame}

\begin{frame}
  \frametitle{Mátrixok}
  \framesubtitle{Elemi mátrixműveletek}

  \vfill

  \begin{block}{Összeadás}
    Ha $\rmat A; \rmat B \in \mathbb R^{n \times k}$, akkor az összegükön a
    megfelelő elempárok összeadásával keletkező mátrixot értjük.
  \end{block}

  \vfill

  \begin{block}{Skalárral való szorzás}
    Egy mátrix és egy skalár szorzata olyan mátrix, melynek minden eleme
    skalárszorosa az eredeti mátrix elemeinek.
  \end{block}

  \vfill
\end{frame}

\begin{frame}
  \frametitle{Mátrixok}
  \framesubtitle{Elemi mátrixműveletek}

  \begin{block}{Mátrix szorzás
      -- asszociatív, disztributív, de nem kommutatív!}
    \[
      \left.\begin{array}{ll}
        \rmat A \in \mathbb R^{m \times n} \\
        \rmat B \in \mathbb R^{n \times p}
      \end{array}\right\}
      \; \rightarrow \;
      \rmat A \cdot \rmat B \in \mathbb R^{m \times p}
    \]

    \def\arraystretch{1.1}
    \begin{align*}
       & \left[\begin{array}{X{2cm}cX{2cm}}
                   b_{11} & \dots  & b_{1p} \\
                   b_{21} & \dots  & b_{2p} \\
                   \vdots & \ddots & \vdots \\
                   b_{n1} & \dots  & b_{np}
                 \end{array}\right]
      \\
      \left[\begin{array}{cccc}
                a_{11} & a_{12} & \dots  & a_{1n} \\
                a_{21} & a_{22} & \dots  & a_{2n} \\
                \vdots & \vdots & \ddots & \vdots \\
                a_{m1} & a_{m2} & \dots  & a_{mn}
              \end{array}\right]
       & \left[\begin{array}{X{2cm}cX{2cm}}
                   \sum a_{1i} b_{i1} & \dots  & \sum a_{1i} b_{ip} \\
                   \sum a_{2i} b_{i1} & \dots  & \sum a_{2i} b_{ip} \\
                   \vdots             & \ddots & \vdots             \\
                   \sum a_{mi} b_{i1} & \dots  & \sum a_{mi} b_{ip}
                 \end{array}\right]
    \end{align*}
  \end{block}
\end{frame}

\begin{frame}
  \frametitle{Mátrixok}
  \framesubtitle{Determináns}

  \begin{block}{A determináns axonometrikus felépítése}
    Legyen $\rmat A \in \mathbb R^{n \times n}$ kvadratikus mátrix, és
    $\det: \mathbb R^{n \times n} \rightarrow \mathbb R$ függvény.
    Az $\rmat A$ mátrix determinánsának nevezzük $\det \rmat A$-t,
    a hozzárendelést pedig az alábbi négy axióma írja le:
    \newcommand\noskp{\vspace{-3mm}}
    \begin{enumerate}
      \item homogén:\noskp
            \[
              \edet{\lambda \rvec a_i} = \lambda \edet{\rvec a_i}
            \]
      \item \noskp additív:\noskp
            \[
              \edet{\rvec a_i + \rvec b_i} =
              \edet{\rvec a_i} + \edet{\rvec b_i}
            \]
      \item \noskp alternáló:\noskp
            \[
              \edet{\rvec a_i & \dots & \rvec a_j} =
              - \edet{\rvec a_j & \dots & \rvec a_i}
            \]
      \item \noskp $\imat$ determinánsa:\noskp
            \[
              \det \begin{pmatrix}
                \uvec e_1 & \uvec e_2 & \dots & \uvec e_n
              \end{pmatrix} = 1
            \]
    \end{enumerate}
  \end{block}
\end{frame}

\begin{frame}
  \frametitle{Mátrixok}
  \framesubtitle{Determináns tulajdonságai}

  \begin{block}{Tételek}
    \begin{itemize}
      \item Egy mátrix determinánsa 0, ha \dots
            \begin{itemize}
              \item az egyik oszlopvektora nullvektor,
              \item van két azonos oszlopvektora,
              \item az oszlopvektorai lineárisan összefüggőek.
            \end{itemize}
      \item Nem változik a determináns, ha az egyik oszlophoz hozzáadjuk egy
            másik oszlop skalárszorosát.
      \item Mátrix és transzponáltjának determinánsa megegyezik.
      \item Determinánsok szorzástétele:
            $\det (\rmat A \cdot \rmat B) = \det \rmat A \cdot \det \rmat B$.
      \item Ha $\rmat A \in \mathbb R^{n \times n}$, akkor
            $\det (\lambda \rmat A) = \lambda^n \det \rmat A$.
      \item \(
            \rvec a \cdot (\rvec b \times \rvec c) = \det \begin{pmatrix}
              \rvec a & \rvec b & \rvec c
            \end{pmatrix}
            \)
    \end{itemize}
  \end{block}
\end{frame}

\begin{frame}
  \frametitle{Mátrixok}
  \framesubtitle{Determinánsos tételek alkalmazása}

  \begin{exercise}{Válaszoljuk meg az alábbi rövid kérdéseket!}
  \begin{enumerate}[a)]
    \item Az $\rmat A \in \mathbb R^{3 \times 3}$ mátrix determinánsa 2.
          Legyen $\rmat A' = 3 \rmat A$.
          Mennyi $\det(\rmat A')$ értéke?
    \item A $\rmat B \in \mathbb R^{3 \times 3}$ mátrix determinánsa 4.
          Az első két oszlopát felcseréltük, a harmadikat pedig
          elosztottuk kettővel. Mennyi lesz az így keletkező $\rmat B'$ mátrix
          determinánsa?
    \item Határozzuk meg $\det(\rmat A' \cdot \rmat B')$ értéket!
  \end{enumerate}

  \exsol[13.1cm]{%
    \begin{enumerate}[a)]
      \item A $\det (\lambda \rmat A) = \lambda^n \det \rmat A$ tétel alapján:
            \[
              \det (3 \rmat A') = 3^3 \det \rmat A = 27 \cdot 2 = 54
              \text.
            \]

            \tcbline
      \item Az oszlopok felcserélése miatt a determináns a $(-1)$-szeresére
            változott, majd a harmadik oszlop 2-vel való osztása miatt a felére
            csökkent, vagyis:
            \[
              \det \rmat B' = \det \rmat B \cdot (-1) \cdot (1/2) = -2
              \text.
            \]

            \tcbline
      \item A determinánsok szorzástétele alapján:
            \[
              \det (\rmat A' \cdot \rmat B')
              = \det \rmat A' \cdot \det \rmat B'
              = 54 \cdot (-2)
              = -108
              \text.
            \]
    \end{enumerate}
  }
\end{exercise}

\end{frame}

\begin{frame}
  \frametitle{Mátrixok}
  \framesubtitle{Determináns számítása}

  \begin{block}{Kifejtési tétel -- előjelszabály!}
    \begin{align*}
      \begin{vmatrix}
        + & - & + \\
        - & + & - \\
        + & - & + \\
      \end{vmatrix}
      \; \rightarrow \;
      \begin{vmatrix}
        a & b & c \\
        d & e & f \\
        g & h & i \\
      \end{vmatrix}
       & = a \begin{vmatrix}
               e & f \\ h & i
             \end{vmatrix}
      - b \begin{vmatrix}
            d & f \\ g & i
          \end{vmatrix}
      + c \begin{vmatrix}
            d & e \\ g & h
          \end{vmatrix}
      \\
       & = a (ei - hf) - b(di - gf) + c(dh - eg)
    \end{align*}
  \end{block}

  \begin{block}{Sarrus-szabály -- csak ($3 \times 3$)-as mátrixoknál!}
    \centering
    \begin{tikzpicture}[ampersand replacement=\&]
      \matrix[
        matrix of math nodes,
        column sep=2mm,
      ] (sarrus) {
        a\vphantom{b} \& b \& c\vphantom{b} \& a \& b \\
        d \& e \& f \& d \& e                         \\
        g \& h\vphantom{g} \& i\vphantom{g} \& g \& h \\
      };

      \draw[red!40!gray, ultra thick, opacity=.5]
      (sarrus-1-1.center) -- (sarrus-3-3.center)
      (sarrus-1-2.center) -- (sarrus-3-4.center)
      (sarrus-1-3.center) -- (sarrus-3-5.center)
      ;

      \draw[blue!40!gray, ultra thick, opacity=.5]
      (sarrus-3-1.center) -- (sarrus-1-3.center)
      (sarrus-3-2.center) -- (sarrus-1-4.center)
      (sarrus-3-3.center) -- (sarrus-1-5.center)
      ;

      \draw[black, thick]
      (sarrus-1-1.north west) -- (sarrus-3-1.south west)
      (sarrus-1-3.north east) -- (sarrus-3-3.south east)
      ;

      \foreach \i in {1,2,3}{
          \node[above=-2.5mm, red!40!gray] at (sarrus-1-\i.north) {$+$};
          \node[below=-1.5mm, blue!40!gray] at (sarrus-3-\i.south) {$-$};
        }

      \node[] at (5,.25) {$\det \rmat A = + aei + bfg + cdh$};
      \node[] at (5,-.25) {$\phantom{\det \rmat A =} - gec - hfa - idb$};
    \end{tikzpicture}
  \end{block}
\end{frame}

\begin{frame}
  \frametitle{Mátrixok}
  \framesubtitle{Rang}

  \begin{block}{Mátrix rangja}
    A mátrix rangjának nevezzük az oszlopvektorai közül a lineárisan függetlenek
    maximális számát. A mátrix rangja elemi átalakítások során nem változik.
    \begin{itemize}
      \item tetszőleges sorát vagy oszlopát egy 0-tól különböző számmal
            megszorozzuk,
      \item tetszőleges sorát vagy oszlopát felcseréljük,
      \item tetszőleges sorához vagy oszlopához egy másik tetszőleges sorát
            vagy oszlopát adjuk.
    \end{itemize}
  \end{block}
\end{frame}

\begin{frame}
  \frametitle{Mátrixok}
  \framesubtitle{Mátrixos feladatok}

  \vfill
  \begin{exercise}{Határozzuk meg az alábbi mátrixok determinánsát!}
  \[
    \rmat A = \begin{bmatrix}
      1 & 4 & 8 \\
      0 & 1 & 0 \\
      2 & 4 & 6
    \end{bmatrix}
    \hspace{2cm}
    \rmat B = \begin{bmatrix}
      a  & 3  & x  \\
      -a & -2 & x  \\
      0  & 1  & 2x \\
    \end{bmatrix}
  \]

  \exsol{
    Határozzuk meg az $\rmat A$ mátrix determinánsát úgy, a kifejtési tétel
    segítségével. Mivel a második sor csak 1 nemzérus elemet tartalmaz, ezért
    innen fogunk kiindulni:
    \[
      \det \rmat A
      = \begin{vmatrix}
        1 & 4 & 8 \\
        0 & 1 & 0 \\
        2 & 4 & 6
      \end{vmatrix}
      = 1 \cdot \begin{vmatrix}
        1 & 8 \\
        2 & 6
      \end{vmatrix}
      = 1 \cdot (1 \cdot 6 - 2 \cdot 8)
      = -10
      \text.
    \]

    \tcbline

    A $\rmat B$ mátrix esetében vegyük észre, hogy a harmadik sor elemei
    pont az előző két sor megfelelő elemeinek összegei. Ebből következik,
    hogy a mátrix rangja nem maximális, vagyis $\det \rmat B = 0$.
  }
\end{exercise}

  \vfill
  \begin{exercise}{Határozzuk meg az alábbi mátrixok rangjait!}
  \[
    \rmat A = \begin{bmatrix}
      1 & 3 & 4 & 5 \\
      3 & 6 & 9 & 9 \\
      2 & 3 & 5 & 4
    \end{bmatrix} % (2)
    \hspace{2cm}
    \rmat B = \begin{bmatrix}
      2  & -2  & 1 & 6  \\
      4  & -4  & 2 & -2 \\
      10 & -10 & 5 & 2
    \end{bmatrix} % (1)
  \]

  \exsol{
    Először határozzuk meg az $\rmat A$ mátrix rangját!
    \begin{align*}
      \rg \rmat A
       & = \rg
      \begin{bamatrix}{4}{5.5mm}
        1 & 3 & 4 & 5 \\
        3 & 6 & 9 & 9 \\
        2 & 3 & 5 & 4
      \end{bamatrix}
      \;
      \begin{matrix}
        \\\\(+S_1 - S_2)
      \end{matrix}
      \\
       & = \rg
      \begin{bamatrix}{4}{5.5mm}
        1 & 3 & 4 & 5 \\
        3 & 6 & 9 & 9 \\
        0 & 0 & 0 & 0
      \end{bamatrix}
      \;
      \begin{matrix}
        \\(-3S_1)\\\,
      \end{matrix}
      \\
       & = \rg
      \begin{bamatrix}{4}{5.5mm}
        \gcc{1} & \gcc{3}  & \gcc{4}  & \gcc{5}  \\
        0       & \gcc{-3} & \gcc{-3} & \gcc{-6} \\
        0       & 0        & 0        & 0
      \end{bamatrix}
      \\
       & =2
    \end{align*}

    \tcbline

    Most pedig határozzuk meg $\rmat B$ mátrix rangját!
    \begin{align*}
      \rg \rmat B
       & = \rg
      \begin{bamatrix}{4}{7.5mm}
        2  & -2  & 1 & 6  \\
        4  & -4  & 2 & -2 \\
        10 & -10 & 5 & 2
      \end{bamatrix}
      \;
      \begin{matrix}
        \\\\(-S_1 - 2S_2)
      \end{matrix}
      \\
       & = \rg
      \begin{bamatrix}{4}{7.5mm}
        2  & -2  & 1 & 6  \\
        4  & -4  & 2 & -2 \\
        0 & 0 & 0 & 0
      \end{bamatrix}
      \;
      \begin{matrix}
        \\(-2S_1)\\\,
      \end{matrix}
      \\
       & = \rg
      \begin{bamatrix}{4}{7.5mm}
        \gcc{2} & \gcc{-2} & \gcc{1} & \gcc{6}   \\
        0       & 0        & 0       & \gcc{-14} \\
        0       & 0        & 0       & 0
      \end{bamatrix}
      \\
       & =2
    \end{align*}
  }
\end{exercise}

  \vfill
\end{frame}

\begin{frame}
  \frametitle{Mátrixok}
  \framesubtitle{Mátrix inverz}

  \vfill

  \begin{block}{Reguláris / Szinguláris mátrix}
    Egy kvadratikus ($\rmat A \in \mathbb R^{n \times n}$) mátrixot
    \textbf{reguláris}nak mondunk, ha determinánsa nem 0.
    \\[3mm]
    Ha a kvadratikus mátrix determinánsa 0, akkor \textbf{szinguláris} mátrixról
    beszélünk.
  \end{block}

  \vfill

  \begin{block}{Inverz}
    Az $\rmat A \in \mathbb R^{n \times n}$ reguláris mátrix inverze alatt azt
    az $\rmat A^{-1} \in \mathbb R^{n \times n}$ mátrixot értjük, melyre
    $\rmat A \cdot \rmat A^{-1} = \imat$ egyenlőség teljesül.
    \\[3mm]
    Szinguláris mátrixnak nem létezik az inverze.
  \end{block}

  \vfill
\end{frame}

\begin{frame}
  \frametitle{Mátrixok}
  \framesubtitle{Inverz meghatározása}

  \begin{block}{Adjugált mátrix segítségével}
    \[
      \rmat A^{-1} = \frac{\adj \rmat A}{\det \rmat A}
      \hspace{1cm}
      \rmat A = \begin{bmatrix}
        a_{11} & a_{12} & a_{13} \\
        a_{21} & a_{22} & a_{23} \\
        a_{31} & a_{32} & a_{33}
      \end{bmatrix}
    \]
    \[
      \adj \rmat A = \begin{bmatrix}
        + \begin{vmatrix}
            a_{22} & a_{23} \\
            a_{32} & a_{33}
          \end{vmatrix}
         &
        - \begin{vmatrix}
            a_{12} & a_{13} \\
            a_{32} & a_{33}
          \end{vmatrix}
         &
        +\begin{vmatrix}
           a_{12} & a_{13} \\
           a_{22} & a_{23}
         \end{vmatrix}
        \\
        - \begin{vmatrix}
            a_{21} & a_{23} \\
            a_{31} & a_{33}
          \end{vmatrix}
         &
        + \begin{vmatrix}
            a_{11} & a_{13} \\
            a_{31} & a_{33}
          \end{vmatrix}
         &
        - \begin{vmatrix}
            a_{11} & a_{13} \\
            a_{21} & a_{23}
          \end{vmatrix}
        \\
        + \begin{vmatrix}
            a_{21} & a_{22} \\
            a_{31} & a_{32}
          \end{vmatrix}
         &
        - \begin{vmatrix}
            a_{11} & a_{12} \\
            a_{31} & a_{32}
          \end{vmatrix}
         &
        +\begin{vmatrix}
           a_{11} & a_{12} \\
           a_{21} & a_{22}
         \end{vmatrix}
      \end{bmatrix}
    \]
  \end{block}
\end{frame}

\begin{frame}
  \frametitle{Mátrixok}
  \framesubtitle{Inverz meghatározása}

  \vfill

  \begin{block}{Gauss-Jordan eliminációval}
    \[
      \left[\begin{array}{*{3}{>{\cdot}{c}}|*{3}c}
           &  &  & 1 & 0 & 0 \\
           &  &  & 0 & 1 & 0 \\
           &  &  & 0 & 0 & 1 \\
        \end{array}\right]
      \quad \sim \quad
      \left[\begin{array}{*{3}{c}|*{3}{>{\cdot}{c}}}
          1 & 0 & 0 &  &  & \\
          0 & 1 & 0 &  &  & \\
          0 & 0 & 1 &  &  & \\
        \end{array}\right]
    \]
  \end{block}

  \vfill

  \begin{exercise}{Határozzuk meg az alábbi mátrixok inverzét!}
  \[
    \rmat A = \begin{bmatrix}
      2 & 3 \\
      3 & 5
    \end{bmatrix}
    \hspace{2cm}
    \rmat B = \begin{bmatrix}
      1 & 4 & 8 \\
      0 & 1 & 0 \\
      2 & 4 & 6
    \end{bmatrix}
  \]

  \exsol{
    Határozzuk meg $\rmat A$ inverzét mindkét tanult módszer segítségével:
    \begin{itemize}
      \item Definíció szerint:
            \begin{itemize}
              \item A mátrix determinánsa:
                    \[
                      \det \rmat A = \begin{vmatrix}
                        2 & 3 \\
                        3 & 5 \\
                      \end{vmatrix} = 2 \cdot 5 - 3 \cdot 3 = 1
                      \text.
                    \]
              \item A mátrix transzponáltja:
                    \[
                      \rmat A^{\mathsf T} = \begin{bmatrix}
                        2 & 3 \\
                        3 & 5 \\
                      \end{bmatrix}
                      \text.
                    \]
              \item A mátrix adjugáltja:
                    \[
                      \adj \rmat A = \begin{bmatrix}
                        +5 & -3 \\
                        -3 & +2
                      \end{bmatrix}
                      \text.
                    \]
              \item Az inverz ezek alapján:
                    \[
                      \rmat A^{-1}
                      = \frac{\adj \rmat A}{\det \rmat A}
                      = \frac{1}{1} \begin{bmatrix}
                        5  & -3 \\
                        -3 & 2
                      \end{bmatrix}  = \begin{bmatrix}
                        5  & -3 \\
                        -3 & 2
                      \end{bmatrix}
                      \text.
                    \]
            \end{itemize}

      \item Gauss-Jordan eliminációval:
            \begin{align*}
               & \left[\begin{array}{X{6mm}X{8mm}|X{9mm}X{6mm}}
                           2 & 3 & 1 & 0 \\
                           3 & 5 & 0 & 1
                         \end{array}\right]\begin{matrix}(/2)\\\,\end{matrix}
              \\
               & \left[\begin{array}{X{6mm}X{8mm}|X{9mm}X{6mm}}
                           1 & 3/2 & 1/2 & 0 \\
                           3 & 5   & 0   & 1
                         \end{array}\right]\begin{matrix}\\(-3S_1)\end{matrix}
              \\
               & \left[\begin{array}{X{6mm}X{8mm}|X{9mm}X{6mm}}
                           1 & 3/2 & 1/2  & 0 \\
                           0 & 1/2 & -3/2 & 1
                         \end{array}\right]\begin{matrix}(-3S_2)\\\,\end{matrix}
              \\
               & \left[\begin{array}{X{6mm}X{8mm}|X{9mm}X{6mm}}
                           1 & 0   & 5    & -3 \\
                           0 & 1/2 & -3/2 & 1
                         \end{array}\right]\begin{matrix}(\cdot2)\\\,\end{matrix}
              \\
               & \left[\begin{array}{X{6mm}X{8mm}|X{9mm}X{6mm}}
                           1 & 0 & 5  & -3 \\
                           0 & 1 & -3 & 2
                         \end{array}\right]
            \end{align*}
    \end{itemize}

    \tcbline

    Határozzuk meg $\rmat B$ inverzét mindkét tanult módszer segítségével:
    \begin{itemize}
      \item Definíció alapján:
            \begin{itemize}
              \item A mátrix determinánsát már korábban meghatároztuk:
                    \[
                      \det \rmat B = -10
                      \text.
                    \]

              \item A mátrix transzponáltja:
                    \[
                      \rmat B^{\mathsf T} = \begin{bmatrix}
                        1 & 0 & 2 \\
                        4 & 1 & 4 \\
                        8 & 0 & 6
                      \end{bmatrix}
                      \text.
                    \]
              \item A mátrix adjugáltja:
                    \newcommand{\qvmat}[4]{\begin{vmatrix}#1 & #2 \\ #3 & #4\end{vmatrix}}
                    \[
                      \adj \rmat B = \begin{bmatrix}
                        +\qvmat{1}{4}{0}{6} & -\qvmat{4}{4}{8}{6} & +\qvmat{4}{1}{8}{0} \\
                        -\qvmat{0}{2}{0}{6} & +\qvmat{1}{2}{8}{6} & -\qvmat{1}{0}{8}{0} \\
                        +\qvmat{0}{2}{1}{4} & -\qvmat{1}{2}{4}{4} & +\qvmat{1}{0}{4}{1} \\
                      \end{bmatrix} = \begin{bmatrix}
                        6  & 8   & -8 \\
                        0  & -10 & 0  \\
                        -2 & 4   & 1  \\
                      \end{bmatrix}
                      \text.
                    \]
              \item Az inverz ezek alapján:
                    \begin{align*}
                      \rmat B^{-1}
                      = \frac{\adj \rmat B}{\det \rmat B}
                      = & \frac{1}{-10}
                      \left[\begin{array}{X{1.2cm}X{1.2cm}X{1.2cm}}
                                6  & 8   & -8 \\
                                0  & -10 & 0  \\
                                -2 & 4   & 1  \\
                              \end{array}\right]
                      \\
                      = & \phantom{\frac{1}{-10}} \left[\begin{array}{X{1.2cm}X{1.2cm}X{1.2cm}}
                                                            -6/10 & -8/10 & 8/10  \\
                                                            0     & 1     & 0     \\
                                                            2/10  & -4/10 & -1/10
                                                          \end{array}\right]
                      \\
                      = & \phantom{\frac{1}{-10}} \left[\begin{array}{X{1.2cm}X{1.2cm}X{1.2cm}}
                                                            -3/5 & -4/5 & 4/5   \\
                                                            0    & 1    & 0     \\
                                                            1/5  & -2/5 & -1/10
                                                          \end{array}\right]
                      \text.
                    \end{align*}
            \end{itemize}

      \item Gauss-Jordan eliminációval:
            \newcommand{\qgj}[6]{\left[\begin{array}{*{3}{X{12mm}}|*{3}{X{12mm}}}
                  #1 \\#3\\#5
                \end{array}\right]\begin{matrix}#2\\#4\\#6\end{matrix}}
            \begin{align*}
                 & \qgj
              {1 & 4    & 8   & 1     & 0     & 0}{}
              {0 & 1    & 0   & 0     & 1     & 0}{}
              {2 & 4    & 6   & 0     & 0     & 1}{(-2S_1)}
              \\
                 & \qgj
              {1 & 4    & 8   & 1     & 0     & 0}{(-4S_2)}
              {0 & 1    & 0   & 0     & 1     & 0}{}
              {0 & -4   & -10 & -2    & 0     & 1}{(+4S_2)}
              \\
                 & \qgj
              {1 & 0    & 8   & 1     & -4    & 0}{}
              {0 & 1    & 0   & 0     & 1     & 0}{}
              {0 & 0    & -10 & -2    & 4     & 1}{(/(-10))}
              \\
                 & \qgj
              {1 & 0    & 8   & 1     & -4    & 0}{(-8S_3)}
              {0 & 1    & 0   & 0     & 1     & 0}{}
              {0 & 0    & 1   & 2/10  & -4/10 & -1/10}{}
              \\
                 & \qgj
              {1 & 0    & 0   & -6/10 & -8/10 & 8/10}{}
              {0 & 1    & 0   & 0     & 1     & 0}{}
              {0 & 0    & 1   & 2/10  & -4/10 & -1/10}{}
              \\
                 & \qgj
              {1 & 0    & 0   & -3/5  & -4/5  & 4/5}{}
              {0 & 1    & 0   & 0     & 1     & 0}{}
              {0 & 0    & 1   & 1/5   & -2/5  & -1/10}{}
            \end{align*}
    \end{itemize}
  }
\end{exercise}


  \vfill
\end{frame}

\begin{frame}
  \frametitle{Mátrixok}
  \framesubtitle{Mátrix egyenletek}

  \begin{exercise}{Oldjuk meg az alábbi mátrix-egyenleteket!}
  \[
    \rmat A = \begin{bmatrix}
      2 & 3 \\
      3 & 5
    \end{bmatrix}
    \hspace{2cm}
    \rmat B = \begin{bmatrix}
      1 & 2 \\
      3 & 4 \\
      6 & 3
    \end{bmatrix}
  \]
  \begin{enumerate}[a)]
    \item $\rmat X \cdot \rmat A = \rmat B$
    \item $\rmat A \cdot \rmat X = \rmat B$
    \item $2(\rmat A + \rmat X) = 3(\rmat X - \rmat A^{-1})$
    \item $\rmat B \cdot \rmat B^{\mathsf T} = \rmat A \cdot \rmat X$
  \end{enumerate}

  \exsol[18.25cm]{
    \begin{enumerate}[a)]
      \item $\rmat X \cdot \rmat A = \rmat B$

            Rendezzük $\rmat X$-re az egyenletet, vagyis szorozzuk meg az
            egyenlet mindkét oldalát $\rmat A$ inverzével. Ekkor az alábbi
            egyenletet kapjuk:
            \[
              \rmat X = \rmat B \cdot \rmat A^{-1}
              \text{.}
            \]
            Az $\rmat A$ mátrix inverzét már korábban meghatároztuk.
            Az egyenletbe behelyettesítve:
            \[
              \rmat X
              = \rmat B \cdot\rmat A^{-1} = \begin{bmatrix}
                1 & 2 \\
                3 & 4 \\
                6 & 3
              \end{bmatrix} \begin{bmatrix}
                5  & -3 \\
                -3 & 2
              \end{bmatrix} = \begin{bmatrix}
                5 - 6   & -3 + 4  \\
                15 - 12 & -9 + 8  \\
                30-9    & -18 + 6
              \end{bmatrix} = \begin{bmatrix}
                -1 & 1   \\
                3  & -1  \\
                21 & -12
              \end{bmatrix}
              \text.
            \]

            \tcbline
      \item $\rmat A \cdot \rmat X = \rmat B$

            Ez az egyenlet nem megoldható, hiszen a $\rmat A^{-1} \rmat B$
            szorzás nem értelmezett, hiszen $\rmat A^{-1} \in
              \mathbb R^{2 \times 2}$, $\rmat B \in \mathbb R^{3 \times 2}$,
            vagyis az $\rmat A^{-1}$ mátrix oszlopainak száma nem egyezik meg
            a $\rmat B$ mátrix sorainak számával.

            \tcbline
      \item $2(\rmat A + \rmat X) = 3(\rmat X - \rmat A^{-1})$

            Rendezzük $\rmat X$-re az egyenletet:
            \begin{gather*}
              2 \rmat A + 2 \rmat X = 3 \rmat X - 3 \rmat A^{-1}
              \quad \rightarrow \quad
              \rmat X = 2 \rmat A + 3 \rmat A^{-1}
              \\
              \rmat X = 2 \begin{bmatrix}
                2 & 3 \\ 3 & 5
              \end{bmatrix} + 3 \begin{bmatrix}
                5 & -3 \\ -3 & 2
              \end{bmatrix} = \begin{bmatrix}
                19 & -3 \\ -3 & 16
              \end{bmatrix}
            \end{gather*}

            \tcbline
      \item $\rmat B \cdot \rmat B^{\mathsf T} = \rmat A \cdot \rmat X$

            Az egyenlet nem megoldható, hiszen $(\rmat B \cdot \rmat B^{\mathsf T})
              \in \mathbb R^{3 \times 3}$, $\rmat A \in \mathbb R^{2 \times 2}$.
    \end{enumerate}
  }
\end{exercise}

\end{frame}
