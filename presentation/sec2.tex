\section{Mátrixok}
\begin{frame}
  \frametitle{Mátrixok}
  \framesubtitle{Alapfogalmak I}

  \vfill

  \begin{block}{Mátrix}
    Az $m$ sorba és $n$ oszlopba rendezett rendezett számokat mátrixoknak
    nevezzük.
  \end{block}

  \vfill

  \begin{block}{Speciális elnevezések}
    \begin{itemize}
      \item $\rmat M \in \mathbb R^{n \times 1}$ -- oszlopmátrix
      \item $\rmat M \in \mathbb R^{1 \times n}$ -- sormátrix
      \item $\rmat M \in \mathbb R^{n \times n}$ -- kvadratikus mátrix
            \[
              \imat = \overset{\text{egységmátrix}}{\begin{bmatrix}
                  1      & \dots  & 0      \\
                  \vdots & \ddots & \vdots \\
                  0      & \dots  & 1
                \end{bmatrix}}
              \hspace{2cm}
              \nmat = \overset{\text{nullmátrix}}{\begin{bmatrix}
                  0      & \dots  & 0      \\
                  \vdots & \ddots & \vdots \\
                  0      & \dots  & 0
                \end{bmatrix}}
            \]
    \end{itemize}
  \end{block}

  \vfill
\end{frame}

\begin{frame}
  \frametitle{Mátrixok}
  \framesubtitle{Alapfogalmak II}

  \vfill

  \begin{block}{Transzponált}
    Egy mátrix transzponáltja a főátlóra való tükörképe. Jele: $A^\mathsf T$.
  \end{block}

  \vfill

  \begin{block}{Szimmetrikus mátrix}
    Ha $A = A^\mathsf T$, akkor a mátrix szimmetrikus.
  \end{block}

  \vfill

  \begin{block}{Antiszimmetrikus mátrix}
    Ha $A = -A^\mathsf T$, akkor a mátrix antiszimmetrikus.
  \end{block}

  \vfill
\end{frame}

\begin{frame}
  \frametitle{Mátrixok}
  \framesubtitle{Elemi mátrixműveletek}

  \vfill

  \begin{block}{Összeadás}
    Ha $\rmat A; \rmat B \in \mathbb R^{n \times k}$, akkor az összegükön a
    megfelelő elempárok összeadásával keletkező mátrixot értjük.
  \end{block}

  \vfill

  \begin{block}{Skalárral való szorzás}
    Egy mátrix és egy skalár szorzata olyan mátrix, melynek minden eleme
    skalárszorosa az eredeti mátrix elemeinek.
  \end{block}

  \vfill
\end{frame}

\begin{frame}
  \frametitle{Mátrixok}
  \framesubtitle{Elemi mátrixműveletek}

  \begin{block}{Mátrix szorzás
      -- asszociatív, disztributív, de nem kommutatív!}
    \[
      \left.\begin{array}{ll}
        \rmat A \in \mathbb R^{m \times n} \\
        \rmat B \in \mathbb R^{n \times p}
      \end{array}\right\}
      \; \rightarrow \;
      \rmat A \cdot \rmat B \in \mathbb R^{m \times p}
    \]

    \def\arraystretch{1.1}
    \begin{align*}
       & \left[\begin{array}{X{2cm}cX{2cm}}
                   b_{11} & \dots  & b_{1p} \\
                   b_{21} & \dots  & b_{2p} \\
                   \vdots & \ddots & \vdots \\
                   b_{n1} & \dots  & b_{np}
                 \end{array}\right]
      \\
      \left[\begin{array}{cccc}
                a_{11} & a_{12} & \dots  & a_{1n} \\
                a_{21} & a_{22} & \dots  & a_{2n} \\
                \vdots & \vdots & \ddots & \vdots \\
                a_{m1} & a_{m2} & \dots  & a_{mn}
              \end{array}\right]
       & \left[\begin{array}{X{2cm}cX{2cm}}
                   \sum a_{1i} b_{i1} & \dots  & \sum a_{1i} b_{ip} \\
                   \sum a_{2i} b_{i1} & \dots  & \sum a_{2i} b_{ip} \\
                   \vdots             & \ddots & \vdots             \\
                   \sum a_{mi} b_{i1} & \dots  & \sum a_{mi} b_{ip}
                 \end{array}\right]
    \end{align*}
  \end{block}
\end{frame}

\begin{frame}
  \frametitle{Mátrixok}
  \framesubtitle{Determináns}

  \begin{block}{A determináns axonometrikus felépítése}
    Legyen $\rmat A \in \mathbb R^{n \times n}$ kvadratikus mátrix, és
    $\det: \mathbb R^{n \times n} \rightarrow \mathbb R$ függvény.
    Az $\rmat A$ mátrix determinánsának nevezzük $\det \rmat A$-t,
    a hozzárendelést pedig az alábbi négy axióma írja le:
    \newcommand\noskp{\vspace{-3mm}}
    \begin{enumerate}
      \item homogén:
            \[
              \edet{\lambda \rvec a_i} = \lambda \edet{\rvec a_i}
            \]
      \item \noskp additív:\noskp
            \[
              \edet{\rvec a_i + \rvec b_i} =
              \edet{\rvec a_i} + \edet{\rvec b_i}
            \]
      \item \noskp alternáló:\noskp
            \[
              \edet{\rvec a_i & \dots & \rvec a_j} =
              - \edet{\rvec a_j & \dots & \rvec a_i}
            \]
      \item \noskp $\imat$ determinánsa:
            \[
              \det \begin{pmatrix}
                \uvec e_1 & \uvec e_2 & \dots & \uvec e_n
              \end{pmatrix} = 1
            \]
    \end{enumerate}
  \end{block}
\end{frame}

\begin{frame}
  \frametitle{Mátrixok}
  \framesubtitle{Determináns tulajdonságai}

  \begin{block}{Tételek}
    \begin{itemize}
      \item Egy mátrix determinánsa 0, ha \dots
            \begin{itemize}
              \item az egyik oszlopvektora nullvektor,
              \item van két azonos oszlopvektora,
              \item az oszlopvektorai lineárisan összefüggőek.
            \end{itemize}
      \item Nem változik a determináns, ha az egyik oszlophoz hozzáadjuk egy
            másik oszlop skalárszorosát.
      \item Mátrix és transzponáltjának determinánsa megegyezik.
      \item Determinánsok szorzástétele:
            $\det (\rmat A \cdot \rmat B) = \det \rmat A \cdot \det \rmat B$.
      \item Ha $\rmat A \in \mathbb R^{n \times n}$, akkor
            $\det (\lambda \rmat A) = \lambda^n \det \rmat A$.
      \item \(
            \rvec a \cdot (\rvec b \times \rvec c) = \det \begin{pmatrix}
              \rvec a & \rvec b & \rvec c
            \end{pmatrix}
            \)
    \end{itemize}
  \end{block}
\end{frame}

\begin{frame}
  \frametitle{Mátrixok}
  \framesubtitle{Determinánsos tételek alkalmazása}

  \input{exercise/determinant-theorems}
\end{frame}

\begin{frame}
  \frametitle{Mátrixok}
  \framesubtitle{Determináns számítása}

  \begin{block}{Kifejtési tétel -- előjelszabály!}
    \begin{align*}
      \begin{vmatrix}
        + & - & + \\
        - & + & - \\
        + & - & + \\
      \end{vmatrix}
      \; \rightarrow \;
      \begin{vmatrix}
        a & b & c \\
        d & e & f \\
        g & h & i \\
      \end{vmatrix}
       & = a \begin{vmatrix}
               e & f \\ h & i
             \end{vmatrix}
      - b \begin{vmatrix}
            d & f \\ g & i
          \end{vmatrix}
      + c \begin{vmatrix}
            d & e \\ g & h
          \end{vmatrix}
      \\
       & = a (ei - hf) - b(di - gf) + c(dh - eg)
    \end{align*}
  \end{block}

  \begin{block}{Sarrus-szabály -- csak ($3 \times 3$)-as mátrixoknál!}
    \centering
    \begin{tikzpicture}[ampersand replacement=\&]
      \matrix[
        matrix of math nodes,
        column sep=2mm,
      ] (sarrus) {
        a\vphantom{b} \& b \& c\vphantom{b} \& a \& b \\
        d \& e \& f \& d \& e                         \\
        g \& h\vphantom{g} \& i\vphantom{g} \& g \& h \\
      };

      \draw[red!40!gray, ultra thick, opacity=.5]
      (sarrus-1-1.center) -- (sarrus-3-3.center)
      (sarrus-1-2.center) -- (sarrus-3-4.center)
      (sarrus-1-3.center) -- (sarrus-3-5.center)
      ;

      \draw[blue!40!gray, ultra thick, opacity=.5]
      (sarrus-3-1.center) -- (sarrus-1-3.center)
      (sarrus-3-2.center) -- (sarrus-1-4.center)
      (sarrus-3-3.center) -- (sarrus-1-5.center)
      ;

      \draw[black, thick]
      (sarrus-1-1.north west) -- (sarrus-3-1.south west)
      (sarrus-1-3.north east) -- (sarrus-3-3.south east)
      ;

      \foreach \i in {1,2,3}{
          \node[above=-2.0mm, red!40!gray] at (sarrus-1-\i.north) {$+$};
          \node[below=-1.5mm, blue!40!gray] at (sarrus-3-\i.south) {$-$};
        }

      \node[] at (5,.25) {$\det \rmat A = + aei + bfg + cdh$};
      \node[] at (5,-.25) {$\phantom{\det \rmat A =} - gec - hfa - idb$};
    \end{tikzpicture}
  \end{block}
\end{frame}

\begin{frame}
  \frametitle{Mátrixok}
  \framesubtitle{Rang}

  \begin{block}{Mátrix rangja}
    A mátrix rangjának nevezzük az oszlopvektorai közül a lineárisan függetlenek
    maximális számát. A mátrix rangja elemi átalakítások során nem változik:
    \begin{itemize}
      \item tetszőleges sorát vagy oszlopát egy 0-tól különböző számmal
            megszorozzuk,
      \item tetszőleges sorát vagy oszlopát felcseréljük,
      \item tetszőleges sorához vagy oszlopához egy másik tetszőleges sorát
            vagy oszlopát adjuk.
    \end{itemize}
  \end{block}
\end{frame}

\begin{frame}
  \frametitle{Mátrixok}
  \framesubtitle{Mátrixos feladatok}

  \vfill
  \input{exercise/determinant}
  \vfill
  \input{exercise/rank}
  \vfill
\end{frame}

\begin{frame}
  \frametitle{Mátrixok}
  \framesubtitle{Mátrix inverz}

  \vfill

  \begin{block}{Reguláris / Szinguláris mátrix}
    Egy kvadratikus ($\rmat A \in \mathbb R^{n \times n}$) mátrixot
    \textbf{reguláris}nak mondunk, ha determinánsa nem 0.
    \\[3mm]
    Ha a kvadratikus mátrix determinánsa 0, akkor \textbf{szinguláris} mátrixról
    beszélünk.
  \end{block}

  \vfill

  \begin{block}{Inverz}
    Az $\rmat A \in \mathbb R^{n \times n}$ reguláris mátrix inverze alatt azt
    az $\rmat A^{-1} \in \mathbb R^{n \times n}$ mátrixot értjük, melyre
    $\rmat A \cdot \rmat A^{-1} = \imat$ egyenlőség teljesül.
    \\[3mm]
    Szinguláris mátrixnak nem létezik az inverze.
  \end{block}

  \vfill
\end{frame}

\begin{frame}
  \frametitle{Mátrixok}
  \framesubtitle{Inverz meghatározása}

  \begin{block}{Adjugált mátrix segítségével}
    \[
      \rmat A^{-1} = \frac{\adj \rmat A}{\det \rmat A}
      \hspace{1cm}
      \rmat A = \begin{bmatrix}
        a_{11} & a_{12} & a_{13} \\
        a_{21} & a_{22} & a_{23} \\
        a_{31} & a_{32} & a_{33}
      \end{bmatrix}
    \]
    \[
      \adj \rmat A = \begin{bmatrix}
        + \begin{vmatrix}
            a_{22} & a_{23} \\
            a_{32} & a_{33}
          \end{vmatrix}
         &
        - \begin{vmatrix}
            a_{12} & a_{13} \\
            a_{32} & a_{33}
          \end{vmatrix}
         &
        +\begin{vmatrix}
           a_{12} & a_{13} \\
           a_{22} & a_{23}
         \end{vmatrix}
        \\
        - \begin{vmatrix}
            a_{21} & a_{23} \\
            a_{31} & a_{33}
          \end{vmatrix}
         &
        + \begin{vmatrix}
            a_{11} & a_{13} \\
            a_{31} & a_{33}
          \end{vmatrix}
         &
        - \begin{vmatrix}
            a_{11} & a_{13} \\
            a_{21} & a_{23}
          \end{vmatrix}
        \\
        + \begin{vmatrix}
            a_{21} & a_{22} \\
            a_{31} & a_{32}
          \end{vmatrix}
         &
        - \begin{vmatrix}
            a_{11} & a_{12} \\
            a_{31} & a_{32}
          \end{vmatrix}
         &
        +\begin{vmatrix}
           a_{11} & a_{12} \\
           a_{21} & a_{22}
         \end{vmatrix}
      \end{bmatrix}
    \]
  \end{block}
\end{frame}

\begin{frame}
  \frametitle{Mátrixok}
  \framesubtitle{Inverz meghatározása}

  \vfill

  \begin{block}{Gauss-Jordan eliminációval}
    \[
      \left[\begin{array}{*{3}{>{\cdot}{c}}|*{3}c}
           &  &  & 1 & 0 & 0 \\
           &  &  & 0 & 1 & 0 \\
           &  &  & 0 & 0 & 1 \\
        \end{array}\right]
      \quad \sim \quad
      \left[\begin{array}{*{3}{c}|*{3}{>{\cdot}{c}}}
          1 & 0 & 0 &  &  & \\
          0 & 1 & 0 &  &  & \\
          0 & 0 & 1 &  &  & \\
        \end{array}\right]
    \]
  \end{block}

  \vfill

  \input{exercise/inverse.tex}

  \vfill
\end{frame}

\begin{frame}
  \frametitle{Mátrixok}
  \framesubtitle{Mátrix egyenletek}

  \begin{exercise}{Oldjuk meg az alábbi mátrix-egyenleteket!}
	\[
		\rmat A = \begin{bmatrix}
			2 & 3 \\
			3 & 5
		\end{bmatrix}
		\hspace{1cm}
		\rmat B = \begin{bmatrix}
			1 & 2 \\
			3 & 4 \\
			6 & 3
		\end{bmatrix}
		\hspace{1cm}
		\rmat Q = \begin{bmatrix}
			\sqrt{3}/2 & -1/2       \\
			1/2        & \sqrt{3}/2
		\end{bmatrix}
	\]
	\begin{multicols}{2}
		\begin{enumerate}
			\item $\rmat X \cdot \rmat A = \rmat B$
			\item $\rmat A \cdot \rmat X = \rmat B$
			\item $2(\rmat A + \rmat X) = 3(\rmat A^{-1} + \rmat X)$
			\item $\rmat X = \rmat A \cdot \rmat Q^{6}$
		\end{enumerate}
	\end{multicols}

	\exsol{
		\begin{enumerate}
			\item $\rmat X \cdot \rmat A = \rmat B$

			      Rendezzük $\rmat X$-re az egyenletet, vagyis szorozzuk meg az
			      egyenlet mindkét oldalát $\rmat A$ inverzével. Ekkor az alábbi
			      egyenletet kapjuk:
			      \[
				      \rmat X = \rmat B \cdot \rmat A^{-1}
				      \text{.}
			      \]
			      Az $\rmat A$ mátrix inverzét már korábban meghatároztuk.
			      Az egyenletbe behelyettesítve:
			      \[
				      \rmat X
				      = \rmat B \cdot\rmat A^{-1} = \begin{bmatrix}
					      1 & 2 \\
					      3 & 4 \\
					      6 & 3
				      \end{bmatrix} \begin{bmatrix}
					      5  & -3 \\
					      -3 & 2
				      \end{bmatrix} = \begin{bmatrix}
					      5 - 6   & -3 + 4  \\
					      15 - 12 & -9 + 8  \\
					      30-9    & -18 + 6
				      \end{bmatrix} = \begin{bmatrix}
					      -1 & 1   \\
					      3  & -1  \\
					      21 & -12
				      \end{bmatrix}
				      \text.
			      \]

			      \tcbline
			\item $\rmat A \cdot \rmat X = \rmat B$
			      Ez az egyenlet nem megoldható.

			      \tcbline
			\item $2(\rmat A + \rmat X) = 3(\rmat A^{-1} + \rmat X)$

			      \tcbline
			\item $\rmat X = \rmat A \cdot \rmat Q^{6}$
		\end{enumerate}
	}
\end{exercise}

\end{frame}
