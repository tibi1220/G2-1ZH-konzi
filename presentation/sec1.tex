\section{Vektorterek}
\begin{frame}
  \frametitle{Vektorterek}
  \framesubtitle{Definíció}

  \justifying
  Legyen $V$ nem üreshalmaz, és $+; \lambda$ két művelet, valamint $T$ test.
  $(V; +; \lambda)$ a $T$ test feletti vektortér, ha az alábbiak teljesülnek:
  \begin{itemize}
    \def\arraystretch{1.2}
    \item $(V; +)$ Abel csoport:\\[1mm]
          \begin{tabular}{p{35mm} l}
            -- asszociatív:        &
            $\rvec a + (\rvec b + \rvec c) = (\rvec a + \rvec b) + \rvec c$,
            \\
            -- kommutatív:         &
            $\rvec a + \rvec b = \rvec b + \rvec a$,
            \\
            -- létezik zérus elem: &
            $\exists \nvec \in V \text{, melyre } \rvec a + \nvec = \rvec a$,
            \\
            -- létezik inverz:     &
            $\forall \rvec a \text{-re } \exists - \rvec a \text{, hogy } \rvec v + (- \rvec v) = \nvec$.
          \end{tabular}
    \item $(V; \lambda)$-ra pedig igaz:\\[1mm]
          \begin{tabular}{p{35mm} l}
            -- asszociatív:      & $(\alpha \beta) \rvec a = \alpha (\beta \rvec a)$.
            \\
            -- egységelem:       & $\varepsilon \in T \text{-re } \varepsilon\rvec a = \rvec a$,
            \\
            -- disztributivitás: & $\alpha(\rvec a + \rvec b) = \alpha \rvec a + \alpha \rvec b$,
            \\
                                 & $(\alpha + \beta) \rvec a = \alpha \rvec a + \beta \rvec a$.
          \end{tabular}
  \end{itemize}
\end{frame}

\begin{frame}
  \frametitle{Vektorterek}
  \framesubtitle{További definíciók I}

  \begin{block}{Lineáris függőség / függetlenség}
    A $(V; +; \lambda)$ vektortér $\rvec a_1; \rvec a_2; \dots; \rvec a_n$
    vektorait \textbf{lineárisan függő}nek mondjuk, ha a
    \[
      \lambda_1 \rvec a_1 +
      \lambda_2 \rvec a_2 +
      \dots +
      \lambda_n \rvec a_n =
      \nvec
    \]
    vektoregyenletnek létezik a triviálistól különböző megoldása is.
    \\[2mm]
    Ellenkező esetben \textbf{lineárisan független}ek.
  \end{block}

  \begin{block}{Altér}
    Legyen $(V; +; \lambda)$ a $T$ test feletti vektortér, és
    $\emptyset \neq L \subset V$. $L$-t altérnek nevezzük $V$-ben, ha
    $(L; +; \lambda)$ ugyancsak vektortér.
  \end{block}
\end{frame}

\begin{frame}
  \frametitle{Vektorterek}
  \framesubtitle{További definíciók II}

  \begin{block}{Generátorrendszer}
    Legyen $\emptyset \neq G \subset V$. Ekkor $G$ által generált altérnek
    nevezzük azt a legszűkebb alteret, amely tartalmazza $G$-t. Ha ez az
    altér maga $V$, akkor $G$ generátorrendszere $V$-nek. ($\mathcal L(G)=V$)
  \end{block}

  \begin{block}{Bázis}
    A $V$ vektortér egy lineárisan független generátorrendszerét a $V$ bázisának
    hívjuk.
  \end{block}

  \begin{block}{Vektortér dimenziója}
    Végesen generált vektortér dimenzióján a bázisainak közös tagszámát értjük.
  \end{block}
\end{frame}

\begin{frame}
  \frametitle{Vektorterek}
  \framesubtitle{Feladatok I}

  \input{exercise/linear-independence}
\end{frame}

\begin{frame}
  \frametitle{Vektorterek}
  \framesubtitle{Feladatok II}

  \begin{exercise}{%
		Döntsük el, hogy vektorteret alkotnak-e az alábbi számhármasok
		$\mathbb R^3$-ban?
	}
	\newcommand{\setmap}[2]{\Big\{\, #1 \,\Big|\, #2 \,\Big\}}
	\begin{enumerate}[a)]
		\item $Q_1 = \setmap{(x_1; x_2; x_3)}{x_1 + x_2 = 0}$
		\item $Q_2 = \setmap{(x_1; x_2; x_3)}{x_1 = \pi}$
		\item $Q_3 = \setmap{(x_1; x_2; x_3)}{x_1 = x_2 = x_3}$
		\item $Q_3 = \setmap{(x_1; x_2; x_3)}{x_1 = (x_2)^2}$
	\end{enumerate}

	\exsol{%
		\begin{enumerate}[a)]
			\item Igen, mert a műveletek sosem mutatnak ki a vektortérből, hiszen ha\\
			      $\alpha(x_{11} + x_{12}) = 0$,
			      akkor $\alpha x_{11} + \alpha x_{12} = 0$.
			\item Nem, hiszen $\pi + \pi \neq \pi$.
			\item Igen, hiszen a műveletek sosem mutatnak ki a vektortérből.
			\item Nem, hiszen $(a + b)^2 \neq a^2 + b^2$.
		\end{enumerate}
	}
\end{exercise}

\end{frame}

\begin{frame}
  \frametitle{Vektorterek}
  \framesubtitle{Feladatok III}

  \input{exercise/span.tex}
\end{frame}
