\section{Vektorterek}
\begin{frame}
  \frametitle{Vektorterek}
  \framesubtitle{Definíció}

  \justifying
  Legyen $V$ nem üreshalmaz, és $+; \lambda$ két művelet, valamint $T$ test.
  $(V; +; \lambda)$ a $T$ test feletti vektortér, ha az alábbiak teljesülnek:
  \begin{itemize}
    \def\arraystretch{1.2}
    \item $(V; +)$ Abel csoport:\\[1mm]
          \begin{tabular}{p{35mm} l}
            -- asszociatív:        &
            $\rvec a + (\rvec b + \rvec c) = (\rvec a + \rvec b) + \rvec c$,
            \\
            -- kommutatív:         &
            $\rvec a + \rvec b = \rvec b + \rvec a$,
            \\
            -- létezik zérus elem: &
            $\exists \nvec \in V \text{, melyre } \rvec a + \nvec = \rvec a$,
            \\
            -- létezik inverz:     &
            $\forall \rvec a \text{-re } \exists - \rvec a \text{, hogy } \rvec a + (- \rvec a) = \nvec$.
          \end{tabular}
    \item $(V; \lambda)$-ra pedig igaz:\\[1mm]
          \begin{tabular}{p{35mm} l}
            -- asszociatív:      & $(\alpha \beta) \rvec a = \alpha (\beta \rvec a)$.
            \\
            -- egységelem:       & $\varepsilon \in T \text{-re } \varepsilon\rvec a = \rvec a$,
            \\
            -- disztributivitás: & $\alpha(\rvec a + \rvec b) = \alpha \rvec a + \alpha \rvec b$,
            \\
                                 & $(\alpha + \beta) \rvec a = \alpha \rvec a + \beta \rvec a$.
          \end{tabular}
  \end{itemize}
\end{frame}

\begin{frame}
  \frametitle{Vektorterek}
  \framesubtitle{További definíciók I}

  \begin{block}{Lineáris függőség / függetlenség}
    A $(V; +; \lambda)$ vektortér $\rvec a_1; \rvec a_2; \dots; \rvec a_n$
    vektorait \textbf{lineárisan függő}nek mondjuk, ha a
    \[
      \lambda_1 \rvec a_1 +
      \lambda_2 \rvec a_2 +
      \dots +
      \lambda_n \rvec a_n =
      \nvec
    \]
    vektoregyenletnek létezik a triviálistól különböző megoldása is.
    \\[2mm]
    Ellenkező esetben \textbf{lineárisan független}ek.
  \end{block}

  \begin{block}{Altér}
    Legyen $(V; +; \lambda)$ a $T$ test feletti vektortér, és
    $\emptyset \neq L \subset V$. $L$-t altérnek nevezzük $V$-ben, ha
    $(L; +; \lambda)$ ugyancsak vektortér.
  \end{block}
\end{frame}

\begin{frame}
  \frametitle{Vektorterek}
  \framesubtitle{További definíciók II}

  \begin{block}{Generátorrendszer}
    Legyen $\emptyset \neq G \subset V$. Ekkor $G$ által generált altérnek
    nevezzük azt a legszűkebb alteret, amely tartalmazza $G$-t. Ha ez az
    altér maga $V$, akkor $G$ generátorrendszere $V$-nek. ($\mathcal L(G)=V$)
  \end{block}

  \begin{block}{Bázis}
    A $V$ vektortér egy lineárisan független generátorrendszerét a $V$ bázisának
    hívjuk.
  \end{block}

  \begin{block}{Vektortér dimenziója}
    Végesen generált vektortér dimenzióján a bázisainak közös tagszámát értjük.
  \end{block}
\end{frame}

\begin{frame}
  \frametitle{Vektorterek}
  \framesubtitle{Feladatok I}

  \begin{exercise}{Vektorteres kiskérdések}
  \begin{enumerate}[a)]
    \item Bázist alkot-e az alábbi vektorhármas $\mathbb R^3$-ban?
          \[
            \Big\{\; (1;2;0);\; (2;0;1);\; (0;1;2) \;\Big\}
          \]

    \item Lineárisan független-e a valós együtthatós polinomok vektorterében
          az alábbi vektorhármas?
          \[
            \Big\{\; x^2 - x - 2 ;\; x + 1 ;\; x^2 + x \;\Big\}
          \]

    \item Hány dimenziós vektorteret alkot az azon legfeljebb 5-ödfokú
          polinomok halmaza, melynek elemeire teljesül, hogy a nulladrendű tag
          együtthatója megegyezik a harmadrendű tag együtthatójával?
  \end{enumerate}

  \exsol[11cm]{%
    \begin{enumerate}[a)]
      \item A vektorhármas lineárisan független, ha vegyes szorzatuk nem zérus,
            vagyis $\rvec a \cdot (\rvec b \times \rvec c) \neq 0$:
            \[
              \begin{pmatrix} 1 \\ 2 \\ 0 \end{pmatrix} \cdot \left(
              \begin{pmatrix} 2 \\ 0 \\ 1 \end{pmatrix} \times
              \begin{pmatrix} 0 \\ 1 \\ 2 \end{pmatrix}
              \right)
              =
              \begin{pmatrix} 1 \\ 2 \\ 0 \end{pmatrix} \cdot
              \begin{pmatrix} -1 \\ -4 \\ 2 \end{pmatrix}
              = -1 - 8 = -9 \neq 0
              \text.
            \]
            Megállapíthatjuk, hogy ezen vektorok nem koplanárisak, ebből
            következik, hogy lineárisan függetlenek, vagyis bázist alkotnak
            $\mathbb R^3$-ban.

      \item Ezen vektorhármas \textbf{nem} lineárisan független, hiszen:
            \[
              x^2 - x - 2 = -2(x + 1) + 1(x^2 + x)
              \text.
            \]

      \item A vektortér egy lehetséges bázisa:
            \[
              \Big\{\; 1 + x^3 ;\; x ;\; x^2 ;\; x^4 ;\; x^5 \;\Big\}
              \text.
            \]
            A bázis elemszáma 5, tehát a vektortér 5 dimenziós.
    \end{enumerate}
  }
\end{exercise}

\end{frame}

\begin{frame}
  \frametitle{Vektorterek}
  \framesubtitle{Feladatok II}

  \begin{exercise}{%
    Döntsük el, hogy alteret alkotnak-e az alábbi számhármasok
    $\mathbb R^3$-ban?
  }
  \newcommand{\setmap}[2]{\Big\{\, #1 \,\Big|\, #2 \,\Big\}}
  \begin{enumerate}[a)]
    \item $Q_1 = \setmap{(x_1; x_2; x_3)}{x_1 + x_2 = 0}$
    \item $Q_2 = \setmap{(x_1; x_2; x_3)}{x_1 = \pi}$
    \item $Q_3 = \setmap{(x_1; x_2; x_3)}{x_1 = x_2 = x_3}$
    \item $Q_3 = \setmap{(x_1; x_2; x_3)}{x_1 = (x_2)^2}$
  \end{enumerate}

  \exsol{%
    \begin{enumerate}[a)]
      \item Igen, mert a műveletek sosem mutatnak ki a vektortérből.
            %    , hiszen ha\\
            % $\alpha(x_{11} + x_{12}) = 0$,
            % akkor $\alpha x_{11} + \alpha x_{12} = 0$.
      \item Nem, hiszen $\pi + \pi \neq \pi$.
      \item Igen, hiszen a műveletek sosem mutatnak ki a vektortérből.
      \item Nem, hiszen $(a + b)^2 \neq a^2 + b^2$.
    \end{enumerate}
  }
\end{exercise}

\end{frame}

\begin{frame}
  \frametitle{Vektorterek}
  \framesubtitle{Feladatok III}

  \begin{exercise}{%
    Alkotnak-e generátorrendszert vagy bázist az alábbi vektorok $\mathbb R^2$-ban
  }
  \begin{center}
    \begin{tikzpicture}[ultra thick]
      \foreach \i/\j/\l in {0/0/a,1/0/b,0/1/c,1/1/d}{
          \draw[draw=gray, dashed, rounded corners, thick]
          (\i*5cm,-\j*3cm)
          node[below right] {\l)}
          rectangle ++(4.5cm,-2.5cm)
          coordinate [midway] (\i\j);

          \draw[-latex, cyan!40!black] ($(\i\j)-(0,7.5mm)$)
          coordinate (\i\j)
          -- ++(1.5,0);
        }

      \begin{scope}[-latex, yellow!40!black]
        \draw (00) -- ++(0,1.5);
        \draw (10) -- ++(-1.5/1.41,1.5/1.41);
        \draw (01) -- ++(-1.5/1.41,1.5/1.41);
        \draw (11) -- ++(-1.5,0);

        \draw[red!40!black] (01) -- ++(0.574,1.386);
      \end{scope}
    \end{tikzpicture}
  \end{center}

  \exsol{%
    \begin{enumerate}[a)]
      \item Bázist és generátorrendszert is alkotnak.
      \item Bázist és generátorrendszert is alkotnak.
      \item Csak generátorrendszert alkotnak, hiszen nem lineárisan függetlenek.
      \item Sem bázist, sem generátorrendszert nem alkotnak.
    \end{enumerate}
  }
\end{exercise}

\end{frame}
