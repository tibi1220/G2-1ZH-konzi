\begin{exercise}{%
    Adjuk meg az alábbi lineáris leképezés mátrixát a standard bázisban!
  }
  \[
    \varphi: \mathbb R^3 \rightarrow \mathbb R^3;
    \begin{bmatrix}
      x \\ y \\ z
    \end{bmatrix} \mapsto \begin{bmatrix}
      y + z \\
      x + z \\
      x + y
    \end{bmatrix}
  \]
  Adjuk meg továbbá:
  \begin{itemize}
    \item a képtér és magtér dimenzióját,
    \item a leképezés rangját és defektusát,
    \item a magtér egy tetszőleges elemét,
    \item az $(1; 2; 3)$ pont képét,
    \item a $(2; 2; 2)$ kép ősképét.
  \end{itemize}

  \exsol[7.25cm]{%
    A leképezés mátrixa:
    \[
      \rmat A = \begin{bmatrix}
        0 & 1 & 1 \\
        1 & 0 & 1 \\
        1 & 1 & 0 \\
      \end{bmatrix}
      \text.
    \]

    Határozzuk meg a mátrix rangját a determinánsának számítása segítségével:
    \[
      \det \rmat A = 1 + 1 = 2
      \text.
    \]

    Mivel a mátrix reguláris ($\det \rmat A \neq 0$), ezért rangja maximális
    ($\rg \rmat A = 3$). A leképezés mátrix-reprezentációjának rangja megegyezik
    a leképezés rangjával és a képtér dimenziójával:
    \[
      \rg \rmat A = \rg \varphi = \dim V_2 = 3
      \text.
    \]
    A rang nullitás tételéből pedig következik, hogy:
    \[
      \defect \varphi = \dim \ker \varphi = \dim V_1 - \rg \varphi = 3 - 3 = 0
      \text.
    \]
    Mivel a leképezés defektusa 0, ezért a magtérnek egyetlen egy eleme van,
    mégpedig a nullvektor:
    \[
      \ker \varphi = \Big\{\; \nvec \;\Big\}
      \text.
    \]
    Az $(1;2;3)$ pont képe:
    \[
      \varphi \begin{bmatrix}
        1 \\ 2 \\ 3
      \end{bmatrix} = \begin{bmatrix}
        2 + 3 \\
        1 + 3 \\
        1 + 2
      \end{bmatrix} = \begin{bmatrix}
        5 \\ 4 \\3
      \end{bmatrix}
      \text.
    \]
    A $(2;2;2)$ kép ősképét meghatározhatjuk az $\rmat A$ mátrix invertálásával,
    vagy egy 3 ismeretből és 3 egyenletből álló lineáris egyenletrendszer
    megoldásával.
    \begin{enumerate}
      \newcommand\qadj[4]{\begin{vmatrix}#1&#2\\#3&#4\end{vmatrix}}
      \item Mátrix invertálásos módszer: \\[2mm]
            Határozzuk meg először az $\rmat A$ mátrix adjugáltját:
            \[
              \adj \rmat A = \begin{bmatrix}
                +\qadj{0}{1}{1}{0} & -\qadj{1}{1}{1}{0} & +\qadj{1}{0}{1}{1} \\
                -\qadj{1}{1}{1}{0} & +\qadj{0}{1}{1}{0} & -\qadj{0}{1}{1}{1} \\
                +\qadj{1}{1}{0}{1} & -\qadj{0}{1}{1}{1} & +\qadj{0}{1}{1}{0} \\
              \end{bmatrix} = \begin{bmatrix}
                -1 & 1  & 1  \\
                1  & -1 & 1  \\
                1  & 1  & -1 \\
              \end{bmatrix}
              \text.
            \]
            A mátrix inverze tehát:
            \[
              \rmat A^{-1}
              = \frac{\adj \rmat A}{\det \rmat A}
              = \frac{1}{2} \begin{bmatrix}
                -1 & 1  & 1  \\
                1  & -1 & 1  \\
                1  & 1  & -1 \\
              \end{bmatrix}
              \text.
            \]
            A keresett kép ősképe tehát:
            \[
              \varphi^{-1} \begin{bmatrix}
                2 \\ 2 \\ 2
              \end{bmatrix} = \frac{1}{2} \begin{bmatrix}
                -1 & 1  & 1  \\
                1  & -1 & 1  \\
                1  & 1  & -1 \\
              \end{bmatrix} \begin{bmatrix}
                2 \\ 2 \\ 2
              \end{bmatrix} = \frac{1}{2} \begin{bmatrix}
                -2 + 2 + 2 \\ 2 - 2 + 2 \\ 2 + 2 - 2
              \end{bmatrix} = \frac{1}{2} \begin{bmatrix}
                2 \\ 2 \\ 2
              \end{bmatrix} = \begin{bmatrix}
                1 \\ 1 \\ 1
              \end{bmatrix}
              \text.
            \]

      \item Lineáris egyenletrendszeres megoldás:
            \newcommand{\qgj}[6]{\left[\begin{array}{*{3}{X{6mm}}|X{6mm}}
                  #1 \\#3\\#5
                \end{array}\right]\begin{matrix}#2\\#4\\#6\end{matrix}}
            \begin{align*}
                 & \qgj
              {0 & 1    & 1  & 2}{}
              {1 & 0    & 1  & 2}{}
              {1 & 1    & 0  & 2}{}
              \\
                 & \qgj
              {1 & 0    & 1  & 2}{}
              {0 & 1    & 1  & 2}{}
              {1 & 1    & 0  & 2}{(-S_1)}
              \\
                 & \qgj
              {1 & 0    & 1  & 2}{}
              {0 & 1    & 1  & 2}{}
              {0 & 1    & -1 & 0}{(-S_2)}
              \\
                 & \qgj
              {1 & 0    & 1  & 2 }{}
              {0 & 1    & 1  & 2 }{}
              {0 & 0    & -2 & -2}{(/(-2))}
              \\
                 & \qgj
              {1 & 0    & 1  & 2}{(-S_3)}
              {0 & 1    & 1  & 2}{(-S_3)}
              {0 & 0    & 1  & 1}{}
              \\
                 & \qgj
              {1 & 0    & 0  & 1}{}
              {0 & 1    & 0  & 1}{}
              {0 & 0    & 1  & 1}{}
            \end{align*}
    \end{enumerate}
    Láthatjuk, hogy mindkét módszerrel ugyanazt az eredményt kaptuk.
  }

  \pagebreak\;\makeatletter\ifkeep@@space\;\phantom{alma}\vfill\par\vspace*{6cm}\phantom{alma}\fi\makeatother
\end{exercise}
