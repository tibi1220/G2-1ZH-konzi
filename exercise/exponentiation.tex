\begin{exercise}{Határozzuk meg $\rmat A^{12}$ és $\rmat B^{25}$ mátrixokat!}
  \[
    \rmat A = \begin{bmatrix}
      -8  & 6  \\
      -15 & 11
    \end{bmatrix}
    \hspace{2cm}
    \rmat B = \begin{bmatrix}
      \sqrt{3} / 2 & -1/2         \\
      1/2          & \sqrt{3} / 2
    \end{bmatrix}
  \]

  \exsol[21.85cm]{%
  A hatványozást elvégezni nagyon hosszadalmas lenne nem diagonális
  mátrixon. Transzformáljuk át a mátrixot a saját koordináta-rendszerébe,
  végezzük el ott a hatványozást, majd utána transzformáljuk vissza a
  jelenlegi ko\-or\-di\-ná\-ta-rendszerünkbe. A bázis-transzformáció
  elvégzéséhez szükségünk van a hatványozandó mátrix sajátértékeire és
  sajátvektoraira. Ezeket már korábban meghatároztuk:
  \[
    \lambda_1 = 1
    \text,
    \quad
    \rvec v_1 = \begin{bmatrix}
      2 \\ 3
    \end{bmatrix}
    \text,
    \qquad
    \lambda_2 = 2
    \text,
    \quad
    \rvec v_2 = \begin{bmatrix}
      3 \\ 5
    \end{bmatrix}
    \text.
  \]
  Az eredeti és a diagonális mátrix közötti kapcsolat:
  \[
    \rmat A = \rmat T \mbfLambda \rmat T^{-1}
    \text.
  \]
  A $\rmat T$ transzformációs mátrixot úgy kapjuk meg, hogy a sajátvektorokat,
  mint oszlopvektorokat a mátrix egyes soraiba helyezzük el:
  \[
    \rmat T = \begin{bmatrix}
      2 & 3 \\
      3 & 5
    \end{bmatrix}
    \text.
  \]
  Szükségünk van továbbá ennek a mátrixnak az inverzére is. Ezt korábban már
  meghatároztuk:
  \[
    \rmat T^{-1} = \begin{bmatrix}
      5  & -3 \\
      -3 & 2
    \end{bmatrix}
    \text.
  \]
  A diagonális mátrix elemei maguk a sajátértékek lesznek, mégpedig olyan
  sorrendben, ahogyan a hozzájuk tartozó sajátvektorokat beírtuk a
  transzformációs mátrixba. Ezt számítással ellenőrizhetjük:
  \[
    \mbfLambda
    = \rmat T^{-1} \rmat A \rmat T
    = \begin{bmatrix}
      5  & -3 \\
      -3 & 2
    \end{bmatrix} \begin{bmatrix}
      -8  & 6  \\
      -15 & 11
    \end{bmatrix} \begin{bmatrix}
      2 & 3 \\
      3 & 5
    \end{bmatrix}
    =
    \begin{bmatrix}
      5  & -3 \\
      -6 & 4
    \end{bmatrix} \begin{bmatrix}
      2 & 3 \\
      3 & 5
    \end{bmatrix}
    =
    \begin{bmatrix}
      1 & 0 \\
      0 & 2
    \end{bmatrix}
    \text.
  \]
  A hatványozást itt már könnyen el tudjuk végezni:
  \[
    \mbfLambda^{12}
    = \begin{bmatrix} 1 & 0 \\ 0 & 2 \end{bmatrix}^{12}
    = \begin{bmatrix} 1^{12} & 0 \\ 0 & 2^{12} \end{bmatrix}
    = \begin{bmatrix} 1 & 0 \\ 0 & 4096  \end{bmatrix}
    \text.
  \]
  Transzformáljuk vissza az eredményül kapott mátrixunkat:
  \begin{align*}
    \rmat A^{12} = \rmat T \mbfLambda^{12} \rmat T^{-1}
     & =
    \begin{bmatrix} 2 & 3 \\ 3 & 5 \end{bmatrix}
    \begin{bmatrix} 1 & 0 \\ 0 & 4096 \end{bmatrix}
    \begin{bmatrix} 5  & -3 \\ -3 & 2 \end{bmatrix}
    \\
     & =
    \begin{bmatrix} 2 & 12288 \\ 3 & 20480 \end{bmatrix}
    \begin{bmatrix} 5  & -3 \\ -3 & 2 \end{bmatrix}
    \\
     & =
    \begin{bmatrix}
      -36854 & 24570 \\
      -61425 & 40951
    \end{bmatrix}
    \text.
  \end{align*}

  \tcbline

  Ahelyett, hogy elkezdenénk vadul számolni, álljunk meg egy percre, és
  vizsgáljuk meg az $\rmat B$ mátrixot. Megállapíthatjuk róla, hogy ez egy
  forgatási mátrix, mégpedig a $z$ tengely körül $+30^\circ$-ot forgat:
  \[
    \rmat B = \begin{bmatrix}
      \cos 30^\circ & - \sin 30^\circ \\
      \sin 30^\circ & \cos 30^\circ
    \end{bmatrix}
    \text.
  \]
  A huszonötödik hatványra való emelés azt jelenti, hogy huszonötször forgatunk
  $30^\circ$-kal egymás után, amely 2 teljes körbefordulást ($2 \times 12 \times
    30^\circ = 2 \times 360^\circ$), majd utána még egy $30^\circ$-os forgatást
  jelent, vagyis ezen mátrix huszonötödik hatványa önmaga:
  \[
    \rmat B^{25} = \rmat B
    \text.
  \]
  }
\end{exercise}
