\begin{exercise}{%
		Határozzuk meg az alábbi mátrixok sajátértékeit és sajátvektorait!
		A sajátvektorok hosszai legyenek egységnyiek, valamint az első koordinátájuk
		legyen pozitív!
	}
	\[
		\rmat A = \begin{bmatrix}
			2 & 3 \\
			6 & 5 \\
		\end{bmatrix}
		\hspace{2cm}
		\rmat B = \begin{bmatrix}
			1 & 0 \\
			0 & 1
		\end{bmatrix}
	\]

	\exsol{%
	Az $\rmat A$ mátrix sajátértékeinek meghatározásához írjuk fel a
	karakterisztikus egyenletet:
	\[
		\det \left( \rmat A - \lambda \imat \right) = 0
		\text{.}
	\]
	Számítsuk ki ezen determináns értékét.
	\[
		\begin{vmatrix}
			2 - \lambda & 3          \\
			6           & 5 -\lambda
		\end{vmatrix}
		= (2 - \lambda)(5 - \lambda) - 3 \cdot 6
		= 10 - 7\lambda + \lambda^2 - 18
		= \lambda^2 - 7\lambda - 8
		= (\lambda - 8)(\lambda + 1)
		= 0
		\text{.}
	\]
	Vagyis a sajátértékek: $\lambda_1 = 8$ és $\lambda_2 = -1$.

	\vspace{.66em}
	A sajátvektorokat az alábbi egyenlet segítségével kereshetjük:
	\[
		(\rmat A - \lambda_i \imat) \rvec v_i = 0
		\text{.}
	\]

	A $\lambda_1 = 8$-hoz tartozó sajátvektor:
	\[
		\rmat A - \lambda_1 \imat = \begin{bmatrix}
			2 - 8 & 3     \\
			6     & 5 - 8
		\end{bmatrix} = \begin{bmatrix}
			-6 & 3  \\
			6  & -3
		\end{bmatrix}
		\quad \rightarrow \quad
		\left[\begin{matrix}
				-6 & 3  \\
				6  & -3
			\end{matrix}\right.\left|\begin{matrix}
				\,0 \\ \,0
			\end{matrix}\right]
		\quad \sim \quad
		\left[\begin{matrix}
				-6 & 3 \\
				0  & 0
			\end{matrix}\right.\left|\begin{matrix}
				\,0 \\ \,0
			\end{matrix}\right]
		\text{.}
	\]
	Ezek alapján a koordináták közötti viszony:
	\[
		-6v_{11} + 3v_{12} = 0
		\quad \rightarrow \quad
		v_{12} = 2 v_{11}
		\text{.}
	\]
	A sajátvektor paraméteresen, majd egységhosszúra normálva:
	\[
		\rvec v_1 = t_1 \begin{bmatrix}
			1 \\ 2
		\end{bmatrix}
		\quad
		\rightarrow
		\quad
		\uvec v_1 = \frac{1}{\sqrt{1^2 + 2^2}}\begin{bmatrix}
			1 \\ 2
		\end{bmatrix} = \begin{bmatrix}
			1 / \sqrt{5} \\
			2 / \sqrt{5}
		\end{bmatrix}
		\text{.}
	\]

	A $\lambda_2 = -1$-hoz tartozó sajátvektor:
	\[
		\rmat A - \lambda_2 \imat = \begin{bmatrix}
			2 + 1 & 3     \\
			6     & 5 + 1
		\end{bmatrix} = \begin{bmatrix}
			3 & 3 \\
			6 & 6
		\end{bmatrix}
		\quad \rightarrow \quad
		\left[\begin{matrix}
				3 & 3 \\
				6 & 6
			\end{matrix}\right.\left|\begin{matrix}
				\,0 \\ \,0
			\end{matrix}\right]
		\quad \sim \quad
		\left[\begin{matrix}
				1 & 1 \\
				0 & 0
			\end{matrix}\right.\left|\begin{matrix}
				\,0 \\ \,0
			\end{matrix}\right]
		\text{.}
	\]
	Ezek alapján a koordináták közötti viszony:
	\[
		v_{21} + v_{22} = 0
		\quad \rightarrow \quad
		v_{22} = -v_{21}
		\text{.}
	\]
	A sajátvektor paraméteresen, majd egységhosszúra normálva:
	\[
		\rvec v_2 = t_2 \begin{bmatrix}
			1 \\ -1
		\end{bmatrix}
		\quad
		\rightarrow
		\quad
		\uvec v_2 = \frac{1}{\sqrt{1^2 + 1^2}}\begin{bmatrix}
			1 \\ -1
		\end{bmatrix} = \begin{bmatrix}
			1 / \sqrt{2} \\
			-1 / \sqrt{2}
		\end{bmatrix}
		\text{.}
	\]

	\tcbline

	A $\rmat{B}$ mátrixról ránézésre megállapítható, hogy sajátértékei $\lambda_1
		= \lambda_2 = 1$. Keressük meg a sajátvektorait:
	\[
		\rmat B - \lambda_{12} \imat = \begin{bmatrix}
			1 - 1 & 0     \\
			0     & 1 - 1
		\end{bmatrix} = \begin{bmatrix}
			0 & 0 \\
			0 & 0
		\end{bmatrix}
		\quad \rightarrow \quad
		\left[\begin{matrix}
				0 & 0 \\
				0 & 0
			\end{matrix}\right.\left|\begin{matrix}
				\,0 \\ \,0
			\end{matrix}\right]
		\text.
	\]

	Látható, hogy ennek a lineáris egyenletrendszernek végtelen sok megoldása van.
	A sajátvektorok ennek tudatában paraméteresen:
	\[
		\rvec v = \begin{bmatrix}
			t_1 \\ t_2
		\end{bmatrix} = \begin{bmatrix}
			t_1 \\ 0
		\end{bmatrix} + \begin{bmatrix}
			0 \\ t_2
		\end{bmatrix} = t_1 \begin{bmatrix}
			1 \\ 0
		\end{bmatrix} + t_2 \begin{bmatrix}
			0 \\ 1
		\end{bmatrix}
		\text{.}
	\]
	Az sajátvektorok egységnyi hosszúra normálva:
	\[
		\uvec v_1 = \begin{bmatrix}
			1 \\ 0
		\end{bmatrix}
		\quad \text{és} \quad
		\uvec v_2 = \begin{bmatrix}
			0 \\ 1
		\end{bmatrix}
		\text{.}
	\]
	}
\end{exercise}
