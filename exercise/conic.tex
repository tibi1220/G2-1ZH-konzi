\begin{exercise}{%
    Milyen alakzatot írnak le az alábbi másodfokú görbék?
    Írjuk fel a kanonikus egyenletüket!
    Írjuk fel a főtengelyek irányvektorait!
  }
  \begin{enumerate}[a)]
    \item $x^2 + 10\sqrt{3} xy + 11y^2 = 64$
    \item $7x^2 + 6\sqrt{3} xy + 13y^2 = 64$
  \end{enumerate}

  \exsol{%
    \begin{enumerate}[a)]
      \item $x^2 + 10\sqrt{3} xy + 11y^2= 64$

            Hozzuk az egyenlet bal oldalát mátrixos alakra:
            \[
              x^2 + 10\sqrt{3} xy + 11 y^2 =
              \begin{bmatrix} x & y \end{bmatrix}
              \begin{bmatrix} 1 & 5\sqrt{3} \\ 5\sqrt{3} & 11 \end{bmatrix}
              \begin{bmatrix} x \\ y \end{bmatrix}
              \text.
            \]
            Határozzuk meg a mátrix sajátértékeit:
            \[
              \begin{vmatrix} 1 - \lambda & 5\sqrt{3} \\ 5\sqrt{3} & 11 - \lambda \end{vmatrix}
              = (1 - \lambda)(11 - \lambda) - 75
              = \lambda^2 - 12 \lambda + 11 - 75
              = \lambda^2 - 12 \lambda -64
              = 0
              \text,
            \]
            \[
              (\lambda - 16)(\lambda + 4) = 0
              \quad \rightarrow \quad
              \lambda_{12} = \begin{cases}
                16 \text, \\
                -4 \text.
              \end{cases}
            \]
            Látható, hogy az egyik sajátérték pozitív, a másik pedig negatív,
            vagyis indefinit. Az alakzat ebből következőleg hiperbola.
            Az egyes sajátértékekhez tartozó sajátvektorok:
            \begin{enumerate}[1)]
              \item $\lambda_1 = 16$
                    \[
                      \begin{vmatrix}
                        1 - 16     & 5 \sqrt{3} \\
                        5 \sqrt{3} & 11 - 16
                      \end{vmatrix} \;\sim\; \begin{vmatrix}
                        -15        & 5 \sqrt{3} \\
                        5 \sqrt{3} & -5
                      \end{vmatrix} \;\sim\; \begin{vmatrix}
                        -\sqrt{3} & 1 \\
                        0         & 0
                      \end{vmatrix}
                      \quad \rightarrow \quad
                      \rvec v_1 = t_1 \begin{bmatrix}
                        1 \\ \sqrt{3}
                      \end{bmatrix}
                    \]
              \item $\lambda_2 = -4$
                    \[
                      \begin{vmatrix}
                        1 + 4      & 5 \sqrt{3} \\
                        5 \sqrt{3} & 11 + 4
                      \end{vmatrix} \;\sim\; \begin{vmatrix}
                        5          & 5 \sqrt{3} \\
                        5 \sqrt{3} & 15
                      \end{vmatrix} \;\sim\; \begin{vmatrix}
                        1 & \sqrt{3} \\
                        0 & 0
                      \end{vmatrix}
                      \quad \rightarrow \quad
                      \rvec v_2 = t_2 \begin{bmatrix}
                        -\sqrt{3} \\ 1
                      \end{bmatrix}
                    \]
            \end{enumerate}

            A hiperbola kanonikus egyenlete:
            \[
              \lambda_1 \xi^2 + \lambda_2 \eta^2 = 64
              \quad \rightarrow \quad
              16 \xi^2 - 4 \eta^2 = 64
              \quad \rightarrow \quad
              \frac{\xi^2}{2^2} - \frac{\eta^2}{4^2} = 1
              \text.
            \]

            \begin{center}
              \begin{tikzpicture}[very thick]
                \begin{scope}[xshift=3.43cm, yshift=2.845cm]
                  \draw[-to] (-4,0) -- (4,0) node[below left] {$x$};
                  \draw[-to] (0,-3) -- (0,3) node[below right] {$y$};

                  \begin{scope}[rotate=60, draw=red!40!black]
                    \draw[-to] (-3,0) -- (3,0) node[below right] {$\xi$};
                    \draw[-to] (0,-3) -- (0,3) node[above right] {$\eta$};
                  \end{scope}
                \end{scope}
                \begin{axis}[
                    xticklabels={,,},
                    yticklabels={,,},
                    axis line style={draw=none},
                    tick style={draw=none},
                    ymin=-50/6.86, xmin=-50/5.69,
                    ymax=50/6.86, xmax=50/5.69,
                  ]
                  \addplot +[
                  no markers,
                  raw gnuplot,
                  ultra thick,
                  empty line = jump,
                  cyan!60!black,
                  ] gnuplot {
                      set contour base;
                      set cntrparam levels discrete 0.003;
                      unset surface;
                      set view map;
                      set isosamples 250;
                      splot x**2 + 17.3205081 * x * y + 11 * y**2 - 64;
                    };
                \end{axis}
              \end{tikzpicture}
            \end{center}

            \tcbline
      \item $7x^2 + 6\sqrt{3} xy + 13y^2 = 64$

            Hozzuk az egyenlet bal oldalát mátrixos alakra:
            \[
              7x^2 + 6\sqrt{3} xy + 13y^2 =
              \begin{bmatrix} x & y \end{bmatrix}
              \begin{bmatrix} 7 & 3\sqrt{3} \\ 3\sqrt{3} & 13 \end{bmatrix}
              \begin{bmatrix} x \\ y \end{bmatrix}
              \text.
            \]
            Határozzuk meg a mátrix sajátértékeit:
            \[
              \begin{vmatrix} 7 - \lambda & 3\sqrt{3} \\ 3\sqrt{3} & 13 - \lambda \end{vmatrix}
              = (7 - \lambda)(13 - \lambda) - 27
              = \lambda^2 - 20 \lambda + 91 - 27
              = \lambda^2 - 20 \lambda + 64
              = 0
              \text,
            \]
            \[
              (\lambda - 16)(\lambda - 4) = 0
              \quad \rightarrow \quad
              \lambda_{12} = \begin{cases}
                16 \text, \\
                4 \text.
              \end{cases}
            \]
            Látható, hogy minkét sajátérték pozitív, vagyis a mátrix pozitív
            definit. Az alakzat ebből következőleg ellipszis.
            Az egyes sajátértékekhez tartozó sajátvektorok:
            \begin{enumerate}[1)]
              \item $\lambda_1 = 16$
                    \[
                      \begin{vmatrix}
                        7 - 16     & 3 \sqrt{3} \\
                        3 \sqrt{3} & 13 - 16
                      \end{vmatrix} \;\sim\; \begin{vmatrix}
                        -9         & 3 \sqrt{3} \\
                        3 \sqrt{3} & -3
                      \end{vmatrix} \;\sim\; \begin{vmatrix}
                        -\sqrt{3} & 1 \\
                        0         & 0
                      \end{vmatrix}
                      \quad \rightarrow \quad
                      \rvec v_1 = t_1 \begin{bmatrix}
                        1 \\ \sqrt{3}
                      \end{bmatrix}
                    \]
              \item $\lambda_2 = 4$
                    \[
                      \begin{vmatrix}
                        7 - 4      & 3 \sqrt{3} \\
                        3 \sqrt{3} & 13 - 4
                      \end{vmatrix} \;\sim\; \begin{vmatrix}
                        3          & 3 \sqrt{3} \\
                        3 \sqrt{3} & 9
                      \end{vmatrix} \;\sim\; \begin{vmatrix}
                        1 & \sqrt{3} \\
                        0 & 0
                      \end{vmatrix}
                      \quad \rightarrow \quad
                      \rvec v_2 = t_2 \begin{bmatrix}
                        -\sqrt{3} \\ 1
                      \end{bmatrix}
                    \]
            \end{enumerate}

            Az ellipszis kanonikus egyenlete:
            \[
              \lambda_1 \xi^2 + \lambda_2 \eta^2 = 64
              \quad \rightarrow \quad
              16 \xi^2 + 4 \eta^2 = 64
              \quad \rightarrow \quad
              \frac{\xi^2}{2^2} + \frac{\eta^2}{4^2} = 1
              \text.
            \]

            \begin{center}
              \begin{tikzpicture}[very thick]
                \begin{scope}[xshift=3.43cm, yshift=2.845cm]
                  \draw[-to] (-3,0) -- (3,0) node[below left] {$x$};
                  \draw[-to] (0,-2.5) -- (0,2.5) node[below right] {$y$};

                  \begin{scope}[rotate=60, draw=red!40!black]
                    \draw[-to] (-2.5,0) -- (2.5,0) node[below right] {$\xi$};
                    \draw[-to] (0,-3) -- (0,3) node[above right] {$\eta$};
                  \end{scope}
                \end{scope}
                \begin{axis}[
                    xticklabels={,,},
                    yticklabels={,,},
                    axis line style={draw=none},
                    tick style={draw=none},
                    ymin=-50/6.86, xmin=-50/5.69,
                    ymax=50/6.86, xmax=50/5.69,
                  ]
                  \addplot +[
                  no markers,
                  raw gnuplot,
                  ultra thick,
                  empty line = jump,
                  cyan!60!black,
                  ] gnuplot {
                      set contour base;
                      set cntrparam levels discrete 0.003;
                      unset surface;
                      set view map;
                      set isosamples 250;
                      splot 7 * x**2 + 10.39230485 * x * y + 13 * y**2 - 64;
                    };
                \end{axis}
              \end{tikzpicture}
            \end{center}
    \end{enumerate}
  }
\end{exercise}
