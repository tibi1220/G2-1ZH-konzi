\begin{exercise}{%
    Milyen alakzatot írnak le az alábbi másodfokú görbék?
  }
  \begin{enumerate}[a)]
    \item $5x^2 + 3y^2 + 12xy = -169$
    \item $5x^2 + 3y^2 + 4xy = 49$
  \end{enumerate}

  \exsol{%
    \begin{enumerate}[a)]
      \item $5x^2 + 3y^2 + 12xy = -169$

            Hozzuk az egyenlet bal oldalát mátrixos alakra:
            \[
              5x^2 + 3y^2 + 12xy =
              \begin{bmatrix} x & y \end{bmatrix}
              \begin{bmatrix} 5 & 6 \\ 6 & 3 \end{bmatrix}
              \begin{bmatrix} x \\ y \end{bmatrix}
              \text.
            \]
            Határozzuk meg a mátrix sajátértékeit:
            \[
              \begin{vmatrix} 5 - \lambda & 6 \\ 6 & 3 - \lambda \end{vmatrix}
              = (5 - \lambda)(3 - \lambda) - 36
              = \lambda^2 - 8 \lambda + 15 - 36
              = \lambda^2 - 8 \lambda -21
              = 0
              \text,
            \]
            \[
              \lambda_{12}
              = \frac{8 \pm \sqrt{8^2 + 4 \cdot 1 \cdot 21}}{2}
              = 4 \pm \sqrt{37}
              = \begin{cases}
                - 2,0828 \text, \\
                10.083 \text.
              \end{cases}
            \]
            Látható, hogy az egyik sajátérték pozitív, a másik pedig negatív,
            vagyis indefinit. Az alakzat ebből következőleg hiperbola.


            \tcbline
      \item $5x^2 + 3y^2 + 4xy = 49$

            Hozzuk az egyenlet bal oldalát mátrixos alakra:
            \[
              5x^2 + 3y^2 + 4xy =
              \begin{bmatrix} x & y \end{bmatrix}
              \begin{bmatrix} 5 & 2 \\ 2 & 3 \end{bmatrix}
              \begin{bmatrix} x \\ y \end{bmatrix}
              \text.
            \]
            Határozzuk meg a mátrix sajátértékeit:
            \[
              \begin{vmatrix} 5 - \lambda & 2 \\ 2 & 3 - \lambda \end{vmatrix}
              = (5 - \lambda)(3 - \lambda) - 4
              = \lambda^2 - 8 \lambda + 15 - 4
              = \lambda^2 - 8 \lambda + 11
              = 0
              \text,
            \]
            \[
              \lambda_{12}
              = \frac{8 \pm \sqrt{8^2 - 4 \cdot 1 \cdot 11}}{2}
              = 4 \pm \sqrt{5}
              = \begin{cases}
                1,7639 \text, \\
                6,2361 \text.
              \end{cases}
            \]
            Látható, hogy minkét sajátérték pozitív, vagyis a mátrix pozitív
            definit. Az alakzat ebből következőleg ellipszis.
    \end{enumerate}
  }
\end{exercise}
