\begin{exercise}{Oldjuk meg az alábbi lineáris egyenletrendszert!}
	\[
		\begin{array}{*{7}{c}}
			x_{1}   & + & 4 x_{2} & + & 8 x_{3} & = & 23 \\
			        &   & x_{2}   &   &         & = & 1  \\
			2 x_{1} & + & 4 x_{2} & + & 6 x_{3} & = & 22
		\end{array}
	\]

	\exsol{%
		Oldjuk meg a feladatot mátrix inverziós módszerrel! Írjuk fel az együttható
		mátrixot, az ismeretlenek vektorát és a konstans vektort:
		\[
			\rmat A = \begin{bmatrix}
				1 & 4 & 8 \\
				0 & 1 & 0 \\
				2 & 4 & 6
			\end{bmatrix}
			\text,
			\quad
			\rvec x = \begin{bmatrix}
				x_1 \\ x_2 \\ x_3
			\end{bmatrix}
			\text,
			\quad
			\rvec b = \begin{bmatrix}
				23 \\ 1 \\ 22
			\end{bmatrix}
			\text.
		\]

		Az $\rmat A$ mátrix inverzét már korábban meghatároztuk:
		\[
			\rmat A^{-1} = \begin{bmatrix}
				-3/5 & -4/5 & 4/5   \\
				0    & 1    & 0     \\
				1/5  & -2/5 & -1/10
			\end{bmatrix}
			\text.
		\]

		Az egyenletrendszer megoldása tehát:
		\[
			\rvec x
			= \rmat A^{-1} \rvec b
			= \begin{bmatrix}
				-3/5 & -4/5 & 4/5   \\
				0    & 1    & 0     \\
				1/5  & -2/5 & -1/10
			\end{bmatrix} \begin{bmatrix}
				23 \\ 1 \\ 22
			\end{bmatrix}
			= \begin{bmatrix}
				-3/5 \cdot 23 - 4/5 \cdot 1 + 4/5 \cdot 22 \\
				0 \cdot 23 + 1 \cdot 1 + 0 \cdot 22        \\
				1/5 \cdot 23 - 2/5 \cdot 1 - 1/10 \cdot 22
			\end{bmatrix}
			= \begin{bmatrix}
				3 \\ 1 \\ 2
			\end{bmatrix}
			\text.
		\]

		Tehát az egyes változók értékei: $x_1 = 3$, $x_2 = 1$, és $x_3 = 2$.

		\tcbline

		Oldjuk meg Gauss-eliminációval is az egyenletrendszert:
		\newcommand{\qgj}[6]{\left[\begin{array}{*{3}{X{8mm}}|X{8mm}}
					#1 \\#3\\#5
				\end{array}\right]\begin{matrix}#2\\#4\\#6\end{matrix}}
		\begin{align*}
			   & \qgj
			{1 & 4    & 8   & 23}{}
			{0 & 1    & 0   & 1 }{}
			{2 & 4    & 6   & 22}{(-2S_1)}
			\\
			   & \qgj
			{1 & 4    & 8   & 23 }{(-4S_2)}
			{0 & 1    & 0   & 1  }{}
			{0 & -4   & -10 & -24}{(+4S_2)}
			\\
			   & \qgj
			{1 & 0    & 8   & 19 }{}
			{0 & 1    & 0   & 1  }{}
			{0 & 0    & -10 & -20}{(/(-10))}
			\\
			   & \qgj
			{1 & 0    & 8   & 19}{(-8S_3)}
			{0 & 1    & 0   & 1 }{}
			{0 & 0    & 1   & 2 }{\;}
			\\
			   & \qgj
			{1 & 0    & 0   & 3}{}
			{0 & 1    & 0   & 1}{}
			{0 & 0    & 1   & 2}{}
		\end{align*}
		% Látható, hogy ezzel a módszerrel is ugyan azokat a megoldásokat kaptuk.
	}
\end{exercise}
