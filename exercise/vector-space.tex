\begin{exercise}{%
    Döntsük el, hogy alteret alkotnak-e az alábbi számhármasok
    $\mathbb R^3$-ban?
  }
  \newcommand{\setmap}[2]{\Big\{\, #1 \,\Big|\, #2 \,\Big\}}
  \begin{enumerate}[a)]
    \item $Q_1 = \setmap{(x_1; x_2; x_3)}{x_1 + x_2 = 0}$
    \item $Q_2 = \setmap{(x_1; x_2; x_3)}{x_1 = \pi}$
    \item $Q_3 = \setmap{(x_1; x_2; x_3)}{x_1 = x_2 = x_3}$
    \item $Q_3 = \setmap{(x_1; x_2; x_3)}{x_1 = (x_2)^2}$
  \end{enumerate}

  \exsol{%
    \begin{enumerate}[a)]
      \item Igen, mert a műveletek sosem mutatnak ki a vektortérből.
            %    , hiszen ha\\
            % $\alpha(x_{11} + x_{12}) = 0$,
            % akkor $\alpha x_{11} + \alpha x_{12} = 0$.
      \item Nem, hiszen $\pi + \pi \neq \pi$.
      \item Igen, hiszen a műveletek sosem mutatnak ki a vektortérből.
      \item Nem, hiszen $(a + b)^2 \neq a^2 + b^2$.
    \end{enumerate}
  }
\end{exercise}
