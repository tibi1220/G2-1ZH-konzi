\begin{exercise}{%
    Mi lesz a $P (2;6;8)$ pont képe, ha hattatjuk rá az alábbi leképezéseket?
  }
  \begin{itemize}
    \item először tükrözzük az $x$ tengelyre,
    \item majd az $y$ tengely körül $-60^\circ$-kal forgatjuk,
    \item végül $y$ irányban a $-5$-szörösére nyújtjuk.
  \end{itemize}

  \exsol{%
    \[
      \begin{array}{rc}
         & \left[\begin{array}{X{2cm}} 2 \\ 6 \\ 8 \end{array}\right]
        \\
        \left[\begin{array}{*{3}{X{22mm}}}
                  1 & 0  & 0  \\
                  0 & -1 & 0  \\
                  0 & 0  & -1
                \end{array}\right]
         & \left[\begin{array}{X{2cm}} 2 \\ -6 \\ -8 \end{array}\right]
        \\
        \left[\begin{array}{*{3}{X{22mm}}}
                  \cos (-60^\circ)  & 0 & \sin (-60^\circ) \\
                  0                 & 1 & 0                \\
                  -\sin (-60^\circ) & 0 & \cos (-60^\circ)
                \end{array}\right]
         & \left[\begin{array}{X{2cm}} 1 + 4 \sqrt{3} \\ -6 \\ \sqrt{3} - 4 \end{array}\right]
        \\
        \left[\begin{array}{*{3}{X{22mm}}}
                  1 & 0  & 0 \\
                  0 & -5 & 0 \\
                  0 & 0  & 0
                \end{array}\right]
         & \left[\begin{array}{X{2cm}} 1 + 4 \sqrt{3} \\ 30 \\ \sqrt{3} - 4 \end{array}\right]
      \end{array}
    \]
  }
\end{exercise}
