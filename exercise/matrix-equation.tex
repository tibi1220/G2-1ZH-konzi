\begin{exercise}{Oldjuk meg az alábbi mátrix-egyenleteket!}
	\[
		\rmat A = \begin{bmatrix}
			2 & 3 \\
			3 & 5
		\end{bmatrix}
		\hspace{1cm}
		\rmat B = \begin{bmatrix}
			1 & 2 \\
			3 & 4 \\
			6 & 3
		\end{bmatrix}
		\hspace{1cm}
		\rmat Q = \begin{bmatrix}
			\sqrt{3}/2 & -1/2       \\
			1/2        & \sqrt{3}/2
		\end{bmatrix}
	\]
	\begin{multicols}{2}
		\begin{enumerate}
			\item $\rmat X \cdot \rmat A = \rmat B$
			\item $\rmat A \cdot \rmat X = \rmat B$
			\item $2(\rmat A + \rmat X) = 3(\rmat A^{-1} + \rmat X)$
			\item $\rmat X = \rmat A \cdot \rmat Q^{6}$
		\end{enumerate}
	\end{multicols}

	\exsol{
		\begin{enumerate}
			\item $\rmat X \cdot \rmat A = \rmat B$

			      Rendezzük $\rmat X$-re az egyenletet, vagyis szorozzuk meg az
			      egyenlet mindkét oldalát $\rmat A$ inverzével. Ekkor az alábbi
			      egyenletet kapjuk:
			      \[
				      \rmat X = \rmat B \cdot \rmat A^{-1}
				      \text{.}
			      \]
			      Az $\rmat A$ mátrix inverzét már korábban meghatároztuk.
			      Az egyenletbe behelyettesítve:
			      \[
				      \rmat X
				      = \rmat B \cdot\rmat A^{-1} = \begin{bmatrix}
					      1 & 2 \\
					      3 & 4 \\
					      6 & 3
				      \end{bmatrix} \begin{bmatrix}
					      5  & -3 \\
					      -3 & 2
				      \end{bmatrix} = \begin{bmatrix}
					      5 - 6   & -3 + 4  \\
					      15 - 12 & -9 + 8  \\
					      30-9    & -18 + 6
				      \end{bmatrix} = \begin{bmatrix}
					      -1 & 1   \\
					      3  & -1  \\
					      21 & -12
				      \end{bmatrix}
				      \text.
			      \]

			      \tcbline
			\item $\rmat A \cdot \rmat X = \rmat B$
			      Ez az egyenlet nem megoldható.

			      \tcbline
			\item $2(\rmat A + \rmat X) = 3(\rmat A^{-1} + \rmat X)$

			      \tcbline
			\item $\rmat X = \rmat A \cdot \rmat Q^{6}$
		\end{enumerate}
	}
\end{exercise}
